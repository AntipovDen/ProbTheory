\documentclass[12pt]{article}
\usepackage[utf8]{inputenc} 
\usepackage[russian]{babel}
\usepackage{amssymb}
\usepackage{amsmath}

\newcommand\N{\mathbb{N}}
\newcommand\Z{\mathbb{Z}}
\newcommand\R{\mathbb{R}}
\newcommand\eps{\varepsilon}
\DeclareMathOperator{\Bin}{Bin}
\DeclareMathOperator{\Geom}{Geom}
\DeclareMathOperator{\Exp}{Exp}
\DeclareMathOperator{\pow}{pow}
\DeclareMathOperator{\Bern}{Bern}
\DeclareMathOperator{\Var}{Var}

\title{Проекты по теории вероятностей}

\begin{document}
\maketitle

В данном документе описаны задания для проектов по теории вероятностей, успешное и качественное выполнение которых гарантирует автомат за экзамен. Над каждым проектом, описанном в этом документе, будет работать группа из двух студентов. Запись происходит через гуглоформу (ссылка в чате). Для записи необходимо иметь минимум два выхода к доске (у обоих участников группы). Проекты будут распределены случайным образом в ближайший четверг 13 мая. При распределении вероятность выбора пары на проект будет пропорциональна суммарному числу выходов к доске у обоих участников данной пары.

Вы также можете предлагать свои собственные проекты, они будут присвоены только вам в случае одобрения. Возможное число участников предложенного проекта может варьироваться в зависимости от моей оценки его сложности.

\textbf{Дедлайн} по предоставлению результатов всех проектов --- экзамен.

\section*{Проект 1. Негативный дрифт}

Дрифт-анализ --- это набор инструментов для получения оценок на время, за которое случайный процесс достигнет какого-то значения. В самой общей формулировке, мы рассматриваем процесс $\{X_t\}_{t \in \N}$, который начинается с $X_0 = a$, и мы хотим найти время $T$, когда впервые выполняется $X_T = b$. Почти все дрифт-теоремы требуют от нас оценки $E[X_{t + 1} - X_{t} \mid X_0, \dots, X_t]$, и на ее основе дают нам оценки на $T$. Так как $T$ является случайной величиной, дрифт теоремы либо дают нам оценки на ее матожидание, либо оценки вероятности того, что $T$ больше или меньше какого-то числа. Наиболее полный обзор существующих дрифт-теорем на сегодняшний день можно найти статье~\cite{Lengler17} (выложена в свободный доступ на arxiv.org).

Дрифт-анализ берет свое начало с так называемого негативного дрифта, представленного в~\cite{Hajek82}. Однако в той статье нет четко сформулированной теоремы, которую можно было бы использовать на практике. В значительно более поздней статье~\cite{OlivetoW11} была сформулирована более применимая на практике теорема, однако нет четкого доказательства, что она следует из того, что написано в~\cite{Hajek82}. Более того, несмотря на то, что статья~\cite{OlivetoW11} прошла через процесс рецензирования, после представления данного результата на конференции в ней были найдены ошибки. Поэтому была выпущена исправленная версия статьи~\cite{OlivetoW12}, выложенная на arxiv.org и не проходившая рецензирование.

Таким образом, на данный момент не существует надежного источника, содержащего применимую на практике негативную дрифт теорему. Задачей данного проекта является четкое доказательство Теоремы 1 из~\cite{OlivetoW12} (или Теоремы 2 из~\cite{OlivetoW11}) на основе оригинальной статьи~\cite{Hajek82}.

Есть вероятность, что у вас просто получится затрекать эту версию теоремы из более ранних источников, что тоже ок.

\section*{Проект 2. Аддитивный дрифт с хвостовыми оценками}

Одной из простейших дрифт теорем является аддитивная дрифт-теорема. Упрощенно говоря, она звучит так. Пусть есть процесс $\{X_t\}_{t \in \N_0}$, такой, что он начинает с $X_0 = a$ и никогда не превышает какой-то $b > a$. Мы хотим найти время $T$, когда впервые $X_T = b$. Если для любого $s < b$ выполняется $E[X_{t + 1} - X_t \mid X_t = s] \ge \delta$, то $E[T] = \frac{b - a}{\delta}$.

Эта простая теорема позволила доказать огромное множество полезных результатов, например, в области эволюционных вычислений. Однако она дает нам информацию только про матожидание $T$, но не про его концентрацию.

Первая удачная попытка формулировки аддитивной дрифт-теоремы, которая бы давала некоторые оценки на концентрацию $T$, была сделана в~\cite{Kotzing16}. Однако она касается только процессов с ограничением изменения (как в неравенстве Хефдинга), либо суб-Гауссовых процессов, а доказать, что процесс суб-Гауссовый не всегда просто. 

Схожий результат был получен в~\cite{AntipovDK19} (Теорема 2.9). Однако он сформулирован только для целочисленного процесса (то есть все $X_t \in \Z$), а также он  дает только верхние оценки, хотя аналогичным способом можно получить и нижние оценки. Плюс ко всему, он основан на непроверенной версии негативной дрифт теоремы, описанной в первом проекте.

В данном проекте требуется устранить недостатки последнего результата, а именно расширить его на вещественно-значный процесс $\{X_t\}_{t \in \N}$, а также получить аналогичную теорему для нижних хвостовых оценок.

\section*{Проект 3. Дрифт с переменной ценой шага.}

Для некоторых случайных процессов может существовать еще один процесс (случайный или детерминированный), от которого рассматриваемый процесс зависит. Например, у нас есть $\{X_t\}_{t \in \N_0}$ и $\{Y_t\}_{t \in \N_0}$. Причем $Y_t$ может рассматриваться как некоторая плата, в зависимости от которой процесс $X_t$ может вести себя по-разному. Допустим, мы рассматриваем процесс $X_t$, который стартует с $X_0 = a$ и хотим, чтобы он достиг какого-то значения $b > a$. Мы можем выбирать $Y_t$, причем мы знаем, что чем больше мы заплатим, тем больше будет ожидаемый прогресс $X_t$ в сторону $b$. Однако в данной формулировке нас интересует не время $T$, когда мы впервые добьемся $X_T \ge b$, а суммарная плата до этого времени $T$, то есть $S = \sum_{t = 0}^T Y_t$. 

Задача данного проекта --- сформулировать и доказать дрифт-теорему, позволяющую оценить $S$, когда нам известна функция $f(y) = E[X_{t + 1} - X_t \mid Y_t = y]$. Данное задание является довольно творческим, так как оставляет вам свободу по накладыванию ограничений на $f(y)$ и на стратегию выбора $Y_t$. Примеры подобной теоремы (но для весьма специфического случая) можно найти в~\cite{AntipovDK19}, в разделах 6 и 4 (лучше смотреть эти разделы именно в этом порядке). Там были даны нужные оценки в случае, когда $f(y)$ --- линейная функция от $y$, а стратегия выбора $Y_t$ --- любая.


\bibliographystyle{alpha}
\bibliography{bibliography.bib}

\end{document}