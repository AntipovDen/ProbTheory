\documentclass[12pt]{article}

\usepackage{a4wide}

\usepackage[utf8]{inputenc} 
\usepackage[russian]{babel}
\usepackage{amssymb}
\usepackage{amsmath}
\usepackage{amsthm}
\usepackage{pgfplots}
\usepgfplotslibrary{statistics,patchplots}
\usetikzlibrary{decorations.pathreplacing,calc,tikzmark, patterns,arrows.meta,snakes}
\pgfmathdeclarefunction{gauss}{3}{%
  \pgfmathparse{1/(#3*sqrt(2*pi))*exp(-((#1-#2)^2)/(2*#3^2))}%
}
\pgfmathdeclarefunction{gauss2d}{6}{%
  \pgfmathparse{1/(#3*#6*2*pi)*exp(-((#1-#2)^2)/(2*#3^2) - ((#4-#5)^2)/(2*#6^2))}%
}
\pgfmathdeclarefunction{circle}{1}{%
\pgfmathparse{2/pi*sqrt(1 - #1^2)}%
}

\usepackage{xspace}

\usepackage{mathtools}
\usepackage{cite}
\usepackage{array}
\usepackage{multirow}
\usepackage{tabularx}
\usepackage{bbm}

\newcommand\N{\mathbb{N}}
\newcommand\R{\mathbb{R}}
\newcommand\F{\mathcal{F}}
\newcommand\eps{\varepsilon}
\newcommand\one{\mathbbm{1}}
\DeclareMathOperator{\Bin}{Bin}
\DeclareMathOperator{\Geom}{Geom}
\DeclareMathOperator{\Pois}{Poisson}
\DeclareMathOperator{\pow}{pow}
\DeclareMathOperator{\Bern}{Bern}
\DeclareMathOperator{\Exp}{Exp}
\DeclareMathOperator{\Var}{Var}
\DeclareMathOperator{\Cov}{Cov}
\DeclareMathOperator{\sign}{sign}

\newtheorem{theorem}{Теорема}
\newtheorem{lemma}{Лемма}

\title{Экзамен по теорверу: процедура и вопросы}

\begin{document}
\maketitle

\section*{Процедура дистанционного экзамена}

\begin{itemize}
  \item Начало экзамена 1 июля в 12:00. Ровно в это время в телеграм-канал будет прислан pdf-файл с билетами для всех, сдающих экзамен онлайн. Это будет сделано с помощью отложенного сообщения, которое может прийти с задержкой до одной минуты.
  \item В это же время стартует наш стандартный зум.
  \item Каждый билет содержит два вопроса (вопросы перечислены в конце файла): один из первого модуля и один из второго. Вопросы выбираются случайным образом.
  \item У вас есть ровно час на подготовку подробного ответа на два вопроса вашего билета. Свои ответы вы засылаете через гуглоформу до 13:00 (ссылка на гуглоформу будет прислана вместе с билетами). Дедлайн строгий, учтите. Лучше что-то не дописать, чем отправить поздно. Отправлять свои ответы можно в любом формате: фотки или pdf. Но pdf предпочтительнее --- мне так проще выкачивать файлы.
  \item После отправки ответов мы делаем перерыв 10 минут и в 13:10 начинаем устную часть экзамена.
  \item На устной части готовые отвечать удаляются со мной в отдельную комнату в зуме. Там я смотрю билеты, и если у меня возникают сомнения в том, что билет написан с пониманием, то задаю вопросы по билету. Хорошо написанный билет гарантирует вам 10 баллов за экзамен. Плохо написанный билет может дать максимум 5 баллов (когда, например, написан только один вопрос) и заканчивает ваш экзамен.
  \item В случае хорошо написанного билета мы продолжаем бегать по теормину. Я задаю (псевдо-)случайные вопросы по всему курсу. Каждые два отвеченных вопроса дают по 5 баллов, максимум 15. Есть максимум три шанса ответить неправильно (четвертый неправильный ответ заканчивает экзамен).
  \item Ответы на вопросы по теормину требуют только формулировок. То есть, например, надо просто рассказать, как выглядят границы Чернова, как формулируется слабый закон больших чисел или что такое гипер-геометрическое распределение. Но подробностей, как в ответе на билет, я требовать не буду.
  \item Оставшиеся 5 баллов можно заработать, решив задачку (на нее дается время до конца экзамена).
  \item Во время устной части (кроме решения задачки) нельзя ничем пользоваться, поэтому на протяжении всего нашего разговора ваша камера должна быть включена.
  \item Если следующего желающего отвечать приходится ждать более двух минут, то экзамен заканчивается для всех. Не прошедшим устную часть будет выставлена оценка на основе их билетов (до 10 баллов).
  \item Замечу, что в зависимости от ``плавания'' в ответах на вопросы, я могу поставить не только оценку, кратную 5, но и добавить или отнять пару баллов. 
  \item Я не буду вести запись экзамена, однако не буду против, если вы захотите провести запись с условием, что она не будет выложена в открытый доступ куда-либо, и что вы предупредите меня о том, что вы ее ведете. 
\end{itemize}


\section*{Процедура очного экзамена}

В целом процедура повторяет процедуру дистанционного экзамена, все отличия описаны ниже.
\begin{itemize}
  \item Начало экзамена 2 июля в 10:00. В это время я пришлю в телеграм-канал билеты и ссылку на гуглоформу, через которую можно засылать ответы на билеты.
  \item До 11:00 вы должны либо отправить ответы на билеты через гуглоформу, либо написать их на бумажке и сдать мне. После этого все покидают аудиторию. 
  \item В 11:10 начнем устную часть экзамена, она тоже будет проводиться тет-а-тет в аудитории. Опять же, я не против ведения записи при тех же условиях, что и на онлайн-экзамене.
\end{itemize}

\newpage

\section*{Вопросы к экзамену}

\begin{enumerate}
  \item Вероятностное пространство $(\Omega, \Sigma, \Pr)$: определение, примеры, элементарные свойства меры. Интерпретация вероятности в реальном мире и взаимодействие теорвера с реальным миром.
  \item Условная вероятность: определение и обоснование определения. Примеры. Правило умножения вероятностей, формула полной вероятности, формула Байеса.
  \item Независимость: интуитивное и формальное определения. Независимость дополнений. Независимость множества событий: попарная и по совокупности. Неэквивалентность этих двух видов независимости.
  \item Определение случайной величины (с.в.). Дискретные с.в. и функция вероятностей. Примеры дискретных распределений и их функции вероятностей: равномерное распределение, распределений Бернулли, биномиальное распределение, геометрическое распределение.
  \item Определение и интерпретация матожидания дискретной с.в. Матожидание равномерного дискретного распределения и распределения Бернулли. Элементарные свойства матожидания, матожидание функции от с.в. и линейность матожидания (одной с.в.).
  \item Дисперсия. Определение, свойства. Дисперсия с.в. Бернулли и дискретной равномерной с.в.
  \item С.в., условная на событии: свойства, примеры. Теорема о полном матожидании. Беспамятство геометрического распределения.
  \item Случайные вектора. Совместная и маргинальная функции вероятности, примеры. Линейность матожидания (нескольких с.в.). Матожидание распределения Бернулли.
  \item С.в., условная на другой с.в. Условные случайные векторы, правило умножения, формула полной вероятности. Условное матожидание, формула полного матожидания.
  \item Независимость и условная независимость с.в. Матожидание и дисперсия независимых с.в. Дисперсия биномиального распределения.
  \item Гипер-геометрическое и степенное распределения: определение, матожидание, дисперсия
  \item Непрерывные с.в. Определение, свойства и интерпретация плотности вероятности. Матожидание и дисперсия непрерывных с.в., их элементарные свойства.
  \item Непрерывное равномерное и экспоненциальное распределения: определение, матожидание, дисперсия. Функция распределения и ее свойства.
  \item Нормальное распределение. Определение, матожидание, дисперсия. Определение значения функции распределения нестандартного нормального распределения с помощью таблиц для стандартного нормального распределения.
  \item Условная плотность вероятности и условное матожидание непрерывных с.в. Беспамятство экспоненциального распределения. Полные вероятность и матожидание для непрерывных с.в.
  \item Смешанные распределения: примеры. Векторы непрерывных с.в.: совместная плотность вероятности, совместная функция вероятности, функции от непрерывных с.в., линейность матожидания
  \item Непрерывная с.в., условная на другой непрерывной с.в. Условная плотность вероятности, правило умножения плотностей, полная вероятности и полное матожидание.
  \item Независимость непрерывных с.в. Матожидание и дисперсия независимых с.в. Пример зависимых с.в. Независимые нормальные распределения.
  \item Формула Байеса для с.в., примеры, когда мы наблюдаем непрерывную с.в. и оцениваем дискретную и наоборот.
  \item Линейные и нелинейные преобразования с.в.
  \item Функции нескольких с.в. Случай независимых с.в. Имитация одним распределением другого распределения.
  \item Ковариация, свойства, коэффициент корреляции.
  \item Условное матожидание и условная дисперсия как случайная величина. Матожидание условных матожидания и дисперсии. Матожидание и дисперсия суммы случайного числа с.в. (случай независимости числа слагаемых от слагаемых).
  \item Равенство Вальда
  \item Неравенство Маркова. Неравенства для целочисленных с.в. Неравенство Чебышева.
  \item Неравенство Хёффдинга. Неравенства Беннета и Бернштейна (без доказательств).
  \item Границы Чернова. 
  \item Сходимость по распределению, по вероятности и почти наверное.
  \item Слабый и сильный законы больших чисел.
  \item Центральная предельная теорема: классическая и версии Линдеберга и Ляпунова. Скорость сходимости к нормальному распределению (теорема Берри-Эссена).
  \item Матожидание, условное на $\sigma$-алгебре. Примеры и свойства. Фильтрации.
  \item Мартингалы, суб- и супер-мартингалы. Лемма про $E[X_{n + k} \mid \F_n]$. Выпуклая (вогнутая) функция от мартингала
  \item Предсказуемые с.в., upcrossing inequality, декомпозиция Дуба.
  \item Ветвящиеся процессы. Неравенства Азумы и МакДиармида.
  \item Процесс Бернулли. Основные свойства.
  \item Слияние и разделение процесса Бернулли и его аппроксимация при маленьком элементарном временном интервале.
  \item Процесс Пуассона. Определение и основные свойства. Распределение Пуассон, сумма двух с.в. Пуассона.
  \item Слияние и разделение процесса Пуассона. Парадокс наблюдателя.
  \item Определение марковского процесса. Марковские процессы с дискретным временем и конечным числом состояний.
  \item Цепи Маркова. Переход через $n$ шагов. Стартовые состояния и матрица переходов.
  \item Возвратные и невозвратные состояния. Периодические состояния. Стационарное распределение, условие единственности. 
  \item Частота посещений состояния и перехода в марковской цепи. Демографический процесс. Скорость сходимости к стационарному распределению.
  \item Задача Эрланга о пропускной способности телефонной сети.
  \item Поглощающие состояния. Время до поглощения. Время до прохода через возвратное состояние.
\end{enumerate}

\end{document}
