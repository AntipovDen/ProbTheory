\documentclass[12pt]{article}
\usepackage[utf8]{inputenc} 
\usepackage[russian]{babel}
\usepackage{amssymb}
\usepackage{amsmath}
\usepackage{a4wide}
\usepackage{hyperref}

\newcommand\N{\mathbb{N}}
\newcommand\R{\mathbb{R}}
\newcommand\eps{\varepsilon}
\DeclareMathOperator{\Bin}{Bin}
\DeclareMathOperator{\Geom}{Geom}
\DeclareMathOperator{\Exp}{Exp}
\DeclareMathOperator{\pow}{pow}
\DeclareMathOperator{\Bern}{Bern}
\DeclareMathOperator{\Var}{Var}
\DeclareMathOperator{\Cov}{Cov}
\newcommand\F{\mathcal{F}}

\begin{document}
\pagenumbering{gobble} 

\section{Белодедова Алина Сергеевна}

\begin{itemize}
  \item Непрерывные с.в. Определение, свойства и интерпретация плотности вероятности. Матожидание и дисперсия непрерывных с.в., их элементарные свойства.
  \item Слабый и сильный законы больших чисел.
\end{itemize}

\section{Бородин Евгений Сергеевич}

\begin{itemize}
  \item Равенство Вальда
  \item Слияние и разделение процесса Пуассона. Парадокс наблюдателя.
\end{itemize}

\section{Вихнин Фёдор Алексеевич}

\begin{itemize}
  \item Условная плотность вероятности и условное матожидание непрерывных с.в. Беспамятство экспоненциального распределения. Полные вероятность и матожидание для непрерывных с.в.
  \item Неравенство Маркова. Неравенства для целочисленных с.в. Неравенство Чебышева.
\end{itemize}

\section{Гарипов Роман Исмагилович}

\begin{itemize}
  \item Линейные и нелинейные преобразования с.в.
  \item Границы Чернова. 
\end{itemize}

\section{Гарипов Эмиль Исмагилович}

\begin{itemize}
  \item Ковариация, свойства, коэффициент корреляции.
  \item Возвратные и невозвратные состояния. Периодические состояния. Стационарное распределение, условие единственности. 
\end{itemize}

\section{Гусев Владислав Сергеевич}

\begin{itemize}
  \item С.в., условная на событии: свойства, примеры. Теорема о полном матожидании. Беспамятство геометрического распределения.
  \item Процесс Бернулли. Основные свойства.
\end{itemize}

\section{Давыдов Артём Вадимович}

\begin{itemize}
  \item Определение случайной величины (с.в.). Дискретные с.в. и функция вероятностей. Примеры дискретных распределений и их функции вероятностей: равномерное распределение, распределений Бернулли, биномиальное распределение, геометрическое распределение.
  \item Процесс Бернулли. Основные свойства.
\end{itemize}

\section{Ибрахим Ахмад Махджуб}

\begin{itemize}
  \item Вероятностное пространство $(\Omega, \Sigma, \Pr)$: определение, примеры, элементарные свойства меры. Интерпретация вероятности в реальном мире и взаимодействие теорвера с реальным миром.
  \item Процесс Пуассона. Определение и основные свойства. Распределение Пуассон, сумма двух с.в. Пуассона.
\end{itemize}

\section{Кирсанов Ярослав Николаевич}

\begin{itemize}
  \item Ковариация, свойства, коэффициент корреляции.
  \item Слабый и сильный законы больших чисел.
\end{itemize}

\section{Кучма Андрей Андреевич}

\begin{itemize}
  \item С.в., условная на другой с.в. Условные случайные векторы, правило умножения, формула полной вероятности. Условное матожидание, формула полного матожидания.
  \item Неравенство Маркова. Неравенства для целочисленных с.в. Неравенство Чебышева.
\end{itemize}

\section{Лабазов Артем Александрович}

\begin{itemize}
  \item Смешанные распределения: примеры. Векторы непрерывных с.в.: совместная плотность вероятности, совместная функция распределения, функции от непрерывных с.в., линейность матожидания
  \item Частота посещений состояния и перехода в марковской цепи. Демографический процесс. Скорость сходимости к стационарному распределению.
\end{itemize}

\section{Малько Егор Александрович}

\begin{itemize}
  \item Определение и интерпретация матожидания дискретной с.в. Матожидание равномерного дискретного распределения и распределения Бернулли. Элементарные свойства матожидания, матожидание функции от с.в. и линейность матожидания (одной с.в.).
  \item Процесс Бернулли. Основные свойства.
\end{itemize}

\section{Мухамеджанов Салават Маратович}

\begin{itemize}
  \item Вероятностное пространство $(\Omega, \Sigma, \Pr)$: определение, примеры, элементарные свойства меры. Интерпретация вероятности в реальном мире и взаимодействие теорвера с реальным миром.
  \item Частота посещений состояния и перехода в марковской цепи. Демографический процесс. Скорость сходимости к стационарному распределению.
\end{itemize}

\section{Надуткин Федор Максимович}

\begin{itemize}
  \item Гипер-геометрическое и степенное распределения: определение, матожидание, дисперсия
  \item Процесс Бернулли. Основные свойства.
\end{itemize}

\section{Нестеренко Виктор Евгеньевич}

\begin{itemize}
  \item Ковариация, свойства, коэффициент корреляции.
  \item Цепи Маркова. Переход через $n$ шагов. Стартовые состояния и матрица переходов.
\end{itemize}

\section{Пак Александр Владимирович}

\begin{itemize}
  \item Независимость и условная независимость с.в. Матожидание и дисперсия независимых с.в. Дисперсия биномиального распределения.
  \item Матожидание, условное на $\sigma$-алгебре. Примеры и свойства. Фильтрации.
\end{itemize}

\section{Синяченко Никита Романович}

\begin{itemize}
  \item Условная вероятность: определение и обоснование определения. Примеры. Правило умножения вероятностей, формула полной вероятности, формула Байеса.
  \item Слабый и сильный законы больших чисел.
\end{itemize}

\section{Сластин Александр Андреевич}

\begin{itemize}
  \item Ковариация, свойства, коэффициент корреляции.
  \item Предсказуемые с.в., upcrossing inequality, декомпозиция Дуба.
\end{itemize}

\section{Ушков Даниил Анатольевич}

\begin{itemize}
  \item С.в., условная на событии: свойства, примеры. Теорема о полном матожидании. Беспамятство геометрического распределения.
  \item Ветвящиеся процессы. Неравенства Азумы и МакДиармида.
\end{itemize}

\section{Черемхина Татьяна Александровна}

\begin{itemize}
  \item Вероятностное пространство $(\Omega, \Sigma, \Pr)$: определение, примеры, элементарные свойства меры. Интерпретация вероятности в реальном мире и взаимодействие теорвера с реальным миром.
  \item Предсказуемые с.в., upcrossing inequality, декомпозиция Дуба.
\end{itemize}

\section{Чернацкий Евгений Геннадьевич}

\begin{itemize}
  \item Непрерывное равномерное и экспоненциальное распределения: определение, матожидание, дисперсия. Функция распределения и ее свойства.
  \item Определение марковского процесса. Марковские процессы с дискретным временем и конечным числом состояний.
\end{itemize}

\section{Шашуловский Артем Владимирович}

\begin{itemize}
  \item Условная вероятность: определение и обоснование определения. Примеры. Правило умножения вероятностей, формула полной вероятности, формула Байеса.
  \item Возвратные и невозвратные состояния. Периодические состояния. Стационарное распределение, условие единственности. 
\end{itemize}

\section{Шик Алексей Александрович}

\begin{itemize}
  \item Непрерывная с.в., условная на другой непрерывной с.в. Условная плотность вероятности, правило умножения плотностей, полная вероятности и полное матожидание.
  \item Матожидание, условное на $\sigma$-алгебре. Примеры и свойства. Фильтрации.
\end{itemize}




\end{document}