\documentclass[12pt]{article}
\usepackage[utf8]{inputenc} 
\usepackage[russian]{babel}
\usepackage{amssymb}
\usepackage{amsmath}
\usepackage{a4wide}
\usepackage{hyperref}

\newcommand\N{\mathbb{N}}
\newcommand\R{\mathbb{R}}
\newcommand\eps{\varepsilon}
\DeclareMathOperator{\Bin}{Bin}
\DeclareMathOperator{\Geom}{Geom}
\DeclareMathOperator{\Exp}{Exp}
\DeclareMathOperator{\pow}{pow}
\DeclareMathOperator{\Bern}{Bern}
\DeclareMathOperator{\Var}{Var}
\DeclareMathOperator{\Cov}{Cov}
\newcommand\F{\mathcal{F}}

\begin{document}
\pagenumbering{gobble} 
\section{Алехин Артем Александрович}

\begin{itemize}
  \item Условное матожидание и условная дисперсия как случайная величина. Матожидание условных матожидания и дисперсии. Матожидание и дисперсия суммы случайного числа с.в. (случай независимости числа слагаемых от слагаемых).
  \item Задача Эрланга о пропускной способности телефонной сети.
\end{itemize}

\section{Андриянов Кирилл Романович}

\begin{itemize}
  \item Условная вероятность: определение и обоснование определения. Примеры. Правило умножения вероятностей, формула полной вероятности, формула Байеса.
  \item Матожидание, условное на $\sigma$-алгебре. Примеры и свойства. Фильтрации.
\end{itemize}

\section{Боже Илона Яновна}

\begin{itemize}
  \item Независимость: интуитивное и формальное определения. Независимость дополнений. Независимость множества событий: попарная и по совокупности. Неэквивалентность этих двух видов независимости.
  \item Мартингалы, суб- и супер-мартингалы. Лемма про $E[X_{n + k} \mid \F_n]$. Выпуклая (вогнутая) функция от мартингала
\end{itemize}

\section{Бородачев Сергей Игоревич}

\begin{itemize}
  \item Определение случайной величины (с.в.). Дискретные с.в. и функция вероятностей. Примеры дискретных распределений и их функции вероятностей: равномерное распределение, распределений Бернулли, биномиальное распределение, геометрическое распределение.
  \item Предсказуемые с.в., upcrossing inequality, декомпозиция Дуба.
\end{itemize}

\section{Васильев Алексей Георгиевич}

\begin{itemize}
  \item Условная вероятность: определение и обоснование определения. Примеры. Правило умножения вероятностей, формула полной вероятности, формула Байеса.
  \item Центральная предельная теорема: классическая и версии Линдеберга и Ляпунова. Скорость сходимости к нормальному распределению (теорема Берри-Эссена).
\end{itemize}

\section{Васильев Леонид Константинович}

\begin{itemize}
  \item Непрерывная с.в., условная на другой непрерывной с.в. Условная плотность вероятности, правило умножения плотностей, полная вероятности и полное матожидание.
  \item Границы Чернова. 
\end{itemize}

\section{Воркожоков Денис Вадимович}

\begin{itemize}
  \item Непрерывные с.в. Определение, свойства и интерпретация плотности вероятности. Матожидание и дисперсия непрерывных с.в., их элементарные свойства.
  \item Частота посещений состояния и перехода в марковской цепи. Демографический процесс. Скорость сходимости к стационарному распределению.
\end{itemize}

\section{Дювенжи Александр Николаевич}

\begin{itemize}
  \item С.в., условная на другой с.в. Условные случайные векторы, правило умножения, формула полной вероятности. Условное матожидание, формула полного матожидания.
  \item Задача Эрланга о пропускной способности телефонной сети.
\end{itemize}

\section{Ешкин Даниил Сергеевич}

\begin{itemize}
  \item Непрерывная с.в., условная на другой непрерывной с.в. Условная плотность вероятности, правило умножения плотностей, полная вероятности и полное матожидание.
  \item Слияние и разделение процесса Бернулли и его аппроксимация при маленьком элементарном временном интервале.
\end{itemize}

\section{Ильин Ярослав Дмитриевич}

\begin{itemize}
  \item Нормальное распределение. Определение, матожидание, дисперсия. Определение значения функции распределения нестандартного нормального распределения с помощью таблиц для стандартного нормального распределения.
  \item Слияние и разделение процесса Пуассона. Парадокс наблюдателя.
\end{itemize}

\section{Казаков Михаил Вячеславович}

\begin{itemize}
  \item Нормальное распределение. Определение, матожидание, дисперсия. Определение значения функции распределения нестандартного нормального распределения с помощью таблиц для стандартного нормального распределения.
  \item Неравенство Маркова. Неравенства для целочисленных с.в. Неравенство Чебышева.
\end{itemize}

\section{Клиначев Александр Викторович}

\begin{itemize}
  \item С.в., условная на другой с.в. Условные случайные векторы, правило умножения, формула полной вероятности. Условное матожидание, формула полного матожидания.
  \item Процесс Пуассона. Определение и основные свойства. Распределение Пуассон, сумма двух с.в. Пуассона.
\end{itemize}

\section{Козлов Михаил Александрович}

\begin{itemize}
  \item Случайные вектора. Совместная и маргинальная функции вероятности, примеры. Линейность матожидания (нескольких с.в.). Матожидание распределения Бернулли.
  \item Центральная предельная теорема: классическая и версии Линдеберга и Ляпунова. Скорость сходимости к нормальному распределению (теорема Берри-Эссена).
\end{itemize}

\section{Косогоров Евгений Михайлович}

\begin{itemize}
  \item Дисперсия. Определение, свойства. Дисперсия с.в. Бернулли и дискретной равномерной с.в.
  \item Сходимость по распределению, по вероятности и почти наверное.
\end{itemize}

\section{Криушенков Илья Сергеевич}

\begin{itemize}
  \item Смешанные распределения: примеры. Векторы непрерывных с.в.: совместная плотность вероятности, совместная функция распределения, функции от непрерывных с.в., линейность матожидания
  \item Сходимость по распределению, по вероятности и почти наверное.
\end{itemize}

\section{Кузин Максим Сергеевич}

\begin{itemize}
  \item Смешанные распределения: примеры. Векторы непрерывных с.в.: совместная плотность вероятности, совместная функция распределения, функции от непрерывных с.в., линейность матожидания
  \item Неравенство Хёффдинга. Неравенства Беннета и Бернштейна (без доказательств).
\end{itemize}

\section{Купчик Антон Михайлович}

\begin{itemize}
  \item Независимость: интуитивное и формальное определения. Независимость дополнений. Независимость множества событий: попарная и по совокупности. Неэквивалентность этих двух видов независимости.
  \item Поглощающие состояния. Время до поглощения. Время до прохода через возвратное состояние.
\end{itemize}

\section{Мартынов Павел Михайлович}

\begin{itemize}
  \item Случайные вектора. Совместная и маргинальная функции вероятности, примеры. Линейность матожидания (нескольких с.в.). Матожидание распределения Бернулли.
  \item Процесс Бернулли. Основные свойства.
\end{itemize}

\section{Мозжевилов Данил Дмитриевич}

\begin{itemize}
  \item Непрерывное равномерное и экспоненциальное распределения: определение, матожидание, дисперсия. Функция распределения и ее свойства.
  \item Границы Чернова. 
\end{itemize}

\section{Морев Савва Игоревич}

\begin{itemize}
  \item Определение случайной величины (с.в.). Дискретные с.в. и функция вероятностей. Примеры дискретных распределений и их функции вероятностей: равномерное распределение, распределений Бернулли, биномиальное распределение, геометрическое распределение.
  \item Возвратные и невозвратные состояния. Периодические состояния. Стационарное распределение, условие единственности. 
\end{itemize}

\section{Наумов Иван Леонидович}

\begin{itemize}
  \item Функции нескольких с.в. Случай независимых с.в. Имитация одним распределением другого распределения.
  \item Возвратные и невозвратные состояния. Периодические состояния. Стационарное распределение, условие единственности. 
\end{itemize}

\section{Панов Иван Андреевич}

\begin{itemize}
  \item Линейные и нелинейные преобразования с.в.
  \item Задача Эрланга о пропускной способности телефонной сети.
\end{itemize}

\section{Степанов Семен Алексеевич}

\begin{itemize}
  \item Независимость: интуитивное и формальное определения. Независимость дополнений. Независимость множества событий: попарная и по совокупности. Неэквивалентность этих двух видов независимости.
  \item Поглощающие состояния. Время до поглощения. Время до прохода через возвратное состояние.
\end{itemize}

\section{Тушканова Анастасия Дмитриевна}

\begin{itemize}
  \item Непрерывные с.в. Определение, свойства и интерпретация плотности вероятности. Матожидание и дисперсия непрерывных с.в., их элементарные свойства.
  \item Слияние и разделение процесса Бернулли и его аппроксимация при маленьком элементарном временном интервале.
\end{itemize}

\section{Холявин Николай Андреевич}

\begin{itemize}
  \item Непрерывное равномерное и экспоненциальное распределения: определение, матожидание, дисперсия. Функция распределения и ее свойства.
  \item Неравенство Хёффдинга. Неравенства Беннета и Бернштейна (без доказательств).
\end{itemize}



\end{document}