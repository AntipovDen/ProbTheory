\documentclass[12pt]{article}

\usepackage{a4wide}

\usepackage[utf8]{inputenc} 
\usepackage[russian]{babel}
\usepackage{amssymb}
\usepackage{amsmath}
\usepackage{pgfplots}
\usepgfplotslibrary{statistics,patchplots}
\usetikzlibrary{decorations.pathreplacing,calc,tikzmark, patterns,arrows.meta}
\pgfmathdeclarefunction{gauss}{3}{%
  \pgfmathparse{1/(#3*sqrt(2*pi))*exp(-((#1-#2)^2)/(2*#3^2))}%
}
\pgfmathdeclarefunction{gauss2d}{6}{%
  \pgfmathparse{1/(#3*#6*2*pi)*exp(-((#1-#2)^2)/(2*#3^2) - ((#4-#5)^2)/(2*#6^2))}%
}

\usepackage{xspace}

\usepackage{mathtools}
\usepackage{cite}
\usepackage{array}
\usepackage{multirow}
\usepackage{tabularx}
\usepackage{bbm}

\newcommand\N{\mathbb{N}}
\newcommand\R{\mathbb{R}}
\newcommand\eps{\varepsilon}
\newcommand\one{\mathbbm{1}}
\DeclareMathOperator{\Bin}{Bin}
\DeclareMathOperator{\Geom}{Geom}
\DeclareMathOperator{\pow}{pow}
\DeclareMathOperator{\Bern}{Bern}
\DeclareMathOperator{\Exp}{Exp}
\DeclareMathOperator{\Var}{Var}
\DeclareMathOperator{\Cov}{Cov}
\DeclareMathOperator{\sign}{sign}

\newtheorem{theorem}{Теорема}

\title{Лекция 9. Полезные инструменты}

\begin{document}
\maketitle

На этой лекции мы пройдем разные неравенства, которые позволяют нам делать некоторые выводы о с.в., когда информация об этих с.в. ограничена

\section{Неравенство Маркова}

Самый простой случай --- когда мы знаем только матожидание с.в. В этому случае нам может помочь неравенство Маркова. Если с.в. $X$ неотрицательно, тогда для всех $a \in \R^+$ верно, что 

\begin{center}
  \begin{tikzpicture}[rounded corners]
    \node [draw, rectangle, fill=blue!20, minimum height = 1.5cm, minimum width = 4.5cm] at (0,0) {
    \begin{minipage}{0.4\textwidth}
      \[
        \Pr(X \ge a) \le \frac{E[X]}{a}
      \]
    \end{minipage}};
  \end{tikzpicture}
\end{center}

Трактовка: $X$ вряд ли сильно больше своего матожидания. Заметим, что это неравенство несет хоть какую-то смысловую нагрузку только для $a \ge E[X]$.

Докажем это неравенство. Для этого рассмотрим другую с.в. $Y$, такую, что

\begin{align*}
  Y = \begin{cases}
    0, \text{ если } X < a,
    a, \text{ иначе.}
  \end{cases}
\end{align*}

Заметим, что так как $Y \le X$, то и $E[Y] \le E[X]$. Надо ли пояснить?

Поэтому
\begin{align*}
  E[X] \ge E[Y] = a\Pr(Y = a) + 0 = a\Pr(X \ge a)
\end{align*}
Разделим обе части на $a$, получим неравнество Маркова.

Посмотрим теперь, насколько оно точное. Возьмем с.в. $X \sim \Exp(1)$ и вспомним, что $E[X] = 1$ и $\Pr(X \ge a) = e^{-a}$.
Неравенство Маркова дает нам куда более слабую оценку: $\Pr(X \ge a) \le \frac{1}{a}$.

Посмотрим также, как применять это неравенство на другом примере. Пусть $X \sim U(-4, 4)$.

\end{document}