\documentclass[12pt]{article}

\usepackage{a4wide}

\usepackage[utf8]{inputenc} 
\usepackage[russian]{babel}
\usepackage{amssymb}
\usepackage{amsmath}
\usepackage{pgfplots}
\usepgfplotslibrary{statistics,patchplots}
\usetikzlibrary{decorations.pathreplacing,calc,tikzmark, patterns,arrows.meta}
\pgfmathdeclarefunction{gauss}{3}{%
  \pgfmathparse{1/(#3*sqrt(2*pi))*exp(-((#1-#2)^2)/(2*#3^2))}%
}
\pgfmathdeclarefunction{gauss2d}{6}{%
  \pgfmathparse{1/(#3*#6*2*pi)*exp(-((#1-#2)^2)/(2*#3^2) - ((#4-#5)^2)/(2*#6^2))}%
}

\usepackage{xspace}

\usepackage{mathtools}
\usepackage{cite}
\usepackage{array}
\usepackage{multirow}
\usepackage{tabularx}
\usepackage{bbm}

\newcommand\N{\mathbb{N}}
\newcommand\R{\mathbb{R}}
\newcommand\eps{\varepsilon}
\newcommand\one{\mathbbm{1}}
\DeclareMathOperator{\Bin}{Bin}
\DeclareMathOperator{\Geom}{Geom}
\DeclareMathOperator{\pow}{pow}
\DeclareMathOperator{\Bern}{Bern}
\DeclareMathOperator{\Exp}{Exp}
\DeclareMathOperator{\Var}{Var}
\DeclareMathOperator{\Cov}{Cov}
\DeclareMathOperator{\sign}{sign}

\newtheorem{theorem}{Теорема}

\title{Лекция 9. Полезные инструменты}

\begin{document}
\maketitle


\section{Равенство Вальда}

На прошлой лекции уже была упрощенная версия этой леммы, однако давайте рассмотрим обобщенный случай.

\begin{theorem}[Равенство Вальда]
  Пусть $\{X_n\}_{n \in \N}$ --- бесконечная последовательность вещественных с.в. Пусть также $N$ --- какая-то целочисленная неотрицательная с.в. (то есть $N$ всегда принимает значения из $\N_0$). Допустим также, что выполнены все условия:
  \begin{enumerate}
    \item Все $X_n$ имеют конечное матожидание.
    \item Для любого $n \in \N$ верно, что $E[X_n \one_{\{N \ge n\}}] = E[X_n] \Pr(N \ge n)$ (можно трактовать как событие $N \ge n$ не очень зависит от с.в. $X_n$).
    \item Следующий ряд сходится (трактовка: просто техническое требование, нужное в доказательстве, но также часто встречающееся на практике, особенно когда $X_n$ неотрицательные).
    \begin{align*}
      \sum_{n = 1}^{+\infty} E[|X_n|\one_{\{N \ge n\}}] < +\infty
    \end{align*}
  \end{enumerate}
  Тогда, если мы обозначим 
  \begin{align*}
    S_N &= \sum_{n = 1}^N X_n, \\
    T_N &= \sum_{n = 1}^N E[X_n],
  \end{align*}
  то верно, что $E[S_N] = E[T_N]$
  
  Если также $N$ имеет конечное матожидание и все $X_n$ имеют одинаковое конечное матожидание, то
  \begin{align*}
    E[S_N] = E[X_1] E[N].
  \end{align*}
\end{theorem}

Именно последнее равенство часто и имеют в виду, когда говорят о равенстве Вальда. Заметим, что по сравнению с тем, что у нас было раньше, мы не требуем полной независимости $X_n$ от $N$, а также допускаем, что у $X_n$ могут быть разные распределения.

Заметим, что в общем случае все три условия необходимы. Если не выполняется первое, то суммы $S_N$ и $T_N$ просто неопределены. 

Необходимость второго условия демонстрируется следующим примером. Возьмем последовательность $X_n$, которые следуют распределению Бернулли с $p = 0.5$. И возьмем довольно зависимый от них $N = 1 - X_1$. В таком случае $S_N$ всегда равно нуль (либо это сумма из нуля слагаемых, либо это одно слагаемое, равное нулю), однако $E[X_1]E[N] = \frac{1}{2} \cdot \frac{1}{2} = \frac{1}{4}$. В данном примере есть явная зависимость события $N \ge 1$ от $X_1$, причем 

\begin{align*}
  E[X_1 \one_{\{N \ge 1\}}] = \frac{1}{2} \cdot 1 \cdot 0 + \frac{1}{2} \cdot 0 \cdot 1 = 0 \ne \frac{1}{2} \cdot \frac{1}{2} = E[X_1] \Pr[N \ge 1].
\end{align*}

Для необходимости третьего (более технического) условия рассмотрим последовательность $X_n = \pm 2^n$, причем знак выбирается равновероятно. И возьмем $N = \min_n \{X_n = +2^n\}$, то есть первый раз, когда мы взяли положительную степень двойки. В такос случае выполняется первое условие, $E[X_n] = 0$, выполняется и второе, так как событие $N \ge n$ зависит только от $X_1, \dots, X_{n - 1}$, но не от $X_n$. При этом третье условие не выполнено:
\begin{align*}
  E[|X_n|\one_{\{N \ge n\}}] = 2^n \cdot \left(\frac{1}{2}\right)^{n - 1} = \frac{1}{2}.
\end{align*}

В таком случае $S_N = 2$, так как независимо от $N$ эта сумма равна $2^N - \sum_{n = 1}^{N - 1} 2^n = 2$. Но так как $E[X_n] = 0$, то справа в равенстве ноль.

\textbf{Доказательство равенства Вальда}



\end{document}