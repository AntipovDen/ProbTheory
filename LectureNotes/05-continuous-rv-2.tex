\documentclass[12pt]{article}

\usepackage{a4wide}

\usepackage[utf8]{inputenc} 
\usepackage[russian]{babel}
\usepackage{amssymb}
\usepackage{amsmath}
\usepackage{pgfplots}
\usepgfplotslibrary{statistics}
\usetikzlibrary{decorations.pathreplacing,calc,tikzmark, patterns,arrows.meta}
\pgfmathdeclarefunction{gauss}{3}{%
  \pgfmathparse{1/(#3*sqrt(2*pi))*exp(-((#1-#2)^2)/(2*#3^2))}%
}

\usepackage{xspace}

\usepackage{mathtools}
\usepackage{cite}
\usepackage{array}
\usepackage{multirow}
\usepackage{tabularx}

\newcommand\N{\mathbb{N}}
\newcommand\R{\mathbb{R}}
\newcommand\eps{\varepsilon}
\DeclareMathOperator{\Bin}{Bin}
\DeclareMathOperator{\Geom}{Geom}
\DeclareMathOperator{\pow}{pow}
\DeclareMathOperator{\Bern}{Bern}
\DeclareMathOperator{\Exp}{Exp}
\DeclareMathOperator{\Var}{Var}


\title{Лекция 5. Непрерывные с.в., часть 2}

\begin{document}
\maketitle


\section{Условная плотность вероятности}

Напомним трактовку плотность вероятности. Это то, сколько вероятностной массы приходится на маленький интервал значений с.в.:

\[f_X(x) \delta \approx \Pr(X \in [x, x + \delta])\]

Но вероятностная мера может меняться при условии, что произошло событие $A$. В этом случае определяем

\[f_{X\mid A} (x) \delta \approx \Pr(X \in [x, x + \delta] \mid A)\]

Точное определение: плотность вероятности с.в. $X$ при условии $A$ есть такая функция $f_{X \mid A}(x)$, что для любого измеримого множества $B$ верно, что
\begin{align*}
  \Pr(X \in B \mid A) = \int_B f_{X \mid A}(x) dx.
\end{align*}

\emph{NB:} у нас опять просто поменялась вероятностная мера. То есть у условной плотности вероятности будут все те же свойства:
\begin{itemize}
  \item $f_{X \mid A}(x) \ge 0$
  \item $\int_{-\infty}^{+\infty} f_{X \mid A} (x) dx = 1$
 \end{itemize}

 Вычисляем аналогично условной функции вероятности. Пусть $x \in A$, тогда

 \begin{align*}
  f_{X\mid A} (x) \delta &\approx \Pr(X \in [x, x + \delta] \mid A) = \frac{\Pr(X \in [x, x + \delta] \cap X \in A)}{\Pr(A)} \\
  &= \frac{\Pr(X \in [x, x + \delta])}{\Pr(A)} \approx \frac{f_X(x) \delta}{\Pr(A)}
 \end{align*}

 Поэтому строго говоря, она вычисляется так:
 \begin{align*}
   f_{X\mid A}(x) = \begin{cases}
     \frac{f_X(x)}{\Pr(A)}, &\text{ если } x \in A, \\
     0, &\text{ иначе.}
   \end{cases}
 \end{align*}
 То есть масштабируем плотность вероятности по событию $A$.

 Иногда придется иметь дело с условной функцией распределения:

 \begin{align*}
   F_{X \mid A}(x) = \Pr(X \le x \mid A)
 \end{align*}

 \section{Условное матожидание}

 Условное матожидание определяем аналогично:

 \begin{align*}
   E(X \mid A) = \int_{-\infty}^{+\infty} x f_{X \mid A}(x) dx
 \end{align*}

 Работает то же самое правило и для функций от с.в. (у нас просто новая плотность вероятности при условии $A$):
 \begin{align*}
   E(g(X) \mid A) = \int_{-\infty}^{+\infty} g(x) f_{X \mid A}(x) dx
 \end{align*}

 \section{Пример условных с.в.}

 Рассмотрим частично равномерное распределение:
 \begin{align*}
   f_X(x)  =\begin{cases}
     0.25, x \in [0, 2) \\
     0.15, x \in [2, 3) \\
     0.35, x \in [3, 4] \\
     0, \text{ иначе.}
   \end{cases}
 \end{align*}


\begin{center}
   \begin{tikzpicture}
    \begin{axis}[
    grid=major,
    ylabel={$f_{X}(x)$},
    xlabel={$x$}]
    
    \addplot[draw=blue, draw=none, fill=blue!20] coordinates
      {(0,0.25) (2,0.25)} \closedcycle;
    \addplot[draw=blue, draw=none, fill=blue!20] coordinates
      {(2,0.15) (3,0.15)} \closedcycle;
    \addplot[draw=blue, draw=none, fill=blue!20] coordinates
      {(3,0.35) (4,0.35)} \closedcycle;
    
    \addplot[draw=blue, ultra thick] coordinates
      {(-1,0) (0,0)};
    \addplot[draw=blue, ultra thick] coordinates
      {(0,0.25) (2,0.25)};
    \addplot[draw=blue, ultra thick] coordinates
      {(2,0.15) (3,0.15)};
    \addplot[draw=blue, ultra thick] coordinates
      {(3,0.35) (4,0.35)};
    \addplot[draw=blue, ultra thick] coordinates
      {(4,0) (5,0)};
    \end{axis}
  \end{tikzpicture}
\end{center}

Пусть $A = [1, 3]$. Тогда $\Pr(A) = 0.25 \cdot 1 + 0.15 \cdot 1 = 0.4$ 

\begin{center}
  \begin{tikzpicture}
   \begin{axis}[
   grid=major,
   ylabel={$f_{X}(x)$},
   xlabel={$x$}]
   
   \addplot[draw=blue, draw=none, fill=blue!20] coordinates
     {(0,0.25) (1,0.25)} \closedcycle;
    \addplot[draw=blue, draw=none, fill=red!20] coordinates
     {(1,0.25) (2,0.25)} \closedcycle;
   \addplot[draw=blue, draw=none, fill=red!20] coordinates
     {(2,0.15) (3,0.15)} \closedcycle;
   \addplot[draw=blue, draw=none, fill=blue!20] coordinates
     {(3,0.35) (4,0.35)} \closedcycle;
   
   \addplot[draw=blue, ultra thick] coordinates
     {(-1,0) (0,0)};
    \addplot[draw=blue, ultra thick] coordinates
      {(0,0.25) (1,0.25)};
    \addplot[draw=red, ultra thick] coordinates
      {(1,0.25) (2,0.25)};
   \addplot[draw=red, ultra thick] coordinates
     {(2,0.15) (3,0.15)};
   \addplot[draw=blue, ultra thick] coordinates
     {(3,0.35) (4,0.35)};
   \addplot[draw=blue, ultra thick] coordinates
     {(4,0) (5,0)};
   \node [red] at (axis cs:2,0.1) {$A$};
  \end{axis}
 \end{tikzpicture}
\end{center}

Значит, новая плотность вероятности выглядит так:
\begin{align*}
  f_{X \mid A}(x)  =\begin{cases}
    \frac{0.25}{0.4} = 0.625, x \in [1, 2) \\
    \frac{0.15}{0.4} = 0.375, x \in [2, 3) \\
    0, \text{ иначе.}
  \end{cases}
\end{align*}

\begin{center}
  \begin{tikzpicture}
   \begin{axis}[
   grid=major,
   ylabel={$f_{X \mid A}(x)$},
   xlabel={$x$}]
   
    \addplot[draw=red, draw=none, fill=red!20] coordinates
      {(0,0.25) (2,0.25)} \closedcycle;
    \addplot[draw=red, draw=none, fill=red!20] coordinates
      {(2,0.15) (3,0.15)} \closedcycle;
    \addplot[draw=red, draw=none, fill=red!20] coordinates
      {(3,0.35) (4,0.35)} \closedcycle;
    
    \addplot[draw=red, ultra thick] coordinates
      {(-1,0) (0,0)};
    \addplot[draw=red, ultra thick] coordinates
      {(0,0.25) (2,0.25)};
    \addplot[draw=red, ultra thick] coordinates
      {(2,0.15) (3,0.15)};
    \addplot[draw=red, ultra thick] coordinates
      {(3,0.35) (4,0.35)};
    \addplot[draw=red, ultra thick] coordinates
      {(4,0) (5,0)};

    \addplot[draw=blue, draw=none, fill=blue, fill opacity=0.2] coordinates
      {(1,0.625) (2,0.625)} \closedcycle;
    \addplot[draw=blue, draw=none, fill=blue, fill opacity=0.2] coordinates
      {(2,0.375) (3,0.375)} \closedcycle;
   
    \addplot[draw=blue, ultra thick] coordinates
      {(-1,0) (1,0)};
    \addplot[draw=blue, ultra thick] coordinates
      {(1,0.625) (2,0.625)};
    \addplot[draw=blue, ultra thick] coordinates
      {(2,0.375) (3,0.375)};
    \addplot[draw=blue, ultra thick] coordinates
      {(3,0) (5,0)};

    \node (exp) [above] at (axis cs:1.875,0) {};
   \end{axis}
   
   \node (label) at (2, -2) {Вот тут цетр масс};
   \draw [->] (label) -- (exp);
 \end{tikzpicture}
\end{center}  

\begin{align*}
  E(X \mid A) = \int_1^2 t \cdot 0.625 dt + \int_2^3 t \cdot 0.375 dt = 1.875
\end{align*}

\section{Беспамятство экспоненциального распределения}

Как уже говорилось, экспоненциальное распределение очень похоже на геометрическое. В том числе вот почему. Пусть продолжительность жизни лампочки $T$ следует $\Exp(\lambda)$. Следует ли поменять лампочку после того, как она проработала время $t$? Посмотрим, сколько она еще проживет, то есть распределение $T - t$ при условии $T \ge t$. 

\begin{align*}
  F_{(T - t) \mid T \ge t}(x) &= \Pr(T - t \ge x \mid T > t) = \frac{\Pr(T - t \ge x \cap T > t)}{\Pr(T > t)} \\
  &= \frac{\Pr(T\ge x + t)}{\Pr(T > t)} = \frac{e^{-\lambda(x + t)}}{e^{-\lambda t}} = e^{-\lambda x},  
\end{align*}
то есть распределение точно то же, как если мы заменим лампочку на новую.

\section{Полные вероятность и матожидание}

Напомним: пусть есть разбиение $\Omega$ на $\{A_i\}$ (не более, чем счетное), тогда

\begin{align*}
  \Pr(B) &= \Pr(A_1) \Pr(B \mid A_1) + \dots +  \Pr(A_n) \Pr(B \mid A_n) + \dots \\
  p_X(x) &= \Pr(A_1) p_{X \mid A_1}(x) + \dots + \Pr(A_n) p_{X \mid A_n}(x) + \dots
\end{align*}

Ничего не меняется и в непрерывном случае. Сначала функция распределения:
\begin{align*}
  F_X(x) = \Pr(X \le x) &= \Pr(A_1) \Pr(X \le x \mid A_1) + \dots + \Pr(A_n) \Pr(X \le x \mid A_n) + \dots \\
  &= \Pr(A_1) F_{X \mid A_1}(x) + \dots + \Pr(A_n) F_{X \mid A_n}(x) + \dots
\end{align*}

Дифференцируем, получаем:
\begin{align*}
  f_X(x) = \Pr(A_1) f_{X \mid A_1}(x) + \dots + \Pr(A_n) f_{X \mid A_n}(x) + \dots
\end{align*}

Умножаем на $x$ и интегрируем по всему $\R$:
\begin{align*}
  E(X) = \Pr(A_1) E(X \mid A_1) + \dots + \Pr(A_n) E(X \mid A_n) + \dots
\end{align*}

Пример (который уже был): 

\begin{align*}
  f_X(x)  =\begin{cases}
    0.25, x \in [0, 2) \\
    0.15, x \in [2, 3) \\
    0.35, x \in [3, 4] \\
    0, \text{ иначе.}
  \end{cases}
\end{align*}


\begin{center}
  \begin{tikzpicture}
   \begin{axis}[
   grid=major,
   ylabel={$f_{X}(x)$},
   xlabel={$x$}]
   
   \addplot[draw=blue, draw=none, fill=blue!20] coordinates
     {(0,0.25) (2,0.25)} \closedcycle;
   \addplot[draw=blue, draw=none, fill=blue!20] coordinates
     {(2,0.15) (3,0.15)} \closedcycle;
   \addplot[draw=blue, draw=none, fill=blue!20] coordinates
     {(3,0.35) (4,0.35)} \closedcycle;
   
   \addplot[draw=blue, ultra thick] coordinates
     {(-1,0) (0,0)};
   \addplot[draw=blue, ultra thick] coordinates
     {(0,0.25) (2,0.25)};
   \addplot[draw=blue, ultra thick] coordinates
     {(2,0.15) (3,0.15)};
   \addplot[draw=blue, ultra thick] coordinates
     {(3,0.35) (4,0.35)};
   \addplot[draw=blue, ultra thick] coordinates
     {(4,0) (5,0)};
   \end{axis}
 \end{tikzpicture}
\end{center}

Посчитаем матожидание

\begin{align*}
  E(X) &= \Pr(X \in [0, 2)) E(X \mid X \in [0, 2)) \\
  &+ \Pr(X \in [2, 3)) E(X \mid X \in [2, 3)) \\
  &+ \Pr(X \in [3, 4)) E(X \mid X \in [3, 4)) 
\end{align*}

Заметим, что на каждом отрезке матожидание -- это положение центра масс, то есть середина отрезка. Поэтому


\begin{align*}
  E(X) = 0.5 \cdot 1 + 0.15 \cdot 2.5 + 0.35 \cdot 3.5 = 2.1
\end{align*}

\section{Смешанные распределения}

Иногда с.в. могут быть ни дискретными, ни непрерывными, например. Пусть у нас есть слеующий эксперимент. Сначала бросам честную монетку, потом, если выпал орел, то выбираем случайное число из отрезка $[0,2]$ (равномерно). Случайная величина $X$ при этом равна $1$ в случае решки и равна выбранному числу в случае орла.

У данной с.в. нет функции вероятностей, как нет и плотности вероятности. Фукнция распределения все-такие есть. Как ее посчитать? По формуле полной вероятности, которая работает и для функции распределения. Пусть $A$ --- событие "выпала решка"

\begin{align*}
  F_X(x) = F_{X \mid A} (x) \Pr(A) + F_{X \mid \bar A}(x) \Pr(\bar A)
\end{align*}

Заметим, что если $A$, то $X = 1$ с вероятностью $1$. То есть, 
\begin{align*}
  F_{X \mid A} = \begin{cases}
    0, x < 1 \\
    1, x \ge 1
  \end{cases}
\end{align*}
\begin{center}
  \begin{tikzpicture}
    \begin{axis}[xmin=-1, xmax=3,
    % width = 0.45\textwidth,
    grid=major,
    ylabel={$F_{X \mid A}(x)$},
    xlabel={$x$}]
    \addplot[draw=blue, ultra thick] coordinates
      {(-1,0) (1, 0)};

    \addplot[draw=blue, ultra thick] coordinates
      {(1,1) (3, 1)};
    \end{axis}
  \end{tikzpicture}
\end{center}

А если $\bar A$, то $X$ следует равномерному распределению на отрезке $[0, 2]$.
\begin{align*}
  F_{X \mid \bar A} = \begin{cases}
    0, x < 0 \\
    x/2, x \in [0, 2] \\
    1, x > 2
  \end{cases}
\end{align*}
\begin{center}
  \begin{tikzpicture}
    \begin{axis}[xmin=-1, xmax=3,
    % width = 0.45\textwidth,
    grid=major,
    ylabel={$F_{X \mid \bar A}(x)$},
    xlabel={$x$}]
    \addplot[draw=blue, ultra thick] coordinates
      {(-1,0) (0,0) (2, 1) (3, 1)};
    \end{axis}
  \end{tikzpicture}
  
\end{center}


\end{document}