\documentclass[12pt]{article}

\usepackage{a4wide}

\usepackage[utf8]{inputenc} 
\usepackage[russian]{babel}
\usepackage{amssymb}
\usepackage{amsmath}
\usepackage{pgfplots}
\usepgfplotslibrary{statistics,patchplots}
\usetikzlibrary{decorations.pathreplacing,calc,tikzmark, patterns,arrows.meta}
\pgfmathdeclarefunction{gauss}{3}{%
  \pgfmathparse{1/(#3*sqrt(2*pi))*exp(-((#1-#2)^2)/(2*#3^2))}%
}

\usepackage{xspace}

\usepackage{mathtools}
\usepackage{cite}
\usepackage{array}
\usepackage{multirow}
\usepackage{tabularx}

\newcommand\N{\mathbb{N}}
\newcommand\R{\mathbb{R}}
\newcommand\eps{\varepsilon}
\DeclareMathOperator{\Bin}{Bin}
\DeclareMathOperator{\Geom}{Geom}
\DeclareMathOperator{\pow}{pow}
\DeclareMathOperator{\Bern}{Bern}
\DeclareMathOperator{\Exp}{Exp}
\DeclareMathOperator{\Var}{Var}


\title{Лекция 5. Непрерывные с.в., часть 2}

\begin{document}
\maketitle


\section{Условная плотность вероятности}

Напомним трактовку плотность вероятности. Это то, сколько вероятностной массы приходится на маленький интервал значений с.в.:

\[f_X(x) \delta \approx \Pr(X \in [x, x + \delta])\]

Но вероятностная мера может меняться при условии, что произошло событие $A$. В этом случае определяем

\[f_{X\mid A} (x) \delta \approx \Pr(X \in [x, x + \delta] \mid A)\]

Точное определение: плотность вероятности с.в. $X$ при условии $A$ есть такая функция $f_{X \mid A}(x)$, что для любого измеримого множества $B$ верно, что
\begin{align*}
  \Pr(X \in B \mid A) = \int_B f_{X \mid A}(x) dx.
\end{align*}

\emph{NB:} у нас опять просто поменялась вероятностная мера. То есть у условной плотности вероятности будут все те же свойства:
\begin{itemize}
  \item $f_{X \mid A}(x) \ge 0$
  \item $\int_{-\infty}^{+\infty} f_{X \mid A} (x) dx = 1$
 \end{itemize}

 Вычисляем аналогично условной функции вероятности. Пусть $x \in A$, тогда

 \begin{align*}
  f_{X\mid A} (x) \delta &\approx \Pr(X \in [x, x + \delta] \mid A) = \frac{\Pr(X \in [x, x + \delta] \cap X \in A)}{\Pr(A)} \\
  &= \frac{\Pr(X \in [x, x + \delta])}{\Pr(A)} \approx \frac{f_X(x) \delta}{\Pr(A)}
 \end{align*}

 Поэтому строго говоря, она вычисляется так:
 \begin{align*}
   f_{X\mid A}(x) = \begin{cases}
     \frac{f_X(x)}{\Pr(A)}, &\text{ если } x \in A, \\
     0, &\text{ иначе.}
   \end{cases}
 \end{align*}
 То есть масштабируем плотность вероятности по событию $A$.

 Иногда придется иметь дело с условной функцией распределения:

 \begin{align*}
   F_{X \mid A}(x) = \Pr(X \le x \mid A)
 \end{align*}

 \section{Условное матожидание}

 Условное матожидание определяем аналогично:

 \begin{align*}
   E(X \mid A) = \int_{-\infty}^{+\infty} x f_{X \mid A}(x) dx
 \end{align*}

 Работает то же самое правило и для функций от с.в. (у нас просто новая плотность вероятности при условии $A$):
 \begin{align*}
   E(g(X) \mid A) = \int_{-\infty}^{+\infty} g(x) f_{X \mid A}(x) dx
 \end{align*}

 \section{Пример условных с.в.}

 Рассмотрим частично равномерное распределение:
 \begin{align*}
   f_X(x)  =\begin{cases}
     0.25, x \in [0, 2) \\
     0.15, x \in [2, 3) \\
     0.35, x \in [3, 4] \\
     0, \text{ иначе.}
   \end{cases}
 \end{align*}


\begin{center}
   \begin{tikzpicture}
    \begin{axis}[
    grid=major,
    ylabel={$f_{X}(x)$},
    xlabel={$x$}]
    
    \addplot[draw=blue, draw=none, fill=blue!20] coordinates
      {(0,0.25) (2,0.25)} \closedcycle;
    \addplot[draw=blue, draw=none, fill=blue!20] coordinates
      {(2,0.15) (3,0.15)} \closedcycle;
    \addplot[draw=blue, draw=none, fill=blue!20] coordinates
      {(3,0.35) (4,0.35)} \closedcycle;
    
    \addplot[draw=blue, ultra thick] coordinates
      {(-1,0) (0,0)};
    \addplot[draw=blue, ultra thick] coordinates
      {(0,0.25) (2,0.25)};
    \addplot[draw=blue, ultra thick] coordinates
      {(2,0.15) (3,0.15)};
    \addplot[draw=blue, ultra thick] coordinates
      {(3,0.35) (4,0.35)};
    \addplot[draw=blue, ultra thick] coordinates
      {(4,0) (5,0)};
    \end{axis}
  \end{tikzpicture}
\end{center}

Пусть $A = [1, 3]$. Тогда $\Pr(A) = 0.25 \cdot 1 + 0.15 \cdot 1 = 0.4$ 

\begin{center}
  \begin{tikzpicture}
   \begin{axis}[
   grid=major,
   ylabel={$f_{X}(x)$},
   xlabel={$x$}]
   
   \addplot[draw=blue, draw=none, fill=blue!20] coordinates
     {(0,0.25) (1,0.25)} \closedcycle;
    \addplot[draw=blue, draw=none, fill=red!20] coordinates
     {(1,0.25) (2,0.25)} \closedcycle;
   \addplot[draw=blue, draw=none, fill=red!20] coordinates
     {(2,0.15) (3,0.15)} \closedcycle;
   \addplot[draw=blue, draw=none, fill=blue!20] coordinates
     {(3,0.35) (4,0.35)} \closedcycle;
   
   \addplot[draw=blue, ultra thick] coordinates
     {(-1,0) (0,0)};
    \addplot[draw=blue, ultra thick] coordinates
      {(0,0.25) (1,0.25)};
    \addplot[draw=red, ultra thick] coordinates
      {(1,0.25) (2,0.25)};
   \addplot[draw=red, ultra thick] coordinates
     {(2,0.15) (3,0.15)};
   \addplot[draw=blue, ultra thick] coordinates
     {(3,0.35) (4,0.35)};
   \addplot[draw=blue, ultra thick] coordinates
     {(4,0) (5,0)};
   \node [red] at (axis cs:2,0.1) {$A$};
  \end{axis}
 \end{tikzpicture}
\end{center}

Значит, новая плотность вероятности выглядит так:
\begin{align*}
  f_{X \mid A}(x)  =\begin{cases}
    \frac{0.25}{0.4} = 0.625, x \in [1, 2) \\
    \frac{0.15}{0.4} = 0.375, x \in [2, 3) \\
    0, \text{ иначе.}
  \end{cases}
\end{align*}

\begin{center}
  \begin{tikzpicture}
   \begin{axis}[
   grid=major,
   ylabel={$f_{X \mid A}(x)$},
   xlabel={$x$}]
   
    \addplot[draw=red, draw=none, fill=red!20] coordinates
      {(0,0.25) (2,0.25)} \closedcycle;
    \addplot[draw=red, draw=none, fill=red!20] coordinates
      {(2,0.15) (3,0.15)} \closedcycle;
    \addplot[draw=red, draw=none, fill=red!20] coordinates
      {(3,0.35) (4,0.35)} \closedcycle;
    
    \addplot[draw=red, ultra thick] coordinates
      {(-1,0) (0,0)};
    \addplot[draw=red, ultra thick] coordinates
      {(0,0.25) (2,0.25)};
    \addplot[draw=red, ultra thick] coordinates
      {(2,0.15) (3,0.15)};
    \addplot[draw=red, ultra thick] coordinates
      {(3,0.35) (4,0.35)};
    \addplot[draw=red, ultra thick] coordinates
      {(4,0) (5,0)};

    \addplot[draw=blue, draw=none, fill=blue, fill opacity=0.2] coordinates
      {(1,0.625) (2,0.625)} \closedcycle;
    \addplot[draw=blue, draw=none, fill=blue, fill opacity=0.2] coordinates
      {(2,0.375) (3,0.375)} \closedcycle;
   
    \addplot[draw=blue, ultra thick] coordinates
      {(-1,0) (1,0)};
    \addplot[draw=blue, ultra thick] coordinates
      {(1,0.625) (2,0.625)};
    \addplot[draw=blue, ultra thick] coordinates
      {(2,0.375) (3,0.375)};
    \addplot[draw=blue, ultra thick] coordinates
      {(3,0) (5,0)};

    \node (exp) [above] at (axis cs:1.875,0) {};
   \end{axis}
   
   \node (label) at (2, -2) {Вот тут цетр масс};
   \draw [->] (label) -- (exp);
 \end{tikzpicture}
\end{center}  

\begin{align*}
  E(X \mid A) = \int_1^2 t \cdot 0.625 dt + \int_2^3 t \cdot 0.375 dt = 1.875
\end{align*}

\section{Беспамятство экспоненциального распределения}

Как уже говорилось, экспоненциальное распределение очень похоже на геометрическое. В том числе вот почему. Пусть продолжительность жизни лампочки $T$ следует $\Exp(\lambda)$. Следует ли поменять лампочку после того, как она проработала время $t$? Посмотрим, сколько она еще проживет, то есть распределение $T - t$ при условии $T \ge t$. 

\begin{align*}
  F_{(T - t) \mid T \ge t}(x) &= \Pr(T - t \ge x \mid T > t) = \frac{\Pr(T - t \ge x \cap T > t)}{\Pr(T > t)} \\
  &= \frac{\Pr(T\ge x + t)}{\Pr(T > t)} = \frac{e^{-\lambda(x + t)}}{e^{-\lambda t}} = e^{-\lambda x},  
\end{align*}
то есть распределение точно то же, как если мы заменим лампочку на новую.

\section{Полные вероятность и матожидание}

Напомним: пусть есть разбиение $\Omega$ на $\{A_i\}$ (не более, чем счетное), тогда

\begin{align*}
  \Pr(B) &= \Pr(A_1) \Pr(B \mid A_1) + \dots +  \Pr(A_n) \Pr(B \mid A_n) + \dots \\
  p_X(x) &= \Pr(A_1) p_{X \mid A_1}(x) + \dots + \Pr(A_n) p_{X \mid A_n}(x) + \dots
\end{align*}

Ничего не меняется и в непрерывном случае. Сначала функция распределения:
\begin{align*}
  F_X(x) = \Pr(X \le x) &= \Pr(A_1) \Pr(X \le x \mid A_1) + \dots + \Pr(A_n) \Pr(X \le x \mid A_n) + \dots \\
  &= \Pr(A_1) F_{X \mid A_1}(x) + \dots + \Pr(A_n) F_{X \mid A_n}(x) + \dots
\end{align*}

Дифференцируем, получаем:
\begin{align*}
  f_X(x) = \Pr(A_1) f_{X \mid A_1}(x) + \dots + \Pr(A_n) f_{X \mid A_n}(x) + \dots
\end{align*}

Умножаем на $x$ и интегрируем по всему $\R$:
\begin{align*}
  E(X) = \Pr(A_1) E(X \mid A_1) + \dots + \Pr(A_n) E(X \mid A_n) + \dots
\end{align*}

Пример (который уже был): 

\begin{align*}
  f_X(x)  =\begin{cases}
    0.25, x \in [0, 2) \\
    0.15, x \in [2, 3) \\
    0.35, x \in [3, 4] \\
    0, \text{ иначе.}
  \end{cases}
\end{align*}


\begin{center}
  \begin{tikzpicture}
   \begin{axis}[
   grid=major,
   ylabel={$f_{X}(x)$},
   xlabel={$x$}]
   
   \addplot[draw=blue, draw=none, fill=blue!20] coordinates
     {(0,0.25) (2,0.25)} \closedcycle;
   \addplot[draw=blue, draw=none, fill=blue!20] coordinates
     {(2,0.15) (3,0.15)} \closedcycle;
   \addplot[draw=blue, draw=none, fill=blue!20] coordinates
     {(3,0.35) (4,0.35)} \closedcycle;
   
   \addplot[draw=blue, ultra thick] coordinates
     {(-1,0) (0,0)};
   \addplot[draw=blue, ultra thick] coordinates
     {(0,0.25) (2,0.25)};
   \addplot[draw=blue, ultra thick] coordinates
     {(2,0.15) (3,0.15)};
   \addplot[draw=blue, ultra thick] coordinates
     {(3,0.35) (4,0.35)};
   \addplot[draw=blue, ultra thick] coordinates
     {(4,0) (5,0)};
   \end{axis}
 \end{tikzpicture}
\end{center}

Посчитаем матожидание

\begin{align*}
  E(X) &= \Pr(X \in [0, 2)) E(X \mid X \in [0, 2)) \\
  &+ \Pr(X \in [2, 3)) E(X \mid X \in [2, 3)) \\
  &+ \Pr(X \in [3, 4)) E(X \mid X \in [3, 4)) 
\end{align*}

Заметим, что на каждом отрезке матожидание -- это положение центра масс, то есть середина отрезка. Поэтому


\begin{align*}
  E(X) = 0.5 \cdot 1 + 0.15 \cdot 2.5 + 0.35 \cdot 3.5 = 2.1
\end{align*}

\section{Смешанные распределения}

Иногда с.в. могут быть ни дискретными, ни непрерывными, например. Пусть у нас есть слеующий эксперимент. Сначала бросам честную монетку, потом, если выпал орел, то выбираем случайное число из отрезка $[0,2]$ (равномерно). Случайная величина $X$ при этом равна $1$ в случае решки и равна выбранному числу в случае орла.

У данной с.в. нет функции вероятностей, как нет и плотности вероятности. Фукнция распределения все-такие есть. Как ее посчитать? По формуле полной вероятности, которая работает и для функции распределения. Пусть $A$ --- событие "выпала решка"

\begin{align*}
  F_X(x) = F_{X \mid A} (x) \Pr(A) + F_{X \mid \bar A}(x) \Pr(\bar A)
\end{align*}

Заметим, что если $A$, то $X = 1$ с вероятностью $1$. То есть, 
\begin{align*}
  F_{X \mid A}(x) = \begin{cases}
    0, x < 1 \\
    1, x \ge 1
  \end{cases}
\end{align*}
\begin{center}
  \begin{tikzpicture}
    \begin{axis}[xmin=-1, xmax=3,
    % width = 0.45\textwidth,
    grid=major,
    ylabel={$F_{X \mid A}(x)$},
    xlabel={$x$}]
    \addplot[draw=blue, ultra thick] coordinates
      {(-1,0) (1, 0)};

    \addplot[draw=blue, ultra thick] coordinates
      {(1,1) (3, 1)};
    \end{axis}
  \end{tikzpicture}
\end{center}

А если $\bar A$, то $X$ следует равномерному распределению на отрезке $[0, 2]$.
\begin{align*}
  F_{X \mid \bar A}(x) = \begin{cases}
    0, x < 0 \\
    x/2, x \in [0, 2] \\
    1, x > 2
  \end{cases}
\end{align*}
\begin{center}
  \begin{tikzpicture}
    \begin{axis}[xmin=-1, xmax=3,
    % width = 0.45\textwidth,
    grid=major,
    ylabel={$F_{X \mid \bar A}(x)$},
    xlabel={$x$}]
    \addplot[draw=blue, ultra thick] coordinates
      {(-1,0) (0,0) (2, 1) (3, 1)};
    \end{axis}
  \end{tikzpicture}
\end{center}

Итоговая функция распределения выглядит так:
\begin{align*}
  F_X(x) = \frac{1}{2}F_{X \mid A}(x) + \frac{1}{2} F_{X \mid \bar A}(x) = \begin{cases}
    0, x < 0 \\
    x/4, x \in [0, 1) \\
    x/4 + 1/2, x \in [1, 2] \\
    1, x > 2
  \end{cases}
\end{align*}
\begin{center}
  \begin{tikzpicture}
    \begin{axis}[xmin=-1, xmax=3,
    % width = 0.45\textwidth,
    grid=major,
    ylabel={$F_{X \mid \bar A}(x)$},
    xlabel={$x$}]
    \addplot[draw=blue, ultra thick] coordinates
      {(-1,0) (0,0) (1, 0.25)};
    \addplot[draw=blue, ultra thick] coordinates
      {(1,0.75) (2,1) (3, 1)};
    \end{axis}
  \end{tikzpicture}
\end{center}

\section{Векторы из непрерывных с.в.}

Пусть у нас есть 2 непрерывных с.в. $X$ и $Y$. Тогда можно говорить о совместной плотности вероятности:
\begin{align*}
  f_{X, Y}(x, y) \delta^2 \approx \Pr(X \in [x, x + \delta] \cap Y \in [y, y + \delta])
\end{align*}

Более строго: если у нас есть такая функция $f_{X ,Y}(x ,y)$, такая что для любого измеримого множества $A \subset \R^2$ верно
\begin{align*}
  \Pr((X, Y) \in A) = \iint\limits_A f_{X, Y}(x, y) dx dy,
\end{align*} 
то с.в. $X$ и $Y$ называются совместно непрерывными, а $f_{X, Y}(x, y)$ называется их совместной плотностью вероятности 

Свойства, аналогичные совместной функции вероятности, только суммы заменены интегралами:

\begin{itemize}
  \item $f_{X, Y}(x, y) \ge 0$
  \item $\int_{-\infty}^{+\infty}\int_{-\infty}^{+\infty}  f_{X, Y}(x, y) dx dy = 1$
\end{itemize}

Как визуализировать:

\begin{center}
  \begin{tikzpicture}
    \begin{axis}
      \addplot3[
        surf,
        samples=3,
        domain=-1:0,
      ]
      {0};

      
      \addplot3[ultra thick]
      coordinates {(-1, -1, 0) (-1, -1, 0.4)};
      \addplot3[ultra thick]
      coordinates {(-1, 0, 0) (-1, 0, 0.5)};
      \addplot3[ultra thick]
      coordinates {(0, -1, 0) (0, -1, 0.3)};
      \addplot3[ultra thick]
      coordinates {(0, 0, 0) (0, 0, 0.4)};

      \addplot3[
        mesh,
        samples=12,
        domain=-2:2,
      ]
      {0.1*x^2-0.1*y^2 + 0.4};

      \addplot3[
        surf,
        samples=3,
        domain=-1:0,
      ]
      {0.1*x^2-0.1*y^2 + 0.4};

    \end{axis}
  \end{tikzpicture}
\end{center}

Чтобы посчитать вероятность, что $X$ попадает в какое-то событие, надо просто посчитать объем подграфика на этом событии

\emph{NB:} Одномерные события имеют вероятность ноль. Например, событие $Y = X$.

\section{Маргинальные распределения}
 Покажем, что 

 \begin{align*}
   \begin{cases}
    f_X(x) &= \int_{-\infty}^{+\infty} f_{X, Y}(x, y) dy \\
    f_Y(y) &= \int_{-\infty}^{+\infty} f_{X, Y}(x, y) dx \\
   \end{cases}
 \end{align*}

 Для этого рассмотрим функцию распределения $X$:

\begin{align*}
  F_X(x) &= \Pr(X \le X) = \int_{-\infty}^x \left(\int_{-\infty}^{+\infty} f_{X, Y} (t, y) dy\right) dt \\
  f_X(x) &= F_X'(x) = \int_{-\infty}^{+\infty} f_{X, Y} (x, y) dy
\end{align*}

\textbf{Пример:} равномерное распределение на множестве $S$ площадью 4.5

\begin{align*}
  f_{X, Y} (x, y) = \begin{cases}
    \frac{1}{|S|}, \text{ если } (x, y) \in S,\\
    0, \text{ иначе}.
  \end{cases}
\end{align*}

\begin{center}
  \begin{tikzpicture}
    \begin{axis}[name = main, width=0.6\textwidth, height=8cm, xlabel={$x$}, ylabel={$y$}]
      \addplot[fill=blue!20, draw=blue, ultra thick]
      coordinates{(1, 1) (2, 1) (2, 2) (3, 2) (3, 3) (2, 4) (1, 4) (1, 1)};

      \node at (axis cs:2,2.5) {\huge{$S$}};
    \end{axis}
    \begin{axis}[at = {(main.south west)}, yshift=-4cm, height=4cm, width=0.6\textwidth, xlabel={$x$}, ylabel={$f_X(x)$}]
      \addplot[draw=none, fill=blue!20]
      coordinates{(1, 2/3) (2, 2/3) (2, 4/9) (3, 1/3) (3, 0) (1, 0)}\closedcycle;
      \addplot[draw=blue, ultra thick]
      coordinates{(1, 2/3) (2, 2/3)};
      \addplot[draw=blue, ultra thick]
      coordinates{(2, 4/9) (3, 1/3)};
    \end{axis}
    \begin{axis}[at = {(main.south west)}, xshift=-5cm, height=8cm, width=0.3\textwidth, xlabel={$f_Y(y)$}, ylabel={$y$}]
      \addplot[draw=none, fill=blue!20]
      coordinates{(2/9, 1) (2/9, 2) (4/9, 2) (4/9, 3) (2/9, 4) (0, 4) (0, 1) (2/9, 1)}\closedcycle;
      \addplot[draw=blue, ultra thick]
      coordinates{(2/9, 1) (2/9, 2)};
      \addplot[draw=blue, ultra thick]
      coordinates{(4/9, 2) (4/9, 3) (2/9, 4)};
    \end{axis}
  \end{tikzpicture}
\end{center}

\section{Более двух с.в. и функции от многих с.в.}

Совместное распределение может быть задано на более, чем одной с.в., тогда есть плотность вероятности $f_{X, Y, Z}(x, y, z)$. Все то же самое, что было со многими дискретными с.в., только вместо сумм интегралы:

\begin{itemize}
  \item $f_{X, Y, Z}(x, y, z) \ge 0$
  \item $\int_{-\infty}^{+\infty} \int_{-\infty}^{+\infty} \int_{-\infty}^{+\infty} f_{X, Y, Z}(x, y, z)dx dy dz = 1$
\end{itemize}

И на многих с.в. мы также можем задавать функции, которые будут по сути новыми с.в.: $Z = g(X, Y)$. Их матожидание считается так:

\begin{align*}
  E(g(X, Y)) = \int_{-\infty}^{+\infty} \int_{-\infty}^{+\infty} g(x, y) f_{X, Y}(x, y) dx dy
\end{align*}

При этом имеет место быть линейность матожидания:

\begin{align*}
  E(\sum_i a_i X_i) = \sum_i a_i E(X_i)
\end{align*}

\section{Совместная функция распределения}

Для нескольких с.в. определим функцию распределения:

\begin{align*}
  F_{X, Y}(x, y) = \Pr(X \le x \cap Y \le Y) = \int_{-\infty}^{x} \left( \int_{-\infty}^{y} f_{X, Y}(s, t) dt \right) ds
\end{align*}

Также можно заметить:
\begin{align*}
  f_{X, Y}(x, y) = \frac{\partial^2F_{X, Y}(x, y)}{\partial x \partial y}
\end{align*}

\section{С.в., условные на других с.в.}

Было в дискретном случае:

\begin{itemize}
  \item $p_{X, Y}(x, y)$ --- совместная функция вероятности
  \item $p_{X \mid A}(x) = \frac{p_X(x)\cdot [x \in A]}{\Pr(A)}$ --- условная функция вероятности (где $[\cdot]$ --- скобка Айверсона)
  \item $p_{X \mid Y}(x \mid y) = \frac{p_{X, Y}(x, y)}{p_Y(y)}$ --- функция вероятности $X$, условная на $Y$ (определена только для таких $y$, что $p_Y(y) > 0$)
\end{itemize}

То же самое есть для непрерывного случая:
\begin{itemize}
  \item $f_{X, Y}(x, y)$ --- совместная плотность вероятности
  \item $f_{X \mid A}(x) = \frac{f_X(x)\cdot [x \in A]}{\Pr(A)}$ --- условная плотность вероятности
  \item $f_{X \mid Y}(x \mid y) = \frac{f_{X, Y}(x, y)}{f_Y(y)}$ --- плотность вероятности $X$, условная на $Y$ (определена только для таких $y$, что $p_Y(y) > 0$)
\end{itemize}

Чуть ближе к формальному определению: 
\begin{align*}
  \Pr\left(X \in [x, x + \delta] \mid Y \in [y, y + \eps]\right) &= \frac{\Pr\left(X \in [x, x + \delta] \cap Y \in [y, y + \eps]\right)}{\Pr(Y \in [y, y + \eps])} \\
  &\approx \frac{f_{X, Y}(x, y) \delta \eps}{f_Y(y) \eps} = \frac{f_{X, Y}(x, y)}{f_Y(y)} \delta
\end{align*}

И совсем формальное. Если существует такая функция $f_{X \mid Y}(x \mid y)$, что для всех $y$ и для всех $A$ верно

\begin{align*}
  \Pr(X \in A \mid Y = y) = \int_A f_{X \mid Y}(x \mid y) dx,
\end{align*}
то эта функция называется условной плотностью вероятности $X$ при условии $Y$.

Опять при условиях у нас просто появляется новая вероятностная мера. Для всех $y: f_Y(y) > 0$ мы имеем
\begin{itemize}
  \item $f_{X \mid Y}(x \mid y) \ge 0$
  \item $\int_{-\infty}^{+\infty} f_{X \mid Y}(x \mid y) dx = \frac{\int_{-\infty}^{+\infty} f_{X, Y}(x \mid y) dx}{f_Y(y)} = 1$
\end{itemize}

Из последнего видно, что стоит воспринимать условную плотность как срез совместной плотности по какому-то значению с.в.. Заметьте, что при этом вероятность этого среза равна нулю (одномерное множество), поэтому только терминами условности на события тут не обойтись.

Работает правило умножения:
\begin{align*}
  f_{X, Y} (x, y) &= f_X(x) f_{Y \mid X}(y \mid x) \\
                  &= f_Y(y) f_{X \mid Y}(x \mid y) \\
\end{align*}

А значит, работают полные вероятность и матожидание. Полная вероятность (следует из правила умножения):
\begin{align*}
  f_X(x) = \int_{-\infty}^{+\infty} f_Y(y) f_{X \mid Y}(x, y) dy
\end{align*}
Также определено условное матожидание
\begin{align*}
  E(X \mid Y = y) = \int_{-\infty}^{+\infty} x f_{X \mid Y}(x, y) dx
\end{align*}
и работает теорема о полном матожидании (TODO: доказать)
\begin{align*}
  E(X) = \int_{-\infty}^{+\infty} f_{Y}(y)E(X \mid Y = y) dy
\end{align*}

Также можно посчитать условное матожидание функции с.в.
\begin{align*}
  E(g(X) \mid Y = y) = \int_{-\infty}^{+\infty} g(x) f_{X \mid Y}(x, y) dx
\end{align*}

\section{Независимость с.в.}

Было у дискретных:

\begin{align*}
  p_{X, Y}(x, y) = p_X(x)p_Y(y)
\end{align*}

Логично определить непрерывные с.в. $X$ и $Y$ независимыми, если

\begin{align*}
  f_{X, Y}(x, y) = f_X(x) f_Y(y)
\end{align*}

Это автоматически подразумевает, что для всех $y$ верно, что $f_{X \mid Y}(x \mid y) = f_X(x)$.

Свойств независимых с.в. те же, что и у дискретных
\begin{itemize}
  \item $E(XY) = E(X) E(Y)$
  \item $\Var(X + Y) = \Var(X) + \Var(Y)$
  \item $g(X)$ и $h(Y)$ тоже будут независимы  
\end{itemize}

\textbf{Пример про зависимые с.в.}

Есть палка длиной $\ell$. Ломаем ее в случайном месте $X\in [0, \ell]$, остаток тоже ломаем в случайном месте $Y \in [0, X]$. Найти:
\begin{itemize}
  \item $f_{X, Y}$ 
  \item $f_Y(y)$
  \item $E[Y]$
\end{itemize}

\end{document}
