\documentclass[12pt]{article}

\usepackage{a4wide}

\usepackage[utf8]{inputenc} 
\usepackage[russian]{babel}
\usepackage{amssymb}
\usepackage{amsmath}
\usepackage{pgfplots}
\usepgfplotslibrary{statistics,patchplots}
\usetikzlibrary{decorations.pathreplacing,calc,tikzmark, patterns,arrows.meta}
\pgfmathdeclarefunction{gauss}{3}{%
  \pgfmathparse{1/(#3*sqrt(2*pi))*exp(-((#1-#2)^2)/(2*#3^2))}%
}
\pgfmathdeclarefunction{gauss2d}{6}{%
  \pgfmathparse{1/(#3*#6*2*pi)*exp(-((#1-#2)^2)/(2*#3^2) - ((#4-#5)^2)/(2*#6^2))}%
}

\usepackage{xspace}

\usepackage{mathtools}
\usepackage{cite}
\usepackage{array}
\usepackage{multirow}
\usepackage{tabularx}

\newcommand\N{\mathbb{N}}
\newcommand\R{\mathbb{R}}
\newcommand\eps{\varepsilon}
\DeclareMathOperator{\Bin}{Bin}
\DeclareMathOperator{\Geom}{Geom}
\DeclareMathOperator{\pow}{pow}
\DeclareMathOperator{\Bern}{Bern}
\DeclareMathOperator{\Exp}{Exp}
\DeclareMathOperator{\Var}{Var}
\DeclareMathOperator{\Cov}{Cov}
\DeclareMathOperator{\sign}{sign}


\title{План на второй модуль}

\begin{document}
% \maketitle

\begin{enumerate}
  \item Разные полезные инструменты
  \begin{itemize}
    \item Неравенство Маркова
    \item Неравенство Чебышева
    \item Неравенство Хёффдинга
    \item Границы Чернова
    \item Без доказательств (если это вообще полезно): неравенство Беннета, неравнество Бернштейна, неравенство Эффрона-Штейна, неравенство Дворецкого-Кайфера-Вольфовица
  \end{itemize}
  \item Сходимость с.в.
  \begin{itemize}
    \item По мере
    \item С вероятностью 1
  \end{itemize}
  \item Слабый и сильный законы больших чисел
  \item Центральная предельная теорема и ее вариации
  \item Равенство Вальда
  \item Случайные процессы
  \begin{itemize}
    \item Мартингалы
    \begin{itemize}
      \item Неравенство Азумы-Хёффдинга
      \item Мартингал Дуба
      \item Неравенство МакДиармида
    \end{itemize}
    \item Процесс Бернулли
    \item Процесс Пуассона
    \item Марковский процесс
    \begin{itemize}
      \item Марковские цепи
    \end{itemize}
  \end{itemize}
  \item Доп темы, если останется время
  \begin{itemize}
    \item Генерация псевдослучайных чисел
    \item Дрифт-теоремы (теоремы сноса)
  \end{itemize}
\end{enumerate}


\end{document}