\documentclass[12pt]{article}

\usepackage{a4wide}

\usepackage[utf8]{inputenc} 
\usepackage[russian]{babel}
\usepackage{amssymb}
\usepackage{amsmath}
\usepackage{pgfplots}
\usepgfplotslibrary{statistics}
\usetikzlibrary{decorations.pathreplacing,calc,tikzmark, patterns,arrows.meta}
\pgfmathdeclarefunction{gauss}{3}{%
  \pgfmathparse{1/(#3*sqrt(2*pi))*exp(-((#1-#2)^2)/(2*#3^2))}%
}

\usepackage{xspace}

\usepackage{mathtools}
\usepackage{cite}
\usepackage{array}
\usepackage{multirow}
\usepackage{tabularx}

\newcommand\N{\mathbb{N}}
\newcommand\R{\mathbb{R}}
\newcommand\eps{\varepsilon}
\DeclareMathOperator{\Bin}{Bin}
\DeclareMathOperator{\Geom}{Geom}
\DeclareMathOperator{\pow}{pow}
\DeclareMathOperator{\Bern}{Bern}
\DeclareMathOperator{\Var}{Var}


\title{Лекция 3. Непрерывные с.в., часть 1}

\begin{document}
\maketitle


  \section{Непрерывные случайные величины}

  \textbf{Определение непрерывных с.в.}

  В случае с дискретными величинами:

  \begin{center}
    \begin{tikzpicture}
        \begin{axis}[ybar, ymin=0,
        grid=major,
        ylabel={$p_{X}(x)$},
        xlabel={$x$}]
        \addplot
        [draw=blue, fill=blue] 
        coordinates
            {(1, 0.1) (2, 0.2) (3,0.2) (4,0.1) (5,0.3) (6,0.1)};
        \end{axis}
        \node (a) [red] at (2.3, 0) {\textbf{\huge{[}}};
        \node (b) [red] at (5.7, 0) {\textbf{\huge{]}}};
        \node [below=1em, red] at (a) {\textbf{a}};
        \node [below=1em, red] at (b) {\textbf{b}};

        \draw [red, ultra thick] (2.3, 0) -- (5.7, 0);
    \end{tikzpicture}
  \end{center}

  \begin{align*}
      \Pr[a \le X \le b] = \sum_{x: a \le x \le b} p_X(x)
  \end{align*}

  Но если $X$ может принимать любые вещественные значения из этого интервала? Тогда нам нужна функция, которая показывает, сколько вероятностной массы лежит на каждом элементарном отрезке.

  \begin{center}
    \begin{tikzpicture}
        \begin{axis}[ybar, ymin=0, ymax=0.3, xmin=-0.1, xmax=5.1,
        grid=major,
        ylabel={$f_{X}(x)$},
        xlabel={$x$}]
        \end{axis}
        \fill [blue!20] (2, 3) to[out=0,in=180] (4, 1) to[out=0,in=180] (6, 4) -- (6, 0) -- (2, 0);

        \draw [thick] (0, 1) to[out=45,in=180] (2, 3)
                     to[out=0,in=180] (4, 1)
                     to[out=0,in=180] (6, 4)
                     to[out=0, in=150] (6.9, 3);

        \node (a) [red] at (2, 0) {\textbf{\huge{[}}};
        \node (b) [red] at (6, 0) {\textbf{\huge{]}}};
        \node [below=1em, red] at (a) {\textbf{a}};
        \node [below=1em, red] at (b) {\textbf{b}};

        \draw [red, ultra thick] (2, 0) -- (6, 0);
    \end{tikzpicture}
  \end{center}

  \begin{align}\label{eq:def-contnuous}
      \Pr[a \le X \le b] = \int_a^b f_X(x) dx 
  \end{align}

  \emph{Определение:} Случайная величина называется непрерывной, если для нее существует такая функция $f_X(x)$, что для любых $a, b \in \R$ (где $a \le b$) верно~\eqref{eq:def-contnuous}.


  $f_X(x)$ --- \emph{плотность вероятности} с.в. $X$:

  \begin{align*}
      \Pr(a \le X \le a + \eps) \approx f_X(a) \cdot \varepsilon 
  \end{align*}

  Плотность вероятности --- аналог функции вероятностей для непрерывных с.в.:

  \begin{itemize}
      \item $p_X(x) \ge 0$
      \item $\sum_x p_X(x) = 1$
  \end{itemize}

  То же самое
  \begin{itemize}
      \item $f_X(x) \ge 0$
      \item $\int_{-\infty}^{+\infty} p_X(x) = 1$
  \end{itemize}

  \emph{NB}:
  
  \begin{align*}
      \Pr(X = a) = \Pr(a \le X \le a) = \int_a^a f_X(x) dx = 0
  \end{align*}

  Поэтому:
  \begin{align*}
      \Pr(a \le X \le b) = \Pr(X = a) + \Pr(X = b) + \Pr(a < X < b) = \Pr(a < X < b) 
  \end{align*}

  \emph{NB}: Мы ушли от понятия событий, но у нас по-прежнему есть какая-то $\Omega$, на которой и задана с.в. $X$. Просто нам сейчас проще быть чисто в терминах с.в.


  \emph{NB}: Переопредилим дискретные с.в. как с.в., для которых есть функция вероятностей, то есть число возможныъх значений которых счетно.

  \emph{NB}: Вы уже могли догадаться, что с.в. могут быть и смешанные, но про это позже.

  
  Пример: равномерное распределение

  \begin{center}
    \begin{tikzpicture}
        \begin{axis}[ybar, ymin=0, ymax=0.3,
        grid=major,
        ylabel={$p_{X}(x)$},
        xlabel={$x$}]
        \addplot
        [draw=blue, fill=blue] 
        coordinates
            {(1, 0.2) (2, 0.2) (3,0.2) (4,0.2) (5,0.2)};
        \end{axis}
    \end{tikzpicture}
  \end{center}

  \begin{center}
    \begin{tikzpicture}
        \begin{axis}[ybar, ymin=0, ymax=0.3, xmin=0, xmax=6,
        grid=major,
        ylabel={$f_{X}(x)$},
        xlabel={$x$}]
        \end{axis}
        \fill [blue!20] (1.15, 3.8) -- (5.7, 3.8) -- (5.7, 0) -- (1.15, 0);

        \draw [ultra thick] (0, 0) -- (1.15, 0);
        \draw [ultra thick] (1.15, 3.8) -- (5.7, 3.8);
        \draw [ultra thick] (5.7, 0) -- (6.85, 0);
        

    \end{tikzpicture}
  \end{center}

  Обобщение: частично равномерное распределение


  \begin{center}
    \begin{tikzpicture}
        \begin{axis}[ybar, ymin=0, ymax=0.4, xmin=0, xmax=6,
        grid=major,
        ylabel={$f_{X}(x)$},
        xlabel={$x$}]
        \end{axis}
        \fill [blue!20] (1.15, 2.8) -- (3.45, 2.8) -- (3.45, 1.4) -- (4.6, 1.4) -- (4.6, 4.2) -- (5.7, 4.2) -- (5.7, 0) -- (1.15, 0);
         

        \draw [ultra thick] (0, 0) -- (1.15, 0);
        \draw [ultra thick] (1.15, 2.8) -- (3.45, 2.8);
        \draw [ultra thick] (3.45, 1.4) -- (4.6, 1.4);
        \draw [ultra thick] (4.6, 4.2) -- (5.7, 4.2);
        \draw [ultra thick] (5.7, 0) -- (6.85, 0);
        

    \end{tikzpicture}
  \end{center}

  \section{Матожидание}

  Для дискретных величин:
  \begin{align*}
      E(X) = \sum_x x p_X(x)
  \end{align*}

  Для непрерывных: заменяем сумму на интеграл, а функцию вероятностей на плотность вероятности
  \begin{align*}
      E(X) = \int_{-\infty}^{+\infty} x f_X(x) dx
  \end{align*}

  Важно: интеграл должен сходиться абсолютно

  \emph{Интерпретация:} центр масс вероятностной массы

  \textbf{Свойства матожидания}

  \begin{itemize}
    \item $X \ge 0 \Rightarrow E[X] > 0$
    \item $X \in [a, b] \Rightarrow E[X] \in [a, b]$
    \item Матожидание функции от с.в.:
    \begin{align*}
        E(g(X)) = \int_{-\infty}^{+\infty} g(x) f_X(x) dx
    \end{align*}
    \item Пример
    \begin{align*}
        E(X^2) = \int_{-\infty}^{+\infty} x^2 f_X(x) dx
    \end{align*}
    \item Линейность: $E(aX + b) = aE(X) + b$
  \end{itemize}

  \section{Дисперсия}

  Как и для дискретных:
  \begin{align*}
      \Var(X) = E((X - \mu)^2) = \int_{-\infty}^{+\infty} (x - \mu)^2 f_X(x) dx
  \end{align*}

  среднеквадратичное отклонение:
  \begin{align*}
      \sigma_X = \sqrt{\Var(X)}
  \end{align*}

  Свойства --- те же:
  \begin{itemize}
      \item $\Var(aX + b) = a^2\Var(X)$
      \item $\Var(X) = E(X^2) - (E(X))^2$
  \end{itemize}

  \section{Моменты стандартных распределений}

  \textbf{Равномерное распределение}

  $X \sim U(a, b)$ 

  Матожидание:
  \begin{align*}
    E[X] = \int_{-\infty}^{+\infty} x f_X(x) dx = \int_{a}^{b} x \frac{1}{b - a} dx = \frac{a + b}{2}. 
  \end{align*}

  Дисперсия:
  \begin{align*}
    E[X^2] &= \int_{-\infty}^{+\infty} x^2 f_X(x) dx = \int_{a}^{b} x^2 \frac{1}{b - a} dx \\
           &= \frac{b^3 - a^3}{3(b - a)} = \frac{b^2 + ab + a^2}{3}\\
    \Var(X) &= E[X^2] - (E[X])^2 = \frac{4b^2 + 4ab + 4a^2 - 3b^2 - 6ab - 3a^2}{12} = \frac{(b - a)^2}{12}. 
  \end{align*}

  \textbf{Экспоненциальное распределение}

  Говорим, что $X$ следует экспоненциальному распределению с параметром $\lambda$, если

  \begin{align*}
    f_X(x) = \begin{cases}
      \lambda e^{-\lambda x}, &x \ge 0 \\
      0, &x < 0
    \end{cases}
  \end{align*}

  \emph{NB}: Это аналог геометрического распределения с параметром $p = \lambda$.

  Матожидание:
  \begin{align*}
    E[X] &= \int_0^{+\infty} x \lambda e^{-\lambda x} dx = - \int_0^{+\infty} x d e^{-\lambda x} = - x e^{-\lambda x} \bigg|_0^{+\infty} + \int_0^{+\infty} e^{-\lambda x} dx \\
    &= 0 - \frac{1}{\lambda} \bigg|_0^{+\infty} = \frac{1}{\lambda}.
  \end{align*}

  Дисперсия (два раза интегрируем по частям):
  \begin{align*}
    E[X^2] &= \int_0^{+\infty} x^2 \lambda e^{-\lambda x} dx = \frac{2}{\lambda}. \\
    \Var(X) &= \frac{2}{\lambda^2} - \frac{1}{\lambda^2} = \frac{1}{\lambda^2}.
  \end{align*}

  Довольно хорошо сконцентрирована, так как вероятность хвоста экспоненциально падает:
  \begin{align*}
    \Pr(X \ge a) = \int_a^{+\infty} \lambda e^{-\lambda x} dx = e^{-\lambda a}.
  \end{align*}


  \section{Функция распределения}

  $F_X(x) = \Pr(X \le x)$ --- \emph{функция распределения} с.в. $X$ (как дискретной, так и непрерывной).


  Как считать:

  \begin{align*}
    F_X(x) = \int_{-\infty}^x f_X(t) dt
  \end{align*}

  Легко заметить: $F_X'(x) = f_X(x)$

  Пример: равномерное распределение.

  \begin{center}
    \begin{tikzpicture}
        \begin{axis}[ymin=-0.1, ymax=1, xmin=0, xmax=5,
        width = 0.45\textwidth,
        grid=major,
        ylabel={$f_{X}(x)$},
        xlabel={$x$},
        ultra thick]
        \addplot[draw=blue,fill=blue!30!white] coordinates
          {(0,0) (1,0)};
          \addplot[draw=none, fill=blue!30!white] coordinates
          {(1,0.33) (4,0.33)} \closedcycle;
        \addplot[draw=blue] coordinates
          {(1,0.33) (4,0.33)};
        \addplot[draw=blue,fill=blue!30!white] coordinates
          {(4,0) (5,0)};
        \end{axis}
    \end{tikzpicture}
    \begin{tikzpicture}
      \begin{axis}[ymin=-0.1, ymax=1.1, xmin=0, xmax=5,
      width = 0.45\textwidth,
      grid=major,
      ylabel={$F_{X}(x)$},
      xlabel={$x$},
      ultra thick]
      \addplot[draw=blue] coordinates
        {(0,0) (1,0) (4, 1) (5, 1)};
      \end{axis}
  \end{tikzpicture}
  \end{center}

   Функцию распределения можно считать и для дискретной случайной величины:

   \begin{center}
    \begin{tikzpicture}
        \begin{axis}[ybar, ymin=0, ymax=1,
        width = 0.45\textwidth,
        grid=major,
        ylabel={$p_{X}(x)$},
        xlabel={$x$}]
        \addplot
        [draw=blue, fill=blue] 
        coordinates
            {(1, 0.25) (3, 0.5) (4,0.25)};
        \end{axis}
    \end{tikzpicture}
    \begin{tikzpicture}
      \begin{axis}[ymin=-0.1, ymax=1.1, xmin=0, xmax=5,
      width = 0.45\textwidth,
      grid=major,
      ylabel={$F_{X}(x)$},
      xlabel={$x$},
      ultra thick]
      \addplot[draw=blue] coordinates
        {(0,0) (1,0)};
      \addplot[draw=blue] coordinates
        {(1,0.25) (3,0.25)};
      \addplot[draw=blue] coordinates
        {(3, 0.75) (4, 0.75)};
      \addplot[draw=blue] coordinates
        {(4, 1) (5, 1)};
      \addplot[only marks, draw=blue] coordinates
        {(1,0.25) (3, 0.75) (4, 1)};
      \end{axis}
  \end{tikzpicture}
  \end{center}

  Свойства функции распределения:
  \begin{itemize}
    \item Неубывающая
    \item $\lim_{x \to + \infty} F_X(x) = 1$
    \item $\lim_{x \to - \infty} F_X(x) = 0$
  \end{itemize}

  \section{Нормальное распределение (распределение Гаусса)}

  Очень важная штука:
  \begin{itemize}
    \item Важна в центральной предельной теореме
    \item Часто на практике неизвестные распределения приближаются нормальным
  \end{itemize}

  \emph{Стандартное нормальное:} $X \sim N(0, 1) \leftrightarrow f_X(x) = \frac{1}{\sqrt{2\pi}}e^{-x^2/2}$

  \begin{center}
    \begin{tikzpicture}
      \begin{axis}[
      grid=major,
      ylabel={$f_{X}(x)$},
      xlabel={$x$}]
      \addplot[domain=-3:3, samples=61, draw=blue, ultra thick]{gauss(x, 0, 1)};
      \addplot[domain=-3:3, samples=61, draw=none, fill=blue!20]{gauss(x, 0, 1)} \closedcycle;
      \end{axis}
  \end{tikzpicture}
  \end{center}

  Откуда берется коэффициент нормализации $\frac{1}{\sqrt{2\pi}}$? Из интеграла Гаусса
  \begin{align*}
    \int_{-\infty}^{+\infty} e^{-x^2} dx = \sqrt{\pi}
  \end{align*}

  Свойства $N(0, 1)$:

  \begin{align*}
    E[X] = \int_{-\infty}^{+\infty} x \frac{1}{\sqrt{2\pi}}e^{-x^2/2} dx = 0,
  \end{align*}
  так как это интеграл нечетной функции, которая в бесконечности очень маленькая.

  \begin{align*}
    \Var(X) &= E[X^2] = \int_{-\infty}^{+\infty} x^2 \frac{1}{\sqrt{2\pi}}e^{-x^2/2} dx = \int_{-\infty}^{+\infty} x \frac{1}{\sqrt{2\pi}}e^{-x^2/2}d\frac{x^2}{2} \\
            &= - \frac{1}{\sqrt{2\pi}} \int_{-\infty}^{+\infty} x \frac{1}{\sqrt{2\pi}}e^{-x^2/2}de^{-x^2/2} \\
            &= - \frac{1}{\sqrt{2\pi}} x e^{-x^2/2} \bigg|_{-\infty}^{+\infty} + \frac{1}{\sqrt{2\pi}} \int_{-\infty}^{+\infty} e^{-x^2/2} dx \\
            &= 0 + \frac{\sqrt{2\pi}}{\sqrt{2\pi}} = 1.
  \end{align*}

  В записи $N(0, 1)$ нолик как раз обозначает матожидание, а единица --- дисперсию

  Обобщенное нормальное распределение: $X \sim N(\mu, \sigma^2) \Leftrightarrow f_X(x) = \frac{1}{\sigma\sqrt{2\pi}}e^{-\frac{(x - \mu)^2}{2\sigma^2}}$

  \begin{itemize}
    \item $\mu$ --- новое матожидание, насколько мы сдвигаем ось симметрии $X$.
    \item $\sigma$ --- новая дисперсия, насколько мы растягиваем распределение от оси симметрии.
  \end{itemize}

  \begin{center}
    \begin{tikzpicture}
      \begin{axis}[
      grid=major,
      ylabel={$f_{X}(x)$},
      xlabel={$x$},
      legend pos=outer north east]
      \addplot[domain=-3:5, samples=81, draw=blue, thick]{gauss(x, 0, 1)};
      \addlegendentry{$\mu = 0, \sigma = 1$}
      \addplot[domain=-3:5, samples=81, draw=red, thick]{gauss(x, 2, 1)};
      \addlegendentry{$\mu = 2, \sigma = 1$}

      \addplot[domain=-3:5, samples=81, draw=none, fill=blue, fill opacity=0.2]{gauss(x, 0, 1)} \closedcycle;
      \addplot[domain=-3:5, samples=81, draw=none, fill=red, fill opacity=0.2]{gauss(x, 2, 1)} \closedcycle;
      \end{axis}
  \end{tikzpicture}
  \end{center}
  
  Чем меньше $\sigma$ (срежнеквадратичное отклонение), тем больше распределение сжато вокруг оси симметрии
  \begin{center}
    \begin{tikzpicture}
      \begin{axis}[
      grid=major,
      ylabel={$f_{X}(x)$},
      xlabel={$x$},
      legend pos=outer north east]
      \addplot[domain=-3:3, samples=81, draw=blue, thick]{gauss(x, 0, 1)};
      \addlegendentry{$\mu = 0, \sigma = 1$}
      \addplot[domain=-3:3, samples=81, draw=red, thick]{gauss(x, 0, 2)};
      \addlegendentry{$\mu = 0, \sigma = 4$}
      \addplot[domain=-3:3, samples=81, draw=green!40!black, thick]{gauss(x, 0, 0.5)};
      \addlegendentry{$\mu = 0, \sigma = 0.25$}

      % \addplot[domain=-3:5, samples=81, draw=none, fill=blue, fill opacity=0.2]{gauss(x, 0, 1)} \closedcycle;
      % \addplot[domain=-3:5, samples=81, draw=none, fill=red, fill opacity=0.2]{gauss(x, 2, 1)} \closedcycle;
      \end{axis}
  \end{tikzpicture}
  \end{center}

\end{document}