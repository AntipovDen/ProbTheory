\documentclass[12pt]{article}

\usepackage{a4wide}

\usepackage[utf8]{inputenc} 
\usepackage[russian]{babel}
\usepackage{amssymb}
\usepackage{amsmath}
\usepackage{pgfplots}
\usepgfplotslibrary{statistics,patchplots}
\usetikzlibrary{decorations.pathreplacing,calc,tikzmark, patterns,arrows.meta}
\pgfmathdeclarefunction{gauss}{3}{%
  \pgfmathparse{1/(#3*sqrt(2*pi))*exp(-((#1-#2)^2)/(2*#3^2))}%
}
\pgfmathdeclarefunction{gauss2d}{6}{%
  \pgfmathparse{1/(#3*#6*2*pi)*exp(-((#1-#2)^2)/(2*#3^2) - ((#4-#5)^2)/(2*#6^2))}%
}

\usepackage{xspace}

\usepackage{mathtools}
\usepackage{cite}
\usepackage{array}
\usepackage{multirow}
\usepackage{tabularx}

\newcommand\N{\mathbb{N}}
\newcommand\R{\mathbb{R}}
\newcommand\eps{\varepsilon}
\DeclareMathOperator{\Bin}{Bin}
\DeclareMathOperator{\Geom}{Geom}
\DeclareMathOperator{\pow}{pow}
\DeclareMathOperator{\Bern}{Bern}
\DeclareMathOperator{\Exp}{Exp}
\DeclareMathOperator{\Var}{Var}


\title{Лекция 7. Функции от с.в.}

\begin{document}
\maketitle

\section{Линейные преобразования}

Как это в общем случае делать посчитать новое распределение линейной функции от с.в.? Рассмотрим $Y = aX + b$ (где $a \ne 0$, иначе это скучный случай, когда $Y = b$ с вероятностью 1). Пусть сначала $X$ дискретная, и нам известна ее функция вероятностей. Тогда

\begin{align*}
  p_Y(y) = \Pr(Y = y) = \Pr(aX + b = y) = \Pr\left(X = \frac{y - b}{a}\right) = p_X\left( \frac{y - b}{a}\right)
\end{align*}

С непрерывными такое не работает, так как вероятность, что непрерывная с.в. равна конкретному числу, есть ноль. Но мы можем работать с функциями распределения! Допустим мы знаем $f_X(x)$ и $F_X(x)$. Рассмотрим случай $a > 0$

\begin{align*}
  F_Y(y) &= \Pr(Y \le y) = \Pr(aX + b \le y) = \Pr\left(X \le \frac{y - b}{a}\right) = F_X\left(\frac{y - b}{a}\right), \\
  f_Y(y) &= F_Y'(y) = f_X\left(\frac{y - b}{a}\right) \cdot \frac{1}{a}.
\end{align*}

И отдельно $a < 0$
\begin{align*}
  F_Y(y) &= \Pr(Y \le y) = \Pr(aX + b \le y) = \Pr\left(X \ge \frac{y - b}{a}\right) = 1 - F_X\left(\frac{y - b}{a}\right), \\
  f_Y(y) &= F_Y'(y) = -f_X\left(\frac{y - b}{a}\right) \cdot \frac{1}{a}.
\end{align*}

Объединяя два случая, получаем:

\begin{align*}
  f_Y(y) = \frac{1}{|a|}f_X\left(\frac{y - b}{a}\right).
\end{align*}

Докажем теперь, что линейное преобразование нормального распределения оставляет его нормальным. Пусть $X \sim N(\mu, \sigma^2)$ и $Y = aX + b$. Значит,

\begin{align*}
  f_Y(y) &= \frac{1}{|a|}f_X\left(\frac{y - b}{a}\right) \\
         &= \frac{1}{|a|\sigma \sqrt{2\pi}} \exp\left(\frac{\left(\frac{y - b}{a} - \mu\right)^2}{2\sigma^2}\right) \\
         &= \frac{1}{(|a|\sigma) \sqrt{2\pi}} \exp\left(\frac{\left(y - (b + a\mu)\right)^2}{2(a\sigma)^2}\right),
\end{align*}
что есть функция плотности вероятности для $N(a\mu + b, (a\sigma)^2)$.

\section{Нелинейные преобразования}

Алгоритм поиска нового распределения $Y = g(X)$:
\begin{enumerate}
  \item Найти функцию распределения $F_Y(y) = \Pr(Y \le y) = \Pr(g(X) \le y)$
  \item Продифференцировать: $f_Y(y) = F_Y'(y)$.
\end{enumerate}

Рассмотрим случай, когда $g(x)$ монотонно возрастает и дифференцируема.

\begin{align*}
  F_Y(y) &= \Pr(Y \le y) = \Pr(g(X) \le y) = \Pr(X \le g^{-1}(y)) = F_X(g^{-1}(y)) \\
  f_Y(y) &= F_Y'(y) = F_X'(g^{-1}(y)) = f_X(g^{-1}(y)) (g^{-1}(y))' = \frac{f_X(g^{-1}(y))}{g'(g^{-1}(y))}
\end{align*}

Аналогично можно рассмотреть случай, когда $g(x)$ монотонно убывает и дифференцируема
\begin{align*}
  F_Y(y) &= \Pr(Y \le y) = \Pr(g(X) \le y) = \Pr(X \ge g^{-1}(y)) = 1 - F_X(g^{-1}(y)) \\
  f_Y(y) &= F_Y'(y) = -F_X'(g^{-1}(y)) = -f_X(g^{-1}(y)) (g^{-1}(y))' = -\frac{f_X(g^{-1}(y))}{g'(g^{-1}(y))} = \frac{f_X(g^{-1}(y))}{|g'(g^{-1}(y))|}
\end{align*}

То есть в общем случае (когда $g(x)$ строго монотонна):
\begin{align*}
  f_Y(y) = \frac{f_X(g^{-1}(y))}{|g'(g^{-1}(y))|}
\end{align*}

Интуитивное объяснение (которое использовалось на практике). Пусть $y = g(x)$

\begin{align*}
  f_Y(y) \delta_y \approx \Pr(y \le Y \le y + \delta_y) = \Pr(x \le X \le x + \delta_y) = f_X(x) \delta_x = f_X(x) \frac{\delta_y}{g'(x)}
\end{align*}

\begin{center}
  \begin{tikzpicture}
      \begin{axis}[xmin=0, xmax=4, ymin=0, ymax=4,
      grid=major,
      ylabel={$Y = g(X)$},
      xlabel={$X$}, smooth]
        \addplot [blue] coordinates {(0, 0) (1, 1.2) (1.5, 1.4) (1.7, 1.5) (2.2, 2) (3, 3) (3.5, 4)};
        \coordinate (x) at (axis cs: 1.5,0);
        \coordinate (dx) at (axis cs:1.7,0);
        \coordinate (y) at (axis cs: 0,1.4);
        \coordinate (dy) at (axis cs:0,1.5);

        \coordinate (xy) at (axis cs:1.5,1.4);
        \coordinate (dxy) at (axis cs:1.7,1.5);

      \end{axis}

      \draw [dashed] (x) -- (xy) -- (y);
      \draw [dashed] (dx) -- (dxy) -- (dy);

      \draw [ultra thick, red] (x) -- (dx) node [pos=0.5, below, black] {$\delta_x$};
      \draw [ultra thick, blue] (y) -- (dy) node (cy) [pos=0.5] {};

      \node (label) at (-2, 0) {$\delta_y = \delta_x g'(x)$};
      \draw [->] (label) -- (cy); 

      \draw [ultra thick] (xy) -- (dxy);
  \end{tikzpicture}
\end{center}

Для немонотонных $g(x)$ алгоритм остается прежний, но нет гарантий, что все пройдет легко. рассмотрим $g(x) = x^2$ и $Y = g(X)$ (при известной плотности $X$).

\begin{align*}
  F_Y(y) &= \Pr(Y \le y) = \Pr(X^2 \le y) = 1 - \Pr(X^2 > y) \\
         &= 1 - \Pr(X > \sqrt{y}) - \Pr(X < -\sqrt{y}) \\
         &= \Pr(X \le \sqrt{y}) = \Pr(X < -\sqrt{y}) = F_X(\sqrt{y}) - F_X(-sqrt{y}) \\
  f_Y(y) &= F_Y'(y) = F_X'(\sqrt{y}) - F_X'(-\sqrt{y}) = f_X(\sqrt{y}) \frac{1}{2\sqrt{y}} - f_X(-\sqrt{y}) \left(-\frac{1}{2\sqrt{y}}\right) \\
         &= \frac{f_X(\sqrt{y}) + f_X(-\sqrt{y}) }{2\sqrt{y}}  
\end{align*}

\section{Функции нескольких с.в.}

Если есть $Z = g(X, Y)$, то алгоритм нахождения ее функции распределения или плотности такой же. Рассмотрим на примере. $X, Y$ -- независимые, обе следуют $U(0, 1)$. Рассмотрим с.в. $Z = \frac{Y}{X}$.

Найдем функцию распределения $F_Z(z)$

\begin{align*}
  F_Z(z) = \Pr(\frac{Y}{X} \le z).
\end{align*}


\begin{center}
  \begin{tikzpicture}
      \begin{axis}[xmin=0, xmax=1.2, ymin=0, ymax=1.2,
      grid=major,
      ylabel={$Y = g(X)$},
      xlabel={$X$}]
        \addplot [draw=none, fill=blue!20] coordinates {(0, 0) (0.5, 1) (1, 1) (1, 0)} \closedcycle;
        \addplot [draw=none, fill=red!20] coordinates {(0, 0) (1, 0.5) (1, 0)} \closedcycle;
        \addplot [black, ultra thick] coordinates {(0, 1) (1, 1) (1, 0)};
        \addplot [red] coordinates {(0, 0) (1.2, 0.6)};
        \addplot [blue] coordinates {(0, 0) (0.6, 1.2)};
      \end{axis}
  \end{tikzpicture}
\end{center}

Если $z$ меньше единицы, то это просто интеграл по красному треугольнику, то есть его площадь (так как совместная плотность $f_{X, Y}(x, y) = 1$ на всем квадрате).

\begin{align*}
  F_Z(z) = \frac{z}{2}, \text{ если } z \in [0, 1].
\end{align*}

Если $z > 1$ , то это площадь всего квадрата минус площадь незакрашенного треугольника

\begin{align*}
  F_Z(z) = 1 - \frac{1}{2z}.
\end{align*}

Объединяя все вместе, получаем:

\begin{align*}
  F_Z(z) = \begin{cases}
    0, &\text{ если } z < 0, \\
    \frac{z}{2}, &\text{ если } z \in [0, 1), \\
    1 - \frac{1}{2z}, &\text{ если } z \ge 1. \\
  \end{cases}
\end{align*}

\subsection{Сумма независимых с.в.}

В данном подразделе мы имеем две независимых с.в. $X$ и $Y$, знаем их распределение и хотим узнать распределение с.в. $Z = X + Y$. 

\textbf{Дискретный случай}

Чтобы посчитать вероятность того, что $Z = z$, необходимо, чтобы одновременно выполнялось:
\begin{itemize}
  \item $X = x$ (где $x$ --- какое-то число из множества значений $X$)
  \item $Y = z - x$.
\end{itemize}

То есть мы берем все пары $(x, y)$ с прямой $x + y = z$ и суммируем вероятности этих пар.

\begin{center}
  \begin{tikzpicture}
    \begin{axis}[xmin=-1, xmax=3, ymin=-1, ymax=3,
      grid=major,
      ylabel={$y$},
      xlabel={$x$}]
      \addplot coordinates {(-2, 4) (-0.5, 2.5) (0, 2) (0.5, 1.5) (1, 1) (2, 0) (4, -2)};
      \coordinate (a1) at (axis cs:-0.5, 2.5);
      \coordinate (a2) at (axis cs:0, 2);
      \coordinate (a3) at (axis cs:0.5, 1.5);
      \coordinate (a4) at (axis cs:1, 1);
      \coordinate (a5) at (axis cs:2, 0);
    \end{axis}

    \node [above right] at (a1) {$(-0.5, 2.5)$};
    \node [above right] at (a2) {$(0, 2)$};
    \node [above right] at (a3) {$(0.5, 1.5)$};
    \node [above right] at (a4) {$(1, 1)$};
    \node [above right] at (a5) {$(2, 0)$};

    
  \end{tikzpicture}
\end{center}

Заметим, что из-за независимости $X$ и $Y$ события $X = x$ и $Y = z - x$ независимы, поэтому 
\[
  \Pr(X = x \cap Y = z - x) = p_X(x) p_Y(z - x). 
\]
Таким образом, формула выглядит так:

\begin{center}
  \begin{tikzpicture}[rounded corners]
    \node [draw, rectangle, fill=blue!20, minimum height = 1.5cm, minimum width = 4.5cm] at (0,0) {
    \begin{minipage}{0.4\textwidth}
      \[
        p_{Z}(z) = \sum_{x} p_X(x) p_Y(z - x)  
      \]
    \end{minipage}};
  \end{tikzpicture}
\end{center}

\textbf{Непрерывный случай}

Аналогичная формула для непрерывных с.в.:
\begin{center}
  \begin{tikzpicture}[rounded corners]
    \node [draw, rectangle, fill=blue!20, minimum height = 1.5cm, minimum width = 4.5cm] at (0,0) {
    \begin{minipage}{0.4\textwidth}
      \[
        f_{Z}(z) = \int_{-\infty}^{+\infty} f_X(x) f_Y(z - x) dx  
      \]
    \end{minipage}};
  \end{tikzpicture}
\end{center}
Чтобы ее доказать, обусловимся сначала на событии $X = x$. Тогда 
\begin{align*}
  f_{Z \mid X}(z \mid x) = f_{Y + x \mid X}(z \mid x) = f_{Y + x}(z) = f_Y(z - x). 
\end{align*}
Далее, по формуле полной вероятности имеем
\begin{align*}
  f_Z(z) = \int_{-\infty}^{+\infty} f_X(x) f_{Z \mid X} (z \mid x) dx = \int_{-\infty}^{+\infty} f_X(x) f_Y(z - x) dx/
\end{align*}

\textbf{Интересное следствие}

Сумма независимых нормальных распределений есть нормальное распределение. Пусть $X \sim N(\mu_X, \sigma_X^2)$ и $Y \sim N(\mu_Y, \sigma_Y^2)$ --- независимы. Посчитаем $f_Z(z)$, где $Z = X + Y$.

\begin{align*}
  f_Z(z) &= \int_{-\infty}^{+\infty} f_X(x) f_Y(z - x) dx \\
         &= \int_{-\infty}^{+\infty} \frac{1}{\sigma_X \sqrt{2\pi}} \exp\left(-\frac{(x - \mu_X)^2}{2\sigma_X^2}\right) \cdot \frac{1}{\sigma_Y \sqrt{2\pi}} \exp\left(-\frac{(z - x - \mu_Y)^2}{2\sigma_Y^2}\right)dx \\
         &= \frac{1}{\sqrt{2\pi(\sigma_X^2 + \sigma_Y^2)}} \exp\left(-\frac{(z - \mu_X - \mu_Y)^2}{2(\sigma_X^2 + \sigma_Y^2)}\right)
\end{align*}
Тут было много несложной, но громоздкой математики, которую мы опустили. Таким образом, $Z \sim N(\mu_X + \mu_Y, \sigma_X^2 + \sigma_Y^2)$. Единственное удивительное здесь -- то, что мы по-прежнему имеем нормальное распределение. Его параметры при этом очевидны из линейности матожиданий и из свойств дисперсии независимых с.в. 

По индукции легко доказать, что для любого конечного множества независимых нормальных с.в. их сумма будет также нормальной с.в.


\section{Эмитация одного распределения другим}

Практически любое современное вычислительное устройство умеет генерировать (псевдо)случайные числа, но чаще всего только равномерное распределение. Этого на самом деле достаточно, чтобы получить случайную величину, следующую любому другому распределению. В данной части мы предполагаем, что мы умеем получать с.в. $X \sim U(0 ,1)$ и хотим получить с.в. $Y$, для которой знаем функцию распределения $F_Y(y)$.

Рассмотрим простой случай. $Y$ -- дискретная с.в. с конечным набором значений. Пусть ее функция вероятности выглядит так:

\begin{center}
  \begin{tikzpicture}
    \begin{axis}[ybar, ymin=0,
      grid=major,
      ytick = {0.35, 0.2, 0.45},
      ylabel={$p_{Y}(y)$},
      xlabel={$y$}]
      \addplot
      [draw=blue, fill=blue] 
      coordinates
        {(0, 0.35) (2,0.2) (3,0.45)};
    \end{axis}
  \end{tikzpicture}
\end{center}

Как мы можем поступить. Возьмем отрезок единичной длины, разобъем его на отрезки длиной, соответствующей вероятности каждого значения с.в., сгенерим $X$ и вернем значение, соответствующее тому значению, в которое попал $X$ .

\begin{center}
  \begin{tikzpicture}
    \draw [ultra thick] (0, 0) rectangle (10, 1);

    \draw (3.5, -0.3) -- (3.5, 1);
    \draw (5.5, -0.3) -- (5.5, 1);
    \draw (0, 0) -- (0, -0.3);
    \draw (10, 0) -- (10, -0.3);

    \node at (1.75, 0.5) {$0$};
    \node at (4.5, 0.5) {$2$};
    \node at (7.75, 0.5) {$3$};

    \draw [<->] (0, -0.3) -- (3.5, -0.3) node [pos=0.5, below] {$0.35$};
    \draw [<->] (3.5, -0.3) -- (5.5, -0.3) node [pos=0.5, below] {$0.2$};
    \draw [<->] (5.5, -0.3) -- (10, -0.3) node [pos=0.5, below] {$0.45$};
  \end{tikzpicture}
\end{center}

Таким методом можно сэмитировать любую с.в. с конечным множеством значений размера $n$ за время запроса к $X \sim U(0, 1)$ плюс время $\Theta(\log(n))$, необходимое для нахождения отрезка, в который мы попали, двоичным поиском.

Интерпретация наших действий на основе функции распределения:

\begin{center}
  \begin{tikzpicture}
    \begin{axis}[
      grid=major,
      xmin = -1,
      xmax = 4,
      ytick = {0, 0.35, 0.55, 1},
      ylabel={$F_{Y}(y)$},
      xlabel={$y$}]
      \addplot
      [draw=blue, ultra thick] 
      coordinates
        {(-1, 0) (0,0) (0, 0.35) (2, 0.35) (2, 0.55) (3, 0.55) (3, 1) (4, 1)};

      \coordinate (y) at (axis cs:-1, 0.75);
      \coordinate (x) at (axis cs:3, 0);
      \coordinate (xy) at (axis cs:3, 0.75);
    \end{axis}

    \draw [red, dashed, ultra thick] (y) -- (xy) -- (x);
    \node [left] at (y) {генерим $X$ здесь};
    \node [below=1cm] at (x) {берем $Y$ отсюда};
  \end{tikzpicture}
\end{center}

Давайте используем эту интерпретацию для непрерывной $Y$ в случае с монотонной $F_Y(y)$.

\begin{center}
  \begin{tikzpicture}
    \begin{axis}[
      grid=major,
      xmin = -1,
      xmax = 4,
      ylabel={$F_{Y}(y)$},
      xlabel={$y$},
      smooth]
      \addplot
      [draw=blue, ultra thick] 
      coordinates
        {(-1, 0) (0, 0.1) (1, 0.5) (2, 0.75) (3, 0.9) (4, 1)};

      \coordinate (y) at (axis cs:-1, 0.75);
      \coordinate (x) at (axis cs:2, 0);
      \coordinate (xy) at (axis cs:2, 0.75);
    \end{axis}

    \draw [red, dashed, ultra thick] (y) -- (xy) -- (x);
    \node [left=0.5cm] at (y) {генерим $X$ здесь};
    \node [below=1.5cm] at (x) {берем $Y$ отсюда};
  \end{tikzpicture}
\end{center}

То есть после выбора $X$ мы берем $Y = F_Y^{-1}(X)$. Почему это работает:

\begin{align*}
  \Pr(Y \le y) = \Pr(F_Y^{-1}(X) \le y) = \Pr(X \le F_Y(y)) = F_Y(y)
\end{align*}

\section{Ковариация}

\end{document}