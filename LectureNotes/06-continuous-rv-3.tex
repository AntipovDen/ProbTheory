\documentclass[12pt]{article}

\usepackage{a4wide}

\usepackage[utf8]{inputenc} 
\usepackage[russian]{babel}
\usepackage{amssymb}
\usepackage{amsmath}
\usepackage{pgfplots}
\usepgfplotslibrary{statistics,patchplots}
\usetikzlibrary{decorations.pathreplacing,calc,tikzmark, patterns,arrows.meta}
\pgfmathdeclarefunction{gauss}{3}{%
  \pgfmathparse{1/(#3*sqrt(2*pi))*exp(-((#1-#2)^2)/(2*#3^2))}%
}
\pgfmathdeclarefunction{gauss2d}{6}{%
  \pgfmathparse{1/(#3*#6*2*pi)*exp(-((#1-#2)^2)/(2*#3^2) - ((#4-#5)^2)/(2*#6^2))}%
}

\usepackage{xspace}

\usepackage{mathtools}
\usepackage{cite}
\usepackage{array}
\usepackage{multirow}
\usepackage{tabularx}

\newcommand\N{\mathbb{N}}
\newcommand\R{\mathbb{R}}
\newcommand\eps{\varepsilon}
\DeclareMathOperator{\Bin}{Bin}
\DeclareMathOperator{\Geom}{Geom}
\DeclareMathOperator{\pow}{pow}
\DeclareMathOperator{\Bern}{Bern}
\DeclareMathOperator{\Exp}{Exp}
\DeclareMathOperator{\Var}{Var}


\title{Лекция 6. Непрерывные с.в., часть 3}

\begin{document}
\maketitle


Закончили на определении плотности вероятности с.в. $X$ при известном значении с.в. $Y$.

\begin{align*}
  \Pr(X \in A \mid Y = y) = \int_A f_{X \mid Y}(x \mid y) dx,
\end{align*}
И ее свойствах:
\begin{itemize}
  \item $f_{X \mid Y}(x \mid y) \ge 0$
  \item $\int_{-\infty}^{+\infty} f_{X \mid Y}(x \mid y) dx = \frac{\int_{-\infty}^{+\infty} f_{X, Y}(x \mid y) dx}{f_Y(y)} = 1$
\end{itemize}

Еще раз подчеркнем, что стоит воспринимать условную плотность как срез совместной плотности по какому-то значению с.в.. Заметьте, что при этом вероятность этого среза равна нулю (одномерное множество), поэтому только терминами условности на события тут не обойтись.

Работает правило умножения:
\begin{align*}
  f_{X, Y} (x, y) &= f_X(x) f_{Y \mid X}(y \mid x) \\
                  &= f_Y(y) f_{X \mid Y}(x \mid y) \\
\end{align*}

А значит, работают полные вероятность и матожидание. Полная вероятность (следует из правила умножения и определения маргинальной плотности вероятности):
\begin{align*}
  f_X(x) = \int_{-\infty}^{+\infty} f_Y(y) f_{X \mid Y}(x\mid y) dy
\end{align*}
Также определено условное матожидание
\begin{align*}
  E(X \mid Y = y) = \int_{-\infty}^{+\infty} x f_{X \mid Y}(x \mid y) dx
\end{align*}
и работает теорема о полном матожидании:
\begin{align*}
  E(X) &= \int_{-\infty}^{+\infty} x f_{X}(x) dx = \int_{-\infty}^{+\infty} \int_{-\infty}^{+\infty} x f_Y(y) f_{X \mid Y}(x\mid y) dy dx \\
       &= \int_{-\infty}^{+\infty}  f_Y(y) \left(\int_{-\infty}^{+\infty} x f_{X \mid Y}(x\mid y) dx\right) dy \\
       &= \int_{-\infty}^{+\infty} f_{Y}(y)E(X \mid Y = y) dy
\end{align*}

Также можно посчитать условное матожидание функции с.в.
\begin{align*}
  E(g(X) \mid Y = y) = \int_{-\infty}^{+\infty} g(x) f_{X \mid Y}(x, y) dx
\end{align*}

\section{Независимость с.в.}

Было у дискретных:

\begin{align*}
  p_{X, Y}(x, y) = p_X(x)p_Y(y)
\end{align*}

Логично определить непрерывные с.в. $X$ и $Y$ независимыми, если

\begin{align*}
  f_{X, Y}(x, y) = f_X(x) f_Y(y)
\end{align*}

Это автоматически подразумевает, что для всех $y$ верно, что $f_{X \mid Y}(x \mid y) = f_X(x)$.

Свойства независимых с.в. те же, что и у дискретных
\begin{itemize}
  \item $E(XY) = E(X) E(Y)$
  \item $\Var(X + Y) = \Var(X) + \Var(Y)$
  \item $g(X)$ и $h(Y)$ тоже будут независимы  
\end{itemize}

\textbf{Пример независимых с.в.}

Пусть есть два трамвая, которые ходят с интервалом 11 и 17 минут, независимо друг от друга. Вы приходите на остановку в случайный момент времени. Сколько ожидаемо вы будете ожидать трамвай, если вам подходит любой из двух?

Определим с.в. Пусть $T_1$ --- время, через которое придет первый трамвай, а $T_2$ --- второй. Заметим, что $T_1 \sim U(0, 11)$, а $T_2 \sim U(0, 17)$, то есть
\begin{align*}
  F_{T_1}(t) &= \begin{cases}
    0, t < 0 \\
    \frac{t}{11}, t \in [0, 11] \\
    1, t > 11,
  \end{cases} \\
  F_{T_2}(t) &= \begin{cases}
    0, t < 0 \\
    \frac{t}{17}, t \in [0, 17] \\
    1, t > 17,
  \end{cases}
\end{align*}

А время, которое придется провести на остановке есть $Y = \min\{T_1, T_2\}$, функция от двух независимых с.в.

Определим функцию распределения $Y$.
\begin{align*}
  F_Y(y) &= \Pr(Y \le y) = \Pr(T_1 \le y \cup T_2 \le y) \\
         &= 1 - \Pr(T_1 > y \cap T_2 > y) \\
         &= 1 - \Pr(T_1 > y) \Pr(T_2 > y) = 1 - (1 - F_{T_1}(y))(1 - F_{T_2}(y))
\end{align*}

Воспользуемся полезной формулой с прошлой практики для неотрицательных с.в. (\emph{которую доказали не все})
\begin{align*}
  E(Y) &= \int_{0}^{+\infty} (1 - F_Y(y)) dy = \int_{0}^{+\infty} (1 - F_{T_1}(y))(1 - F_{T_2}(y))dy \\
  &= \int_{0}^{11} \left(1 - \frac{y}{11}\right)\left(1 - \frac{y}{17}\right) dy \approx 4.31
\end{align*}



\textbf{Пример про зависимые с.в.}

Есть палка длиной $\ell$. Ломаем ее в случайном месте $X\in [0, \ell]$, остаток тоже ломаем в случайном месте $Y \in [0, X]$. Какова длина остатка $Y$? Заметим, что величины зависимы: если $X \ge \ell/2$, то есть шанс, что $Y \ge \ell/2$, а иначе нет.

Совместная функция распределения:

\begin{align*}
  f_{X, Y} (x, y) = f_X(x) f_{Y \mid X}(y \mid x) = \frac{1}{\ell x}
\end{align*}

Теперь можем найти плотность вероятности $Y$ и его матожидание. Для этого интегрируем совместную ПВ по всем возможным $X$, то есть от $y$ до $\ell$ 
\begin{align*}
  f_Y(y) = \int_y^\ell f_{X, Y} (x, y) dx = \int_0^\ell \frac{1}{\ell x} dx = \frac{1}{\ell} \ln\frac{\ell}{y}.
\end{align*}

Два способа посчитать матожидание. Первый:
\begin{align*}
  E(Y) = \int_0^\ell y f_Y(y)dy = \int_0^\ell \frac{y}{\ell} \ln\frac{\ell}{y} dy = \left( \frac{\ln(\ell)}{\ell} \cdot \frac{y^2}{2} - \frac{y^2 \ln(y)}{2\ell} + \frac{y^2}{4\ell}\right) \bigg|_0^\ell = \frac{\ell}{4}
\end{align*}

Второй:
\begin{align*}
  E(Y) = \int_0^\ell f_X(x) E(Y \mid X = x) dx = \int_0^\ell \frac{1}{\ell} \cdot \frac{x}{2} dx = \frac{x^2}{4\ell} \bigg|_0^\ell = \frac{\ell}{4}
\end{align*}

Интуитивная логика: первый раз палка ломается в среднем посередине, потом снова в среднем посередине, то есть средняя длина должна быть $\ell / 4$. Она работает не всегда, а только для $Y$, матожидание которых линейно относительно $X$. Пусть $E(Y \mid X = x) = g(x)$ тогда:
\begin{align*}
  E(Y) = \int_{-\infty}^{+\infty} f_X(x) E(Y \mid X = x) dx = \int_{-\infty}^{+\infty} f_X(x) g(x) dx = E(g(X)) \ne g(E(X)).
\end{align*}

\textbf{Независимые нормальные распределения}

Пусть есть $X \sim N(0, 1)$ и $Y \sim N(0, 1)$, и они независимы. Тогда

\begin{align*}
  f_{X, Y}(x, y) = f_X(x) f_Y(y) = \frac{1}{\sqrt{2\pi}} e^{-x^2/2} \cdot \frac{1}{\sqrt{2\pi}} e^{-x^2/2} = \frac{1}{2\pi} \exp\left(-\frac{x^2 + y^2}{2}\right).
\end{align*}

То есть их совместная плотность вероятности пропорциональна $e^{-r^2/2}$, где $r$ --- расстояние до точки (0, 0). 

\begin{center}
  \begin{tikzpicture}
    \begin{axis}
      \addplot3[
        surf,
        samples=30,
        domain=-3:3,
      ]
      {gauss2d(x, 0, 1, y, 0, 1)};

    \end{axis}
  \end{tikzpicture}
\end{center}

Эквиплотные линии:

\begin{center}
  \begin{tikzpicture}
    \begin{axis}[grid=major,
                 ylabel={$Y$},
                 xlabel={$X$},
                 height=8cm,
                 width=8cm,
                 xmin=-3.5, xmax=3.5, ymin=-3.5, ymax=3.5]
      \coordinate (Center) at (axis cs:0,0);
    \end{axis}
    \draw [ultra thick, blue] (Center) circle (2.7);
    \draw [ultra thick, yellow] (Center) circle (1.8);
    \draw [ultra thick, red] (Center) circle (0.9);
\end{tikzpicture}
\end{center}

Если у нас нестандартные нормальные распределения, то бугорок может сплющиваться или растягиваться вдоль оси, и его центр будет свдигаться 

\begin{center}
  \begin{tikzpicture}
    \begin{axis}
      \addplot3[
        surf,
        samples=30,
        domain=-3:3,
      ]
      {gauss2d(x, 1, 1.5, y, -1, 0.75)};

    \end{axis}
  \end{tikzpicture}
\end{center}

Эквиплотные линии превратятся в эллипсы.

\begin{center}
  \begin{tikzpicture}
    \begin{axis}[grid=major,
                 ylabel={$Y$},
                 xlabel={$X$},
                 height=8cm,
                 width=8cm,
                 xmin=-4, xmax=6, ymin=-5, ymax=5]
      \coordinate (Center) at (axis cs:1,-1);
    \end{axis}
    \draw [ultra thick, blue] (Center) ellipse (3 and 1.5);
    \draw [ultra thick, yellow] (Center) ellipse (2 and 1);
    \draw [ultra thick, red] (Center) ellipse (1 and 0.5);
\end{tikzpicture}
\end{center}

Заметим, что оси эллипса должны быть направлены вдоль осей, иначе с.в. будут зависимые

\section{Формула Байеса для случайных величин}

\end{document}