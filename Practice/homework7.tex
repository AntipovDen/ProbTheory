\documentclass[12pt]{article}
\usepackage[utf8]{inputenc} 
\usepackage[russian]{babel}
\usepackage{amssymb}
\usepackage{amsmath}

\newcommand\N{\mathbb{N}}
\newcommand\R{\mathbb{R}}
\newcommand\eps{\varepsilon}
\DeclareMathOperator{\Bin}{Bin}
\DeclareMathOperator{\Geom}{Geom}
\DeclareMathOperator{\Exp}{Exp}
\DeclareMathOperator{\pow}{pow}
\DeclareMathOperator{\Bern}{Bern}
\DeclareMathOperator{\Var}{Var}

\title{Седьмое домашнее задание: корреляция, симуляция и суммы}

\begin{document}
\maketitle

\setcounter{section}{3}
\section{Непрерывные с.в.}

\setcounter{subsection}{4}
\subsection{Онлайн: Полезная формула 2}

Докажите, что если неотрицательная с.в. $X$ имеет функцию распределения $F_X(x)$, то для всех $\alpha \in \R \setminus \{0\}$ верно, что
\begin{align*}
    E(X^\alpha) = |\alpha| \int_0^{+\infty} x^{\alpha - 1}\left(1 - F_X(x)\right)dx
\end{align*}
Заметьте, что $X$ может быть как дискретной, так и непрерывной с.в. (смешанные пока не рассматриваем).

\setcounter{subsection}{7}
\subsection{Оффлайн: Свойства функции распределения}

Докажите, что для любой функции распределения $F(x)$ верно, что

\begin{align*}
    &1.\ \lim_{x \to \infty} x\int_x^{+\infty} \frac{1}{z} d F(z) = 0 \\
    &2.\ \lim_{x \to 0+} x\int_x^{+\infty} \frac{1}{z} d F(z) = 0 \\
\end{align*}

\section{Непрерывные с.в.}
\setcounter{subsection}{3}
\subsection{Оффлайн: Преобразования нормального распределения 1}

Пусть $X \sim N(0, 1)$. Найдите $E[X\cos X]$ и $E[\frac{X}{1 + X^2}]$.


\section{Непрерывные с.в.}
\setcounter{subsection}{3}
\subsection{Пара независимых с.в.}

Пусть с.в. $X$ и $Y$ независимы, и $Y$ имеет симметричное распределение (то есть функции распределения $Y$ и $-Y$ совпадают). Докажите, что для любого $r \in [1, 2]$ верно,что 
\begin{align*}
    E(|X + Y|^r) \le E(|X|^r) + E(|Y|^r),
\end{align*}
если все матожидания конечны.


\setcounter{subsection}{5}
\subsection{Дискретная плюс непрерывная}

Пусть $X$ --- непрерывная с.в., а $Y$ --- просто какая-то с.в. Докажите, что $Z = X + Y$ является непрерывной с.в. (то есть имеет плотность вероятности)

\subsection{Два распада частицы}

Одна частица единичной массы распадается на две частицы массы $X$ и $1 - X$, причем $X$ -- с.в. с плотностью вероятности $f_X(x)$, такой, что
\begin{itemize}
    \item $f_X(x) = 0$, если $x \notin [0, 1]$
    \item $f_X(x) = f_X(1 - x)$ (симметрия).
\end{itemize}
Обозначим $X_1$ и $X_2$ --- массы меньшей и большей частицы соответственно. Каждая из этих частиц распадается по такому же закону, и у нас остается четыре частицы с массами
\begin{itemize}
    \item $X_{11}$ -- меньшая часть меньшей частицы.
    \item $X_{12}$ -- большая часть меньшей частицы.
    \item $X_{21}$ -- меньшая часть большей частицы.
    \item $X_{22}$ -- большая часть большей частицы.
\end{itemize}
Определите плотность вероятности $X_{11}$ и совместную плотность вероятности $X_{11}$ и $X_{22}$.

\subsection{Измерение скорости}

В России на дорогах не штрафуют при превышении скорости на не более, чем 20 км/ч. Оправдывается это тем, что у спидометра и у радара, которым измеряют скорость, есть погрешность, которая складывается. Допустим, что погрешность радара и погрешность спидометра следуют нормальному распределению $N(0, \sigma^2)$. Определите распределение показаний спидометра машины при известном показании радара.

\subsection{Оффлайн: Ожидание на остановке}

Вы приходите на остановку маршрутки, которая ходит каждые 10 минут, но вы не можете знать, как давно уехала предыдущая маршрутка. Но вы видите, что еще $N$ человек уже ждут маршрутку. Определите плотность вероятности с.в. $T$, равной времени, через которое отправится следующая маршрутка, если известно, что число людей на остановке следует распределению Пуассона с параметром $t$, равным времени, которое прошло с отправления предыдущей маршрутки. То есть через $t$ минут после отправления предыдущего рейса $\Pr(N = n) = \frac{t^n}{n!}e^{-t}$.  

\section{Новое д.з.}

\subsection{Ломаем палки}

У вас есть $n$ палочек разной длины. Каждую палочку можно ломать на сколько угодно частей любой длины. Части (в том числе и от разных палочек) затем можно склеивать друг с другом, получая новые палочки. При помощи таких операций вам надо добиться того, чтобы все его палочки (их число может измениться) стали равной длины и чтобы при этом каждая из них была склеена не более чем из двух кусочков.
\begin{itemize}
    \item Докажите, что это возможно сделать.
    \item Всегда ли можно сделать это так, чтобы палочек в итоге оказалось не больше, чем $n$?
    \item Предложите алгоритм, который с помощью описанной конструкции генерирует значения случайной величины, принимающей значения от $1$ до $n$ с данными вероятностями $p_1, \dots, p_n$ за $O(1)$ времени (при этом разрешен препроцессинг, занимающий полиномиальное время). Как источник случайных чисел, вам доступна только функция, выдающая числа, равномерно распределенные на отрезке $[0, 1]$ за $O(1)$ времени. 
\end{itemize}

\subsection{Симмуляция}

У вас есть источник случайных чисел, который умеет выдавать с.в. $X$, следующую какому-то распределению. Причем вам известна функция распределения $F_X(x)$, и вы знаете, что она непрерывна. Придумайте алгоритм получения с.в. $Y$, следующей какому-то другому распределению с функцией распределения $F_Y(y)$, за максимум один запрос к имеющемуся источнику с.в. $X$.

\subsection{Точки на окружности}

Это продолжение задачи с контрольной про точки на окружности. Есть окружность единичного радиуса, и на ней расставлены $n$ точек $A_1, \dots A_n$, причем позиция каждой точки случайна и равномерно распределена по окружности. Упорядочим точки по часовой стрелки, начиная с $A_1$, получим новую последовательность точек $A_1, A_{(2)}, \dots, A_{(n)}$. Пусть $X$ --- длина дуги $A_1 A_{(2)}$, а $Y$ --- длина дуги $A_{(2)} A_{(3)}$ (обе считаются по часовой стрелке). Найдите совместное распределение (функцию распределения или плотность вероятности) и коэффициент корреляции $X$ и $Y$. 

\subsection{Функции от двух независимых распределений}

Пусть $X$ и $Y$ --- две независимые с.в., причем обе следуют нормальному распределению $N(\mu, \sigma^2)$. Пусть с.в. $U = \alpha X + \beta Y$ и $V = \alpha X - \beta Y$ для каких-то $\alpha, \beta \in \R$. Найдите коэффициент корреляции между $U$ и $V$.

\subsection{Броски кости}

Бросаем кость d6 $n$ раз. Посчитайте коэффициент корреляции между числом выпавших единиц и числом выпавших шестерок.

\subsection{Три зависимые величины}

Приведите пример трех с.в. $X$, $Y$ и $Z$, таких, что коэффициент корреляции любых двух из них равен $-1$, или докажите, что таких с.в. не существует.

\subsection{Корреляция шаров}

Пусть в корзине $N$ шаров, из них $K$ черных и $M$ белых. Достаем из корзины $n$ шаров без возвращения, из которых $X$ оказывается черными, а $Y$ --- белыми. Найдите коэффициент корреляции $X$ и $Y$, если 
\begin{enumerate}
    \item $K + M = N$.
    \item $K +M < N$ (то есть есть шары еще какого-то другого цвета).
\end{enumerate}

\subsection{Загаданная картинка}

Есть экран, состоящий из $n$ пикселей, каждый из которых может принимать только два значения (включен или выключен). Нам загадали какую-то картинку, которая может быть изображена на экране, и мы пытаемся ее угадать. Для этого мы генерируем случайное изображение на экране, пока оно не совпадет с загаданным. Более точно процесс отгадывания выглядит так:
\begin{enumerate}
    \item Сначала все пиксели не горят.
    \item Мы зажигаем каждый пиксель с вероятностью $\frac{1}{2}$.
    \item Узнаем, не угадали ли мы загаданное изображение.
    \item Если не угадали, гасим все пиксели и продолжаем с шага 2.
\end{enumerate}
Посчитайте матожидание числа пикселей, которые мы зажжем, пока не угадаем изображение. 


\end{document}