\documentclass[12pt]{article}
\usepackage[utf8]{inputenc} 
\usepackage[russian]{babel}
\usepackage{amssymb}
\usepackage{amsmath}

\newcommand\N{\mathbb{N}}
\newcommand\R{\mathbb{R}}
\newcommand\eps{\varepsilon}
\DeclareMathOperator{\Bin}{Bin}
\DeclareMathOperator{\Geom}{Geom}
\DeclareMathOperator{\Exp}{Exp}
\DeclareMathOperator{\pow}{pow}
\DeclareMathOperator{\Bern}{Bern}
\DeclareMathOperator{\Var}{Var}

\title{Шестое домашнее задание: непрерывные с.в.}

\begin{document}
\maketitle

\section{Ломаем палки}

У Пети есть $n$ палочек разной длины. Каждую палочку можно ломать на сколько угодно частей любой длины. Части (в том числе и от разных палочек) затем можно склеивать друг с другом, получая новые палочки. Петя при помощи таких операций хочет добиться того, чтобы все его палочки (их число может измениться) стали равной длины и чтобы при этом каждая из них была склеена не более чем из двух кусочков.
\begin{itemize}
    \item Докажите, что Петя сможет это сделать.
    \item Всегда ли он сможет сделать это так, чтобы палочек в итоге оказалось не больше, чем $n$?
    \item Предложите алгоритм, который с помощью описанной конструкции генерирует значения случайной величины, принимающей значения от $1$ до $n$ с данными вероятностями $p1, \dots, pn$ за $O(1)$ времени (при этом разрешен препроцессинг, занимающий полиномиальное время). Можно считать, что вам доступна функция, выдающая числа, равномерно распределенные на отрезке $[0, 1]$ за $O(1)$ времени. 
\end{itemize}
\end{document}