\documentclass[12pt]{article}
\usepackage[utf8]{inputenc} 
\usepackage[russian]{babel}
\usepackage{amssymb}
\usepackage{amsmath}

\newcommand\N{\mathbb{N}}
\newcommand\R{\mathbb{R}}
\newcommand\eps{\varepsilon}
\DeclareMathOperator{\Bin}{Bin}
\DeclareMathOperator{\Geom}{Geom}
\DeclareMathOperator{\pow}{pow}
\DeclareMathOperator{\Bern}{Bern}
\DeclareMathOperator{\Var}{Var}

\title{Третье домашнее задание}

\begin{document}
\maketitle

\section{Дисперсия геометрического распределения}

Случайная величина $X$ следует геометрическому распределению $\Geom(p)$. Вычислите ее дисперсию.

\section{Параллельное подкидывание монет 1}

Бесконечно много раз поочередно кидаем 2 честных монеты. Определите вероятность того, что после $n$ пар бросков число орлов, выпавших на первой монете равно числу орлов, выпавших на второй. Есть ли матожидание у общего числа таких совпадений?

\section{Параллельное подкидывание монет 2}
В условиях предыдущей задачи мы перестаем подкидывать монеты, как только у нас впервые происходит совпадение числа выпавших орлов на обеих монетах. Каково матожидание совершенных бросков?

\section{Два геометрических распределения}
Пусть есть случайные величины $X_1 \sim \pow(p)$ и $X_2 \sim \Geom(p)$ (с одинаковым параметром $p$). Введем две новых случайных величины: $Y = \min\{X_1, X_2\}$ и $Z = X_1 - X_2$. Зависимы ли $Y$ и $Z$?

\section{Два показательных распределения}
Вопрос тот же, что в предыдущей задаче, только $X_1 \sim \pow(+\infty, \beta)$ и $X_2 \sim \pow(+\infty, \beta)$ (с одинаковым параметром $\beta > 1$)

\section{Сумма двух с.в.}
Пусть есть с.в. $Z = X + Y$, где $X$ и $Y$ --- не константы. Известно, что $Z$ принимает значения $0$, $1$ или $2$, каждое с вероятностью $1/3$. Могут ли $X$ и $Y$ быть независимыми?

\section{Альтернативная дисперсия}
На лекциях вскользь говорилось, что теоретически отклонение от среднего можно было бы считать и через модуль, а не квадрат. Рассмотрим $n$ подбрасываний честной монеты. Каково матожидание модуля отклонения числа выпавших орлов от $\frac{n}{2}$, если $n$ --- нечетное? А если $n$ --- четное?

\end{document}