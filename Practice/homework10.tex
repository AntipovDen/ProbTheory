\documentclass[12pt]{article}
\usepackage[utf8]{inputenc} 
\usepackage[russian]{babel}
\usepackage{amssymb}
\usepackage{amsmath}

\newcommand\N{\mathbb{N}}
\newcommand\R{\mathbb{R}}
\newcommand\eps{\varepsilon}
\DeclareMathOperator{\Bin}{Bin}
\DeclareMathOperator{\Geom}{Geom}
\DeclareMathOperator{\Exp}{Exp}
\DeclareMathOperator{\pow}{pow}
\DeclareMathOperator{\Bern}{Bern}
\DeclareMathOperator{\Var}{Var}

\title{Десятое домашнее задание: ЦПТ, равенство Вальда}

\begin{document}
\maketitle

\section{Контейнеры на корабле}

На корабль грузят контейнеры, веса которых независимы и следуют распределению $\Exp(\frac{1}{2})$. 
\begin{itemize}
    \item Какова должна быть грузоподъемность корабля, чтобы при погрузке 100 контейнеров она превышалась с вероятностью не более $0.05$?
    \item Сколько максимум контейнеров мы можем погрузить, чтобы их вес превышал 200 с вероятностью не более 0.05?
    \item Грузим контейнеры до тех пор, пока их суммарный вес не станет больше 200. Какова вероятность, что мы погрузили хотя бы 100 контейнеров?
\end{itemize}

\section{Предельные распределения}

Пусть $\{X_n\}_{n \in \N}$ --- последовательность независимых с.в., следующих стандартному нормальному распределению $N(0, 1)$. Пусть
\begin{align*}
    Y_n^2 = \sum_{i = 1}^n X_i^2, \ Z_n = \frac{X_{n + 1}}{\frac{1}{n}Y_n^2}.
\end{align*}
Найти предельные распределения\footnote{Под предельным распределением последовательности с.в. мы понимаем такую функцию $F$, к которой поточечно сходятся функции распределения с.в. из последовательности (кроме, возможно, счетного числа точек, в которых эта функция может быть неопределена).} $Y_n^2$ и $Z_n$ при $n \to \infty$.

\section{Задача с контрольной}
Теперь вы можете решить задачу с контрольной, не гугля про гамма-распределение, а воспользовавшись центральной предельной теоремой. 

У Коли есть $N$ лампочек, время жизни которых следует распределению $\Exp(1)$. Как только одна лампочка перегорает, Коля моментально меняет ее на следующую. Последняя лампочка перегорела через время $t$ после начала работы первой лампочки, но к тому моменту Коля забыл, сколько у него было лампочек в самом начале. Правда Коля знает, что обычно, когда он покупает лампочки, то он покупает случайное число лампочек, выбирая его из геометрического распределения $\Geom(p)$. Найдите примерное (насколько это позволяет ЦПТ) распределение $N$ с учетом того, что мы знаем, сколько суммарно проработали лампочки. 

\section{Счастливые билеты}

Каждый билет в автобусе имеет свой уникальный номер, состоящий из 6 цифр. Билет называется \emph{счастливым}, если сумма первых трех чисел совпадает ссуммой последних трех чисел. Посчитайте число различных счастливых билетов честно, а также приближенно с помощью ЦПТ.

\section{Зависимые опыты}

Проводим $n$ независимых экспериментов, каждый эксперимент состоит из трех испытаний Бернулли (то есть исход испытания --- 0 или 1). Результаты первых двух испытаний независимы друг от друга и равновероятны. Исход третьего испытания такой, что число единиц в этой группе из трех испытаний --- четное (если первые два испытания имеют результат 00 ии 11, то третье --- 0, иначе --- 1). Докажите, что при $n$, стремящемся к бесконечности, число успешных испытаний такое же, как если бы мы проводили $3n$ независимых испытаний по схеме Бернулли. 

\section{Само-нормализация}

Пусть $\{X_n\}_{n \in \N}$ --- последовательность независимых с.в. с нулевым матожиданием и одинаковой ненулевой конечной дисперсией $\sigma^2$. Докажите, что с.в.
\begin{align*}
    Y_n = \frac{\sum_{i = 1}^n X_i}{\left(\sum_{i = 1}^n X_i^2\right)^{1/2}}
\end{align*}
сходится по распределению к стандартному нормальному распределению.


\end{document}