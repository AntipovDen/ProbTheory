\documentclass[12pt]{article}
\usepackage[utf8]{inputenc} 
\usepackage[russian]{babel}
\usepackage{amssymb}
\usepackage{amsmath}
\usepackage{a4wide}

\newcommand\N{\mathbb{N}}
\newcommand\R{\mathbb{R}}
\newcommand\eps{\varepsilon}
\DeclareMathOperator{\Bin}{Bin}
\DeclareMathOperator{\Geom}{Geom}
\DeclareMathOperator{\Exp}{Exp}
\DeclareMathOperator{\pow}{pow}
\DeclareMathOperator{\Bern}{Bern}
\DeclareMathOperator{\Var}{Var}
\DeclareMathOperator{\Cov}{Cov}

\title{Задачи на первую КР}

\begin{document}
\maketitle

\section{Корреляция}

\subsection{Четверть круга}

$(X, Y)$ --- координаты случайно и равномерно выбранной точки из четверти круга единичного радиуса с центром в $(0, 0)$, лежащей в первом квадранте. То есть это множество точек $(x, y)$, где $x^2 + y^2 \le 1$, $x \ge 0$ и $y \ge 0$. Найдите коэффициент корреляции между $X$ и $Y$. Для взятия интегралов  допустимо пользоваться вольфрамом.

\subsection{Линейное некоррелируемое преобразование}

Пусть $X$ и $Y$ --- две с.в. с нулевыми матожиданиями, единичными дисперсиями и кофэффициентом корреляции $\rho$. Для каких $a$ и $b$ случайные величины $X - aY$ и $Y - bX$ не коррелируют?

\subsection{Корреляция сумм}

Пусть $X_1, \dots, X_{m + n}$ --- независимые, одинаково распределенные с.в. с конечными дисперсиями. Пусть $m < n$. Чему равен коэффициент корреляции между $X_1 + X_2 + \dots + X_n$ и $X_{m + 1} + \dots + X_{m + n}$?

\section{Полезные неравенства}

\subsection{Определение честной монеты}

У вас есть две монеты: честная и нечестная с вероятностью выпадения орла равной $q = \frac{1}{4}$. Но вы не знаете, какая из монет честная. Вы бросаете каждую монету по $n$ раз. Каким должен быть $n$, чтобы вы могли быть уверенными в том, какая из монет --- честная с вероятностью $0.99$? 

\subsection{Случайный вектор}

Пусть $X_1, \dots, X_n$ --- с.в. Бернулли с вероятностью $\frac{1}{2}$. Пусть $X = (\sum_{i = 1}^n X_i)^2$. Определите матожидание $X$ и наиболее точно оцените вероятность того, что он отклоняется от него на более, чем $10\%$.

\subsection{Возвращение с вечеринки}

Вы идете домой после вечеринки. До дома вам надо пройти $n$ шагов. Но из-за усталости, каждый шаг, который вы делаете, оказывается в направлении дома с вероятностью $\frac{3}{4}$ и с вероятностью $\frac{1}{4}$ --- от дома. Сколько шагов вам придется сделать, чтобы вероятность того, что вы дошли до дома была хотя бы $0.9$? 

\section{ЗБЧ, ЦПТ}

\subsection{Отрицательная ковариация}

Пусть $\{X_n\}_{n \in \N}$ --- последовательность с.в. с одинаковыми распределениями и конечными матожиданиями и дисперсиями, но с зависимостью. Причем ковариация любой пары с.в. $\Cov(X_i, X_j) \le 0$. Покажите, что для них выполняется слабый закон больших чисел.

\subsection{Зависимые тройки}
Пусть $\{X_n\}_{n \in \N}$ --- последовательность с.в. с одинаковыми распределениями, конечными матожиданиями и дисперсией, причем
\begin{itemize}
    \item Для любого $i$, кратного трем, $X_{i - 2} = - X_{i - 1} = X_{i}$. 
    \item Для любого $i$ и $j$, таких что $|i - j| \ge 3$ с.в. $X_i$ и $X_j$ независимы.
\end{itemize}
Докажите, что для данной последовательности выполняется слабый закон больших чисел

\subsection{Предельное распределение}

Пусть $\{X_n\}_{n \in \N}$ --- последовательность одинаково распределенных независимых с.в. с нулевыми матожиданиями и единичными дисперсиями. Найти предельное распределение с.в.
\begin{align*}
    \sqrt{n}\frac{X_1 + \dots + X_n}{X_1^2 + \dots + X_n^2}.
\end{align*}

\section{Мартингалы}

\subsection{Бактерии}

Пусть у нас есть $n$ бактерий, половина которых типа $A$, а остальные --- типа $B$. Мы проводим серию экспериментов. Во время каждого эксперимента мы $n$ раз берем случайную бактерию и создаем ее копию. Затем мы добавляем созданные копии к общей популяции бактерий (то есть после каждого эксперимента число бактерий увеличивается на $n$). Пусть $X_t$ --- число бактерий типа $A$, которые мы создали во время эксперимента $t$. Является ли $X_t$ мартингалом?

\subsection{Пример мартингала}
Пусть $\{X_n\}_{n \in \N}$ --- последовательность независимых с.в. с нулевыми матожиданиями и конечными дисперсиями. Пусть $S_n = X_1 + \dots + X_n$ и $B_n = \Var(X_1) + \cdot + \Var(X_n)$. Покажите, что последовательность $S_n^2 - B_n$ является мартингалом.

\subsection{Пример субмартингала}

Пусть $\{X_n\}_{n \in \N}$ --- последовательность независимых неотрицательных с.в. с конечными матожиданиями. Пусть $S_n = X_1 + \dots + X_n$. Докажите, что $S_n$ --- субмартингал.


\end{document}