\documentclass[12pt]{article}
\usepackage[utf8]{inputenc} 
\usepackage[russian]{babel}
\usepackage{amssymb}
\usepackage{amsmath}

\newcommand\N{\mathbb{N}}
\newcommand\R{\mathbb{R}}
\newcommand\eps{\varepsilon}
\DeclareMathOperator{\Bin}{Bin}
\DeclareMathOperator{\Geom}{Geom}
\DeclareMathOperator{\Exp}{Exp}
\DeclareMathOperator{\pow}{pow}
\DeclareMathOperator{\Bern}{Bern}
\DeclareMathOperator{\Var}{Var}

\title{Восьмое домашнее задание: полезные неравенства}

\begin{document}
\maketitle

\section{Неравенство Гаусса (или неравенство Чебышева для унимодальных распределений)}

Пусть $X$ --- унимодальная непрерывная с.в., то есть она имеет плотность вероятности с единственным максимумом в точке $x_0$ (других экстремумов вообще нет). Докажите, что если $\tau = E[(X - x_0)^2]$ (что эквивалентно $\tau = \Var(X) + (E[X] - x_0)^2$), то для любого $\eps > 0$ верно, что
\begin{align*}
    \Pr(|X - x_0| \ge \eps\tau) \le \frac{4}{9\eps^2}.
\end{align*}

\emph{Подсказка:} докажите сначала, что если функция $g(x)$ не возрастает, то для любого $\eps > 0$ верно, что
\begin{align*}
    \eps^2 \int_\eps^{+\infty} g(x) dx \le \frac{4}{9} \int_{0}^{+\infty} x^2 g(x) dx.
\end{align*} 

\section{Бактерии}

В банке есть $n$ бактерий двух типов: $A$ и $B$, причем бактерий обоих типов равное число (то есть по $\frac{n}{2}$). Мы проводим с ними серию экспериментов. Каждый эксперимент заключается в следующем. Мы $n$ раз выбираем случайную бактерию из банки (равновероятно), создаем ее копию и сажаем эту копию в новую банку. В каждом следующем эксперименте мы выбираем бактерии из банки, в которую мы сажали копии в предыдущем эксперименте.

Пусть $X_t$ --- число бактерий типа $A$ в самой последней банке после $t$ таких экспериментов. Докажите, что если $T \le \sqrt{\frac{n}{\ln(n)}}$, то вероятность того, что для всех $t \in [1..T]$ выполнялось $X_t \ge \frac{n}{4}$, есть $o(1)$, где нотация о-маленького берется при $n \to \infty$.

\section{Произведение с.в.}

Пусть $X_1, \dots X_n$ --- независимые, строго положительные, одинаково распределенные с.в., причем $\Var[\ln(X_1)] = \sigma^2$. Докажите, что

\begin{align*}
    \Pr\left(\prod_{i = 1}^n X_i \le (E[X_1])^n e^{\eps n}\right) \ge 1 - \frac{\sigma^2}{\eps^2 n}
\end{align*}

\section{Условные вероятности}

Бросаем честную кость $d6$ $n$ раз. Оцените вероятность того, что единиц выпало больше, чем $\frac{n}{6}$, если известно, что выпало не меньше, чем $\frac n3$ шестерок. Докажите, что эта вероятность не больше, чем $e^{-n/300}$.

\section{Стандартная битовая мутация}

Пусть у нас есть битовая строка длины $n$, и мы применяем к ней стандартную битовую мутацию с силой $\alpha$: мы меняем значение каждого бита строки с вероятностью $\frac{\alpha}{n}$, решение для каждого бита принимается независимо от решения для остальных бит. Таким образом, ожидаемое число бит, которые мы переключили, есть $\alpha$.

Допустим, мы хотим переключить ровно $k$ бит. Покажите, что если мы выберем неверный $\alpha$, то вероятность этого будет очень мала. А именно, покажите, что если $p_k(\alpha)$ есть вероятность переключить ровно $k$ бит при силе мутации $\alpha$, то для любого $\eps \in (0, 1]$
\begin{itemize}
    \item $\alpha = (1 - \eps)k \Rightarrow p_k(\alpha) \le \exp(-\frac{\eps^2 k}{2})$
    \item $\alpha = (1 + \eps)k \Rightarrow p_k(\alpha) \le \exp(-\frac{\eps^2 k}{2(1 + \eps)})$
\end{itemize}

\emph{Указание:} используйте границы Чернова, хоть мы их и полностью пройдем только на следующей паре.

\end{document}