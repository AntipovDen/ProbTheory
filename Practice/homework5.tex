\documentclass[12pt]{article}
\usepackage[utf8]{inputenc} 
\usepackage[russian]{babel}
\usepackage{amssymb}
\usepackage{amsmath}

\newcommand\N{\mathbb{N}}
\newcommand\R{\mathbb{R}}
\newcommand\eps{\varepsilon}
\DeclareMathOperator{\Bin}{Bin}
\DeclareMathOperator{\Geom}{Geom}
\DeclareMathOperator{\pow}{pow}
\DeclareMathOperator{\Bern}{Bern}
\DeclareMathOperator{\Var}{Var}

\title{Пятое домашнее задание: непрерывные с.в.}

\begin{document}
\maketitle

\section{Игла Бюффона}

На бесконечной плоскости нарисовано бесконечно много параллельных прямых, расстояние между двумя соседними равно $d$. Мы бросаем на эту плоскость иголку длины $\ell$. Вычислите вероятность того, что иголка не пересечет ни одной прямой. Посчитайте матожидание числа пересеченных прямых.

\section{Плотность вероятности функции от с.в.}

Пусть $X$ имеет непрерывную плотность вероятности $f_X(x)$, причем существуют $\alpha$ и $\beta > \alpha$, что $\Pr(\alpha \le X \le \beta) = 1$. Пусть $g(x)$ --- монотонно возрастающая и дифференцируемая функция на $[\alpha, \beta]$. Докажите, что у с.в. $Y = g(X)$ плотность вероятности
\begin{align*}
    f_Y(y) = \begin{cases}
        \frac{f(g^{-1}(y))}{g'(g^{-1}(y))}, &\text{ если } y \in [g(\alpha), g(\beta)] \\
        0, &\text{ иначе.}
    \end{cases}
\end{align*}

Продемонстрируйте правильность на $X \sim U(\alpha, \beta)$ и $g(x) = ax + b$. 

\section{Квадрат с.в.}

Пусть у непрерывной с.в. $X$ плотность вероятности $f_X(x)$. Найдите плотность вероятности с.в. $X^2$. Вычислите ее, если известно, что $X \sim N(0, 1)$ (стандартное нормальное распределение).

\section{Преобразования нормального распределения 1}

Пусть $X \sim N(0, 1)$. Найдите $E[X\cos X]$ и $E[\frac{X}{1 + X^2}]$.

\section{Преобразования нормального распределения 2}

Пусть $X \sim N(0, 1)$. Найдите $E[\cos X]$ и $\Var[\cos X]$.

\section{Про трамвай}

Для трамвайного маршрута длины $d$ известна функция $F(x, y)$, которая равна вероятности того, что пассажир, проехавший на этом трамвае, зашел на расстоянии не более $x$ от начала маршрута и сошел на расстоянии не более $y$ от начала маршрута. Определите для каждой точки маршрута $z \in [0, d]$ вероятность того, что пассажир проехал точку $z$.

\section{Случайный вектор}

Две случайных величины имеют совместную плотность распределения 
\begin{align*}
    f_{X, Y}(x, y) = \begin{cases}
        xe^{-x(y + 1)}, \text{ если } x, y \ge 0 \\
        0, \text{ иначе.}
    \end{cases}
\end{align*}

Найти их маргинальные плотности распределения и условные плотности распределения $f_{X \mid Y}(x \mid y)$ и $f_{Y \mid X}(y \mid x)$.

\section{Распад атомов}

У вас есть $m_0$ радиоактивных атомов. Определите матожидание массы атомов $m(t)$, которая у вас останется через время $t$, если известно, что для любого целого атома вероятность $p$ распасться в ближайшую единицу времени равна $e^{-1}$.  

\end{document}