\documentclass[12pt]{article}
\usepackage[utf8]{inputenc} 
\usepackage[russian]{babel}
\usepackage{amssymb}
\usepackage{amsmath}


\title{Второе домашнее задание}

\begin{document}
\maketitle

\section{Шары и одна корзина}

В корзину забрасывается $n$ шаров, цвет каждого забрасываемого шара может быть либо черным, либо белым, причем равновероятно. Потом проводится эксперимент: $k$ раз из корзины извлекается шар, записывается его цвет, и шар возвращается обратно. Какова вероятность того, что в урне только белые шары, если все шары, которые мы достали, --- белые? Какова эта вероятность, если $k \le n$ и шары не возвращаются в корзину во время эксперимента?

\section{Шары и много корзин}

Бросаем $k$ шаров в одну из $n$ корзин. Причем для каждого шара выбираем корзину равновероятно и независимо от выбора корзин для других шаров. Посчитайте матожидание и дисперсию числа непустых корзин.

\section{Шары и маленькие корзины}

Из ВУЗа выпускаются $k$ студентов, и они хотят это отметить. В городе есть $n$ различных заведений, где это можно сделать (театры, музеи и т.д.), но вместимость каждого заведения --- не более $r$ человек. Каждый студент случайным образом равновероятно выбирает одно из заведений и направляется в него. Если оно оказывается заполненным --- студента не впускают, и он идет домой отмечать с друзьями онлайн. Определите матожидание числа студентов, отмечавших выпускной дома. Считайте, что другие жители города выбираться в этот день в те же заведения не рискуют. 

\section{Случайный вектор}

Пусть есть случайный вектор $X = (X_1, \dots, X_n)$, где все $X_i$ --- какие-то дискретные случайные величины. Если известно, что существует матожидание каждой из этих величин, то существует ли матожидание длины этого вектора $Z = \sqrt{X_1^2 + \dots + X_n^2}$? Верно ли обратное: если существует матожидание $Z$, то существует матожидание каждого $X_i$?

\section{Календарь}

Вычислите матожидание числа дней в этом году, если известно число дней в прошлом году. Учитывайте все особенности Григорианского календаря.

\section{Анализы крови}

$N$ людей сдают кровь на качественный анализ (результат анализа --- либо ``положительный'', либо ``отрицательный''). Так как каждый тест довольно дорогостоящий, лаборатория прибегает к следующей оптимизации. Люди объединяются в группы по $k$ человек, и образцы крови всех людей одной группы смешиваются. Смешанная кровь тестируется, и если результат отрицательный, то у всех людей из этой группы считается отрицательный результат. В этом случае проводится ровно один тест. Однако если результат на смешанном образце положительный, то образцы крови каждого человека из этой группы тестируются отдельно. В этом случае проводится $k + 1$ тест. Посчитайте матожидание числа проведенных тестов, если известно, что у каждого из $N$ сдающих анализ вероятность положительного результата $p$. Как лучше всего выбрать $k$ при известном $p$? Не обязательно давать точный ответ, достаточно описать алгоритм подбора $k$.

\section{Анализы крови 2}

В предыдущей задаче имеет ли смысл разбивать группу, у которой был смешанный тест, на более мелкие группы? Имеет ли смысл делать это более одного раза? Предложите оптимальную стратегию, минимизирующую ожидаемое число тестов.

\end{document}