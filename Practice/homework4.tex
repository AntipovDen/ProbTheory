\documentclass[12pt]{article}
\usepackage[utf8]{inputenc} 
\usepackage[russian]{babel}
\usepackage{amssymb}
\usepackage{amsmath}

\newcommand\N{\mathbb{N}}
\newcommand\R{\mathbb{R}}
\newcommand\eps{\varepsilon}
\DeclareMathOperator{\Bin}{Bin}
\DeclareMathOperator{\Geom}{Geom}
\DeclareMathOperator{\pow}{pow}
\DeclareMathOperator{\Bern}{Bern}
\DeclareMathOperator{\Var}{Var}

\title{Четвертое домашнее задание: непрерывные с.в.}

\begin{document}
\maketitle

\section{Про функцию распределения}

Покажите, что для любой с.в. функция распределения имеет не более, чем счетное число точек разрыва.

\section{Функция распределения как случайная величина}

Пусть есть какая-то с.в. $X$, у которой есть \emph{непрерывная} функция распределения $F_X(x) = \Pr(X \le x)$. Пусть есть другая с.в. $Y = F_X(X)$. Найдите распределение $Y$.

\section{Из непрерывной в дискретную}

Приведите пример непрерывной с.в. $X$ с плотностью вероятности $f_X(x)$ и \emph{непрерывной} функции $g(x)$, так, чтобы с.в. $Y = g(X)$ была дискретной.

\section{Полезная формула 1}

Докажите, что если с.в. $X$ имеет функцию распределения $F_X(x)$, то верно, что
\begin{align*}
    E(X) = \int_0^{+\infty} \left(1 - F_X(x)\right)dx - \int_{-\infty}^0 F_X(x)dx
\end{align*}
Заметьте, что $X$ может быть как дискретной, так и непрерывной с.в. (смешанные пока не рассматриваем).


\section{Полезная формула 2}

Докажите, что если неотрицательная с.в. $X$ имеет функцию распределения $F_X(x)$, то для всех $\alpha \in \R \setminus \{0\}$ верно, что
\begin{align*}
    E(X^\alpha) = |\alpha| \int_0^{+\infty} x^{\alpha - 1}\left(1 - F_X(x)\right)dx
\end{align*}
Заметьте, что $X$ может быть как дискретной, так и непрерывной с.в. (смешанные пока не рассматриваем).

\section{Логнормальная с.в.}

Пусть $X \sim N(0, 1)$. Найти плотность вероятности, матожидание и дисперсию с.в. $Y = e^X$.

\section{Еще одно распределение}

Пусть 
\begin{align*}
    f_X(x) = \begin{cases}
        Ax^\alpha e^{-x/\beta}, &\text{ если } x \ge 0, \\
        0, &\text{ иначе.}
    \end{cases}
\end{align*}

Найти коэффициент нормализации $A$, матожидание и дисперсию с.в. $X$ c этой плотностью вероятности.

\section{Свойства функции распределения}

Докажите, что для любой функции распределения $F(x)$ верно, что

\begin{align*}
    &1.\ \lim_{x \to \infty} x\int_x^{+\infty} \frac{1}{z} d F(z) = 0 \\
    &2.\ \lim_{x \to 0+} x\int_x^{+\infty} \frac{1}{z} d F(z) = 0 \\
\end{align*}


\end{document}