\documentclass[12pt]{article}
\usepackage[utf8]{inputenc} 
\usepackage[russian]{babel}
\usepackage{amssymb}
\usepackage{amsmath}

\newcommand\N{\mathbb{N}}
\newcommand\R{\mathbb{R}}
\newcommand\eps{\varepsilon}
\DeclareMathOperator{\Bin}{Bin}
\DeclareMathOperator{\Geom}{Geom}
\DeclareMathOperator{\Exp}{Exp}
\DeclareMathOperator{\pow}{pow}
\DeclareMathOperator{\Bern}{Bern}
\DeclareMathOperator{\Var}{Var}

\title{Одиннадцатое домашнее задание: Мартингалы}

\begin{document}
\maketitle

\section{Примеры мартингалов}

Пусть есть последовательность независимых с.в. $\{X_n\}_{n \in \N}$, которые следуют одинаковому распределению, причем \[\Pr[X_n = +1] = 1 - \Pr[X_n = -1] = p \ne 0.5. \]
Пусть $S_n = \sum_{i = 1}^n X_i$. Докажите, что относительно стандартной фильтрации мартингалами являются:
\begin{itemize}
    \item $S_n - 2n(2p - 1)$
    \item $(\frac{1 - p}{p})^{S_n}$
\end{itemize}

\section{Бактерии (опять)}

Условия те же, что в задаче 8.2. В банке есть $n$ бактерий двух типов: $A$ и $B$, причем бактерий обоих типов равное число (то есть по $\frac{n}{2}$). Мы проводим с ними серию экспериментов. Каждый эксперимент заключается в следующем. Мы $n$ раз выбираем случайную бактерию из банки (равновероятно), создаем ее копию и сажаем эту копию в новую банку. В каждом следующем эксперименте мы выбираем бактерии из банки, в которую мы сажали копии в предыдущем эксперименте.

Пусть $X_t$ --- число бактерий типа $A$ в самой последней банке после $t$ таких экспериментов. Докажите, что $X_t$ --- мартингал. 

\section{Бактерии: распространение признака}

В банке есть $\mu$ бактерий. Каждая бактерия либо обладает полезным свойством $A$, либо не обладает. Мы проводим серию экспериментов. В каждом эксперименте серии мы $\lambda \ge e\mu$ раз берем случайную бактерию из банки и создаем ее копию. С вероятностью $(1 - \frac{1}{e})$ копия также претерпевает мутацию, в результате которой она может потерять свойство $A$, если она имела, или приобрести свойство $A$, если она его не имела. После того, как мы создали $\lambda$ новых бактерий, мы выбираем $\mu$ из них, и следующий эксперимент проводим над отобранными бактериями. При отборе мы в первую очередь выбираем бактерии со свойством $A$, а потом уже --- без свойства $A$.

Пусть $X_t$ --- число бактерий со свойством $A$ в самой последней банке после $t$ таких экспериментов. Является ли $A$ субмартингалом? 

\section{Бактерии: вымирание признака}

В банке есть $\mu$ бактерий. Каждая бактерия либо обладает полезным свойством $A$, либо не обладает. Мы проводим серию экспериментов. В каждом эксперименте серии мы $\lambda \le e\mu$ (в отличие от предыдущей задачи, где было больше) раз берем случайную бактерию из банки и создаем ее копию. С вероятностью $(1 - \frac{1}{e})$ копия также претерпевает мутацию, в результате которой она точно (а не возможно, как было в предыдущей задаче) теряет свойство $A$, если она его имела, или точно приобретает свойство $A$, если она его не имела. После того, как мы создали $\lambda$ новых бактерий, мы выбираем $\mu$ из них, и следующий эксперимент проводим над отобранными бактериями. При отборе мы в первую очередь выбираем бактерии со свойством $A$, а потом уже --- без свойства $A$.

Пусть $X_t$ --- число бактерий со свойством $A$ в самой последней банке после $t$ таких экспериментов. Является ли $A$ супермартингалом? 

\section{Мартингал по средним}

Пусть $\{X_n\}_{n \in \N}$ --- последовательность с.в. с конечными матожиданиями, причем
\begin{align*}
    E[X_{n + 1} \mid X_1, \dots, X_n] = \frac{X_1 + \dots + X_n}{n}.
\end{align*}

Докажите, что $Y_n = \frac{X_1 + \dots + X_n}{n}$ является мартингалом. Можно доказывать либо через фильтрации, либо просто через матожидание условное на всех предыдущих с.в.

\end{document}