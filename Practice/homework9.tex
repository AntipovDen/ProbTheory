\documentclass[12pt]{article}
\usepackage[utf8]{inputenc} 
\usepackage[russian]{babel}
\usepackage{amssymb}
\usepackage{amsmath}

\newcommand\N{\mathbb{N}}
\newcommand\R{\mathbb{R}}
\newcommand\eps{\varepsilon}
\DeclareMathOperator{\Bin}{Bin}
\DeclareMathOperator{\Geom}{Geom}
\DeclareMathOperator{\Exp}{Exp}
\DeclareMathOperator{\pow}{pow}
\DeclareMathOperator{\Bern}{Bern}
\DeclareMathOperator{\Var}{Var}

\title{Девятое домашнее задание: сходимость, ЗБЧ}

\begin{document}
\maketitle

\section{Неинтуитивность пространства $\{0, 1\}^n$}

Рассмотрим просторанство битовых строк длины $n$. Обозначим за $N_k$ множество всех битовых строк из этого пространства, в которых ровно $k$ нулей. Очевидно, для любого $k$ все элементы $N_k$ равноудалены от строки $1^n$ (если считаем расстояние Хэмминга). Покажите, что если $k = o(\sqrt{n})$, то для любого $d < k$ любая строчка с $d$ нулями равноудалена почти от всех строчек из $N_k$ на расстояние $k + d$. Более формально, пусть $x^* = 1^{n -d}0^d$. Требуется показать, что если $k = o(\sqrt{n})$ и $d < k$, то
\begin{align*}
    |\{x \in N_k: H(x, x^*) = k + d\}| = N_k(1 - o(1)),
\end{align*}
где все о-маленькие используются при $n \to +\infty$.

\emph{Подсказка:} границы Чернова могут помочь, так как работают и для гипер-геометрического распреления. Однако, в данном случае они могут только сказать, что искомое множество имеет мощность не меньше, чем $N_k(1 - e^{-1/2})$.

\section{Упражнение с лекции}

Докажите, что сходимость почти наверное влечет за собой сходимость по вероятности, которая в свою очередь влечет сходимость по распределению.

\section{ЗБЧ для зависимых с.в.}

Пусть дана последовательность независимых с.в. $X_n$, следующих стандартному нормальному распределению $N(0, 1)$. И есть последовательность $Y_n = \cos\frac{X_n}{X_{n + 1}}$. Работает ли слабый закон больших числел для последовательности $Y_n$? То есть 
\begin{itemize}
    \item Верно ли, что все $Y_n$ имеют одинаковое маргинальное распределение с конечным матожиданием $\mu$? 
    \item Если обозначить $M_n = \frac{1}{n}\sum_{i = 1}^{n} Y_i$ и $\mu = E[Y_1]$, то верно ли, что $M_n \to \mu$ по вероятности?
\end{itemize}

\section{Сумма попарных произведений}
Пусть дана последовательность независимых с.в. $X_n$, следующих одинаковому распределению, причем для всех $i$ верно, что $E[X_i] = \mu$ и $\Var(X_i) = \sigma^2 < \infty$.
Докажите, что $Y_n = \binom{n}{2}^{-1} \sum_{1 \le i < j \le n} X_i X_j$ сходится по вероятности к $\mu^2$:
\begin{align*}
    \forall \eps \ \lim_{n \to \infty} \Pr(|Y_n - \mu^2| > \eps) = 0.
\end{align*}

\section{Сходимость по вероятности}

Пусть $X$ -- непрерывная с.в. с функцией распределения $F_X(x)$. Пусть $Y_n$ --- последовательность независимых (в том числе от $X$) с.в., которая сходится к 1 по вероятности. Докажите, что
\begin{align*}
    \lim_{n \to \infty}\Pr(X + Y_n < x) = F(x - 1).
\end{align*}

\section{Закон больших чисел писан не для всех}

Пусть дана последовательность независимых с.в. $X_n$, такая, что $X_n = 2^n$ с веротяностью $1/2$ и $X_n = - 2^n$ с вероятностью $1/2$. Покажите, что для данной последовательности не работает даже слабый ЗБЧ (то есть среднее первых $n$ ее членов не сходится по вероятности). Покажите также, что для последовательности с.в. $Y_n$, которая с веротяностью $1 - 2^{-(n - 1)}$ равна нулю, а иначе равновероятно равна $\pm 2^n$, слабый ЗБЧ выполняется.

\section{ЗБЧ для средних}

Пусть есть последовательность независимых, одинаково распределенных $X_n$ с ненулевой конечной дисперсией. Пусть $M_n = \frac{1}{n}\sum_{i = 1}^n X_i$. Выполняются ли слабый и сильный ЗБЧ для $M_n$? 

\end{document}