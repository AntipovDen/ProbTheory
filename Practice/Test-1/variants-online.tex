\documentclass[12pt]{article}
\usepackage[utf8]{inputenc} 
\usepackage[russian]{babel}
\usepackage{amssymb}
\usepackage{amsmath}
\usepackage{a4wide}
\usepackage{hyperref}

\newcommand\N{\mathbb{N}}
\newcommand\R{\mathbb{R}}
\newcommand\eps{\varepsilon}
\DeclareMathOperator{\Bin}{Bin}
\DeclareMathOperator{\Geom}{Geom}
\DeclareMathOperator{\Exp}{Exp}
\DeclareMathOperator{\pow}{pow}
\DeclareMathOperator{\Bern}{Bern}
\DeclareMathOperator{\Var}{Var}

\begin{document}
\pagenumbering{gobble} 

\section{Алехин Артем Александрович}

\subsection{Независимость}
Есть три события $A$, $B$ и $C$. Известно, что $A$ и $B$ --- независимые, и $A$ и $C$ --- независимые. Также известно, что \[\Pr(A \cap B \cap C) = \Pr(A)\Pr(B)\Pr(C).\] Являются ли события $B$ и $C$ независимыми? Либо докажите независимость, либо приведите пример, в котором эти события будут зависимы.

\subsection{Прятки}
$K$ детей играют в прятки (плюс один вода). Пока вода считает, каждый ребенок бежит в одно из $N$ мест, где можно спрятаться, выбирая каждое убежище равновероятно. Если оказывается, что более одного человека претендует на одно укрытие, то оно достается только одному, а остальные не успевают перепрятаться и проигрывают сразу же, как вода оборачивается. Вычислите матожидание числа детей, которые проигрывают таким обидным способом. Так как формула матожидания может получиться сложной, докажите, что эта формула неотрицательна.

\subsection{Экспедиция}
В экспедиции, которая должна продлиться $n$ дней, каждый день берется часть от продуктов и съедается. Так как начпрод в экспедиции не очень точен, эта часть есть случайная величина, равная в $k$-й день $X_k$. То есть если в начале $k$-ого дня в запасах было $N$ кг еды, то к началу следующего дня их будет $(1 - X_k)N$ кг. Матожидание $X_k$ в $k$-й день экспедиции есть $\frac{1}{n - k + 1}$, причем все $X_k$ независимы друг от друга. Определите матожидание доли еды, оставшейся к началу последнего дня экспедиции.

\subsection{Сколько было лампочек}
У Коли есть $N$ лампочек, время жизни которых следует распределению $\Exp(1)$. Как только одна лампочка перегорает, Коля моментально меняет ее на следующую. Последняя лампочка перегорела через время $t$ после начала работы первой лампочки, но к тому моменту Коля забыл, сколько у него было лампочек в самом начале. Правда Коля знает, что обычно, когда он покупает лампочки, то он покупает случайное число лампочек, выбирая его из геометрического распределения $\Geom(p)$. Найдите распределение $N$ с учетом того, что мы знаем, сколько суммарно проработали лампочки. 

\emph{Подсказка: почитайте про то, что такое гамма-распределение и как оно связано с экспоненциальным.}

\newpage
\section{Андриянов Кирилл Романович}

\subsection{Независимость}
Есть три события $A$, $B$ и $C$. Известно, что $A$ и $B$ --- независимые, и $A$ и $C$ --- независимые. Также известно, что \[\Pr(A \cap B \cap C) = \Pr(A)\Pr(B)\Pr(C).\] Являются ли события $B$ и $C$ независимыми? Либо докажите независимость, либо приведите пример, в котором эти события будут зависимы.

\subsection{Пока не будет слишком много брака}
Вероятность того, что станок произведет бракованную деталь равна $p$. Станку только что провели поверку, а следующую проведут после того, как он изготовит ровно $k$ бракованных деталей. Посчитайте матожидание числа \emph{небракованных} деталей, которые выпустит станок до следующей поверки.

\subsection{Ремонт}
Петя и Вася решили сделать ремонт (каждый в своем доме). Каждому из них надо нарезать 10 полосок обоев. Петя поступает следующим образом: он отрезает первую полоску обоев, а затем отмеряет длину новой полоски, прикладывая ее к предыдущей отрезанной. Вася же отмеряет каждую полоску, прикладывая к самой первой отрезанной полоске. Каждый раз разница длин новой полоски и той, по которой ее отмеряли, является случайной величиной, которая достаточно хорошо приближается нормальным распределением $N(0, 1)$ (если считать в сантиметрах). Определите:
\begin{enumerate}
    \item Дисперсию длины 10-ой отрезанной полоски у Пети и у Васи.
    \item Матожидание модуля разности длин 10-ой полоски Пети и 10-ой полоски Васи. 
\end{enumerate}

\subsection{Точки на окружности}
Точки $A_1, \dots, A_n$ расположены на окружности единичного радиуса, причем положение каждой точки не зависит от положения других и равномерно распределено по всей окружности. Упорядочим точки по часовой стрелке, начиная с $A_1$, получим последовательность $A_1, A_{(2)}, \dots, A_{(n)}$. Найти распределение длины дуги окружности между точками $A_1$ и $A_{(2)}$ (длина дуги считается по часовой стрелке).

\newpage
\section{Боже Илона Яновна}

\subsection{Игра}
Есть корзина, в которой лежит $N$ шаров: $K$ черных и $M = N - K$ белых. Два игрока поочередно достают из корзины по шару. Выигрывает тот, кто первым достанет белый шар. Какова вероятность, что выиграет первый игрок? В итоговой формуле можно оставить только одну сумму и биномиальные коэффициенты. Посчитайте также эту вероятность точно для ситуации, когда белый шар единственный (то есть $M = 1$).

\subsection{Прятки}
$K$ детей играют в прятки (плюс один вода). Пока вода считает, каждый ребенок бежит в одно из $N$ мест, где можно спрятаться, выбирая каждое убежище равновероятно. Если оказывается, что более одного человека претендует на одно укрытие, то оно достается только одному, а остальные не успевают перепрятаться и проигрывают сразу же, как вода оборачивается. Вычислите матожидание числа детей, которые проигрывают таким обидным способом. Так как формула матожидания может получиться сложной, докажите, что эта формула неотрицательна.

\subsection{Экспедиция}
В экспедиции, которая должна продлиться $n$ дней, каждый день берется часть от продуктов и съедается. Так как начпрод в экспедиции не очень точен, эта часть есть случайная величина, равная в $k$-й день $X_k$. То есть если в начале $k$-ого дня в запасах было $N$ кг еды, то к началу следующего дня их будет $(1 - X_k)N$ кг. Матожидание $X_k$ в $k$-й день экспедиции есть $\frac{1}{n - k + 1}$, причем все $X_k$ независимы друг от друга. Определите матожидание доли еды, оставшейся к началу последнего дня экспедиции.

\subsection{Сколько было лампочек}
У Коли есть $N$ лампочек, время жизни которых следует распределению $\Exp(1)$. Как только одна лампочка перегорает, Коля моментально меняет ее на следующую. Последняя лампочка перегорела через время $t$ после начала работы первой лампочки, но к тому моменту Коля забыл, сколько у него было лампочек в самом начале. Правда Коля знает, что обычно, когда он покупает лампочки, то он покупает случайное число лампочек, выбирая его из геометрического распределения $\Geom(p)$. Найдите распределение $N$ с учетом того, что мы знаем, сколько суммарно проработали лампочки. 

\emph{Подсказка: почитайте про то, что такое гамма-распределение и как оно связано с экспоненциальным.}

\newpage
\section{Бородачев Сергей Игоревич}

\subsection{Независимость}
Есть три события $A$, $B$ и $C$. Известно, что $A$ и $B$ --- независимые, и $A$ и $C$ --- независимые. Также известно, что \[\Pr(A \cap B \cap C) = \Pr(A)\Pr(B)\Pr(C).\] Являются ли события $B$ и $C$ независимыми? Либо докажите независимость, либо приведите пример, в котором эти события будут зависимы.

\subsection{Игральная кость}Бросаем кость $d6$, пока не выпадет единица. Каково матожидание суммы всех сделанных бросков?

\subsection{Еще одна функция нормального распределения}
Пусть $X \sim N(0, 1)$ и 
\begin{align*}
    Y = \begin{cases}
        X, &\text{ если } |X| \le 1, \\
        -X, &\text{ иначе.}
    \end{cases}
\end{align*}
Найдите распределение $Y$ и $X + Y$.

\subsection{Сколько было лампочек}
У Коли есть $N$ лампочек, время жизни которых следует распределению $\Exp(1)$. Как только одна лампочка перегорает, Коля моментально меняет ее на следующую. Последняя лампочка перегорела через время $t$ после начала работы первой лампочки, но к тому моменту Коля забыл, сколько у него было лампочек в самом начале. Правда Коля знает, что обычно, когда он покупает лампочки, то он покупает случайное число лампочек, выбирая его из геометрического распределения $\Geom(p)$. Найдите распределение $N$ с учетом того, что мы знаем, сколько суммарно проработали лампочки. 

\emph{Подсказка: почитайте про то, что такое гамма-распределение и как оно связано с экспоненциальным.}

\newpage
\section{Васильев Алексей Георгиевич}

\subsection{Переливание крови}
Существует 4 группы крови, которые обозначаются как 0, А, В и АВ. Группу крови АВ можно переливать любым рецепиентам, группы А и В --- только рецепиентам с такой же группой, или с группой 0, а группу 0 можно переливать только рецепиентам с такой же группой крови. Распространение групп крови (по крайней мере в Петербурге) среди населения примерно следующее:
\begin{itemize}
    \item 0 --- 32\% населения,
    \item А --- 35\% населения,
    \item В --- 25\% населения,
    \item АВ --- 8\% населения.
\end{itemize}
Вычислите вероятность того, что хотя бы один из трех случайно выбранных доноров подойдет для переливания случайно выбранному человеку. 

\emph{Замечание: на возможность переливания влияет еще и резус-фактор: человеку с отрицательным р-ф можно переливать только кровь от донора с отрицательным р-ф. Поэтому для приближения задачи к более реальной можно считать, что р-ф рецепиента положительный. Хоть это условие и меняет распределение по группам крови, для упрощения расчетов будем считать, что распределение остается прежним.}

\subsection{Пока не будет слишком много брака}
Вероятность того, что станок произведет бракованную деталь равна $p$. Станку только что провели поверку, а следующую проведут после того, как он изготовит ровно $k$ бракованных деталей. Посчитайте матожидание числа \emph{небракованных} деталей, которые выпустит станок до следующей поверки.

\subsection{Ремонт}
Петя и Вася решили сделать ремонт (каждый в своем доме). Каждому из них надо нарезать 10 полосок обоев. Петя поступает следующим образом: он отрезает первую полоску обоев, а затем отмеряет длину новой полоски, прикладывая ее к предыдущей отрезанной. Вася же отмеряет каждую полоску, прикладывая к самой первой отрезанной полоске. Каждый раз разница длин новой полоски и той, по которой ее отмеряли, является случайной величиной, которая достаточно хорошо приближается нормальным распределением $N(0, 1)$ (если считать в сантиметрах). Определите:
\begin{enumerate}
    \item Дисперсию длины 10-ой отрезанной полоски у Пети и у Васи.
    \item Матожидание модуля разности длин 10-ой полоски Пети и 10-ой полоски Васи. 
\end{enumerate}

\subsection{Бросание кости}
Какое минимальное число бросков кости $d6$ надо совершить, чтобы сумма выпавших очков была больше 100 с вероятностью не меньше $\frac{1}{2}$? 

\emph{Подсказка: тут не нужны никакие разбиения на слагаемые, вопрос специально про вероятность 0.5 и число 100, которое не делится на 3.5. Зафиксируйте число бросков и разбейте все возможные исходы на кое-какие пары.}

\newpage
\section{Васильев Леонид Константинович}

\subsection{Про паучков}
Очень милая паучиха делает кладку из $n$ яиц с вероятностью $\frac{\lambda^n e^{-\lambda}}{n!}$, где $\lambda$ --- некоторое положительное число. К сожалению, каждый паучок, вылупившийся из кладки, выживает и доживает до репродуктивного возраста лишь с вероятностью $p$, независимо от своих сестер и братьев. Определите вероятность того, что из одной кладки выживет и разовьется ровно $k$ особей. Определите вероятность того, что было отложено $n$ яиц, если развилось ровно $k$ особей.

\subsection{Игральная кость}Бросаем кость $d6$, пока не выпадет единица. Каково матожидание суммы всех сделанных бросков?

\subsection{Экспоненциальная лампочка}
Время работы лампочки до того, как она перегорит, есть случайная величина, следующая экспоненциальному распределению с параметром $\lambda$. У вас есть $n$ таких лампочек, и как только одна перегорает, вы заменяете ее следующей. Определите матожидание и дисперсию времени, когда перегорит последняя лампочка.

\subsection{Сколько было лампочек}
У Коли есть $N$ лампочек, время жизни которых следует распределению $\Exp(1)$. Как только одна лампочка перегорает, Коля моментально меняет ее на следующую. Последняя лампочка перегорела через время $t$ после начала работы первой лампочки, но к тому моменту Коля забыл, сколько у него было лампочек в самом начале. Правда Коля знает, что обычно, когда он покупает лампочки, то он покупает случайное число лампочек, выбирая его из геометрического распределения $\Geom(p)$. Найдите распределение $N$ с учетом того, что мы знаем, сколько суммарно проработали лампочки. 

\emph{Подсказка: почитайте про то, что такое гамма-распределение и как оно связано с экспоненциальным.}

\newpage
\section{Воркожоков Денис Вадимович}

\subsection{Игра}
Есть корзина, в которой лежит $N$ шаров: $K$ черных и $M = N - K$ белых. Два игрока поочередно достают из корзины по шару. Выигрывает тот, кто первым достанет белый шар. Какова вероятность, что выиграет первый игрок? В итоговой формуле можно оставить только одну сумму и биномиальные коэффициенты. Посчитайте также эту вероятность точно для ситуации, когда белый шар единственный (то есть $M = 1$).

\subsection{Игральная кость}Бросаем кость $d6$, пока не выпадет единица. Каково матожидание суммы всех сделанных бросков?

\subsection{Еще одна функция нормального распределения}
Пусть $X \sim N(0, 1)$ и 
\begin{align*}
    Y = \begin{cases}
        X, &\text{ если } |X| \le 1, \\
        -X, &\text{ иначе.}
    \end{cases}
\end{align*}
Найдите распределение $Y$ и $X + Y$.

\subsection{Сколько было лампочек}
У Коли есть $N$ лампочек, время жизни которых следует распределению $\Exp(1)$. Как только одна лампочка перегорает, Коля моментально меняет ее на следующую. Последняя лампочка перегорела через время $t$ после начала работы первой лампочки, но к тому моменту Коля забыл, сколько у него было лампочек в самом начале. Правда Коля знает, что обычно, когда он покупает лампочки, то он покупает случайное число лампочек, выбирая его из геометрического распределения $\Geom(p)$. Найдите распределение $N$ с учетом того, что мы знаем, сколько суммарно проработали лампочки. 

\emph{Подсказка: почитайте про то, что такое гамма-распределение и как оно связано с экспоненциальным.}

\newpage
\section{Дювенжи Александр Николаевич}

\subsection{Переливание крови}
Существует 4 группы крови, которые обозначаются как 0, А, В и АВ. Группу крови АВ можно переливать любым рецепиентам, группы А и В --- только рецепиентам с такой же группой, или с группой 0, а группу 0 можно переливать только рецепиентам с такой же группой крови. Распространение групп крови (по крайней мере в Петербурге) среди населения примерно следующее:
\begin{itemize}
    \item 0 --- 32\% населения,
    \item А --- 35\% населения,
    \item В --- 25\% населения,
    \item АВ --- 8\% населения.
\end{itemize}
Вычислите вероятность того, что хотя бы один из трех случайно выбранных доноров подойдет для переливания случайно выбранному человеку. 

\emph{Замечание: на возможность переливания влияет еще и резус-фактор: человеку с отрицательным р-ф можно переливать только кровь от донора с отрицательным р-ф. Поэтому для приближения задачи к более реальной можно считать, что р-ф рецепиента положительный. Хоть это условие и меняет распределение по группам крови, для упрощения расчетов будем считать, что распределение остается прежним.}

\subsection{Пока не будет слишком много брака}
Вероятность того, что станок произведет бракованную деталь равна $p$. Станку только что провели поверку, а следующую проведут после того, как он изготовит ровно $k$ бракованных деталей. Посчитайте матожидание числа \emph{небракованных} деталей, которые выпустит станок до следующей поверки.

\subsection{Экспоненциальная лампочка}
Время работы лампочки до того, как она перегорит, есть случайная величина, следующая экспоненциальному распределению с параметром $\lambda$. У вас есть $n$ таких лампочек, и как только одна перегорает, вы заменяете ее следующей. Определите матожидание и дисперсию времени, когда перегорит последняя лампочка.

\subsection{Бросание кости}
Какое минимальное число бросков кости $d6$ надо совершить, чтобы сумма выпавших очков была больше 100 с вероятностью не меньше $\frac{1}{2}$? 

\emph{Подсказка: тут не нужны никакие разбиения на слагаемые, вопрос специально про вероятность 0.5 и число 100, которое не делится на 3.5. Зафиксируйте число бросков и разбейте все возможные исходы на кое-какие пары.}

\newpage
\section{Ешкин Даниил Сергеевич}

\subsection{Независимость}
Есть три события $A$, $B$ и $C$. Известно, что $A$ и $B$ --- независимые, и $A$ и $C$ --- независимые. Также известно, что \[\Pr(A \cap B \cap C) = \Pr(A)\Pr(B)\Pr(C).\] Являются ли события $B$ и $C$ независимыми? Либо докажите независимость, либо приведите пример, в котором эти события будут зависимы.

\subsection{Прятки}
$K$ детей играют в прятки (плюс один вода). Пока вода считает, каждый ребенок бежит в одно из $N$ мест, где можно спрятаться, выбирая каждое убежище равновероятно. Если оказывается, что более одного человека претендует на одно укрытие, то оно достается только одному, а остальные не успевают перепрятаться и проигрывают сразу же, как вода оборачивается. Вычислите матожидание числа детей, которые проигрывают таким обидным способом. Так как формула матожидания может получиться сложной, докажите, что эта формула неотрицательна.

\subsection{Еще одна функция нормального распределения}
Пусть $X \sim N(0, 1)$ и 
\begin{align*}
    Y = \begin{cases}
        X, &\text{ если } |X| \le 1, \\
        -X, &\text{ иначе.}
    \end{cases}
\end{align*}
Найдите распределение $Y$ и $X + Y$.

\subsection{Распад частицы}
Вы наблюдаете частицу, но точно не знаете, какую. У вас есть два варианта: либо это частица $A$, либо $B$, причем с равной вероятностью. Частица распадается через время $t$ после начала наблюдения. Вы знаете, что время распада частицы $A$ следует распределению $\Exp(\lambda_A)$, а время распада частицы $B$ --- распределению $\Exp(\lambda_B)$. Определите вероятность того, что наблюдалась частица $A$.

\newpage
\section{Казаков Михаил Вячеславович}

\subsection{Переливание крови}
Существует 4 группы крови, которые обозначаются как 0, А, В и АВ. Группу крови АВ можно переливать любым рецепиентам, группы А и В --- только рецепиентам с такой же группой, или с группой 0, а группу 0 можно переливать только рецепиентам с такой же группой крови. Распространение групп крови (по крайней мере в Петербурге) среди населения примерно следующее:
\begin{itemize}
    \item 0 --- 32\% населения,
    \item А --- 35\% населения,
    \item В --- 25\% населения,
    \item АВ --- 8\% населения.
\end{itemize}
Вычислите вероятность того, что хотя бы один из трех случайно выбранных доноров подойдет для переливания случайно выбранному человеку. 

\emph{Замечание: на возможность переливания влияет еще и резус-фактор: человеку с отрицательным р-ф можно переливать только кровь от донора с отрицательным р-ф. Поэтому для приближения задачи к более реальной можно считать, что р-ф рецепиента положительный. Хоть это условие и меняет распределение по группам крови, для упрощения расчетов будем считать, что распределение остается прежним.}

\subsection{Дисперсия независимых с.в.}
Докажите, что для двух независимых с.в. $X$ и $Y$ с конечными матожиданием и дисперсией выполняется равенство $\Var[XY] = \Var[X]\Var[Y]$ тогда и только тогда, когда выполнено одно из условий:
\begin{enumerate}
    \item $E[X] = E[Y] = 0$
    \item $X$ и $Y$ --- константы (то есть есть какие-то $x, y \in \R$ такие, что $\Pr[X = x \cap Y = y] = 1$)
    \item Одна из этих с.в. константа, равная нулю
\end{enumerate}

\subsection{Ремонт}
Петя и Вася решили сделать ремонт (каждый в своем доме). Каждому из них надо нарезать 10 полосок обоев. Петя поступает следующим образом: он отрезает первую полоску обоев, а затем отмеряет длину новой полоски, прикладывая ее к предыдущей отрезанной. Вася же отмеряет каждую полоску, прикладывая к самой первой отрезанной полоске. Каждый раз разница длин новой полоски и той, по которой ее отмеряли, является случайной величиной, которая достаточно хорошо приближается нормальным распределением $N(0, 1)$ (если считать в сантиметрах). Определите:
\begin{enumerate}
    \item Дисперсию длины 10-ой отрезанной полоски у Пети и у Васи.
    \item Матожидание модуля разности длин 10-ой полоски Пети и 10-ой полоски Васи. 
\end{enumerate}

\subsection{Бросание кости}
Какое минимальное число бросков кости $d6$ надо совершить, чтобы сумма выпавших очков была больше 100 с вероятностью не меньше $\frac{1}{2}$? 

\emph{Подсказка: тут не нужны никакие разбиения на слагаемые, вопрос специально про вероятность 0.5 и число 100, которое не делится на 3.5. Зафиксируйте число бросков и разбейте все возможные исходы на кое-какие пары.}

\newpage
\section{Клиначев Александр Викторович}

\subsection{Игра}
Есть корзина, в которой лежит $N$ шаров: $K$ черных и $M = N - K$ белых. Два игрока поочередно достают из корзины по шару. Выигрывает тот, кто первым достанет белый шар. Какова вероятность, что выиграет первый игрок? В итоговой формуле можно оставить только одну сумму и биномиальные коэффициенты. Посчитайте также эту вероятность точно для ситуации, когда белый шар единственный (то есть $M = 1$).

\subsection{Пока не будет слишком много брака}
Вероятность того, что станок произведет бракованную деталь равна $p$. Станку только что провели поверку, а следующую проведут после того, как он изготовит ровно $k$ бракованных деталей. Посчитайте матожидание числа \emph{небракованных} деталей, которые выпустит станок до следующей поверки.

\subsection{Ремонт}
Петя и Вася решили сделать ремонт (каждый в своем доме). Каждому из них надо нарезать 10 полосок обоев. Петя поступает следующим образом: он отрезает первую полоску обоев, а затем отмеряет длину новой полоски, прикладывая ее к предыдущей отрезанной. Вася же отмеряет каждую полоску, прикладывая к самой первой отрезанной полоске. Каждый раз разница длин новой полоски и той, по которой ее отмеряли, является случайной величиной, которая достаточно хорошо приближается нормальным распределением $N(0, 1)$ (если считать в сантиметрах). Определите:
\begin{enumerate}
    \item Дисперсию длины 10-ой отрезанной полоски у Пети и у Васи.
    \item Матожидание модуля разности длин 10-ой полоски Пети и 10-ой полоски Васи. 
\end{enumerate}

\subsection{Распад частицы}
Вы наблюдаете частицу, но точно не знаете, какую. У вас есть два варианта: либо это частица $A$, либо $B$, причем с равной вероятностью. Частица распадается через время $t$ после начала наблюдения. Вы знаете, что время распада частицы $A$ следует распределению $\Exp(\lambda_A)$, а время распада частицы $B$ --- распределению $\Exp(\lambda_B)$. Определите вероятность того, что наблюдалась частица $A$.

\newpage
\section{Козлов Михаил Александрович}

\subsection{Игра}
Есть корзина, в которой лежит $N$ шаров: $K$ черных и $M = N - K$ белых. Два игрока поочередно достают из корзины по шару. Выигрывает тот, кто первым достанет белый шар. Какова вероятность, что выиграет первый игрок? В итоговой формуле можно оставить только одну сумму и биномиальные коэффициенты. Посчитайте также эту вероятность точно для ситуации, когда белый шар единственный (то есть $M = 1$).

\subsection{Прятки}
$K$ детей играют в прятки (плюс один вода). Пока вода считает, каждый ребенок бежит в одно из $N$ мест, где можно спрятаться, выбирая каждое убежище равновероятно. Если оказывается, что более одного человека претендует на одно укрытие, то оно достается только одному, а остальные не успевают перепрятаться и проигрывают сразу же, как вода оборачивается. Вычислите матожидание числа детей, которые проигрывают таким обидным способом. Так как формула матожидания может получиться сложной, докажите, что эта формула неотрицательна.

\subsection{Еще одна функция нормального распределения}
Пусть $X \sim N(0, 1)$ и 
\begin{align*}
    Y = \begin{cases}
        X, &\text{ если } |X| \le 1, \\
        -X, &\text{ иначе.}
    \end{cases}
\end{align*}
Найдите распределение $Y$ и $X + Y$.

\subsection{Точки на окружности}
Точки $A_1, \dots, A_n$ расположены на окружности единичного радиуса, причем положение каждой точки не зависит от положения других и равномерно распределено по всей окружности. Упорядочим точки по часовой стрелке, начиная с $A_1$, получим последовательность $A_1, A_{(2)}, \dots, A_{(n)}$. Найти распределение длины дуги окружности между точками $A_1$ и $A_{(2)}$ (длина дуги считается по часовой стрелке).

\newpage
\section{Косогоров Евгений Михайлович}

\subsection{Случайная мутация}
Рассмотрим ген некоторой особи, который является строчкой из ноликов и единиц длины $n$. Так как нам ничего неизвестно про этот ген, мы считаем, что каждый его бит равновероятно и независимо от других бит может быть равен нулю или единице. Применим к этой особи стандартную битовую мутацию: каждый бит ее гена изменит свое значение с вероятностью $\frac{1}{n}$ и остается прежним с вероятностью $(1 - \frac{1}{n})$, где $n$ --- длина гена. Выяснилось, что полученная таким образом особь имеет ген, который состоит только из единиц, то есть является битовой строчкой $1^n$. Определите вероятность того, что в гене исходной особи, которую мы мутировали, был ровно один нолик.

\subsection{Прятки}
$K$ детей играют в прятки (плюс один вода). Пока вода считает, каждый ребенок бежит в одно из $N$ мест, где можно спрятаться, выбирая каждое убежище равновероятно. Если оказывается, что более одного человека претендует на одно укрытие, то оно достается только одному, а остальные не успевают перепрятаться и проигрывают сразу же, как вода оборачивается. Вычислите матожидание числа детей, которые проигрывают таким обидным способом. Так как формула матожидания может получиться сложной, докажите, что эта формула неотрицательна.

\subsection{Ремонт}
Петя и Вася решили сделать ремонт (каждый в своем доме). Каждому из них надо нарезать 10 полосок обоев. Петя поступает следующим образом: он отрезает первую полоску обоев, а затем отмеряет длину новой полоски, прикладывая ее к предыдущей отрезанной. Вася же отмеряет каждую полоску, прикладывая к самой первой отрезанной полоске. Каждый раз разница длин новой полоски и той, по которой ее отмеряли, является случайной величиной, которая достаточно хорошо приближается нормальным распределением $N(0, 1)$ (если считать в сантиметрах). Определите:
\begin{enumerate}
    \item Дисперсию длины 10-ой отрезанной полоски у Пети и у Васи.
    \item Матожидание модуля разности длин 10-ой полоски Пети и 10-ой полоски Васи. 
\end{enumerate}

\subsection{Бросание кости}
Какое минимальное число бросков кости $d6$ надо совершить, чтобы сумма выпавших очков была больше 100 с вероятностью не меньше $\frac{1}{2}$? 

\emph{Подсказка: тут не нужны никакие разбиения на слагаемые, вопрос специально про вероятность 0.5 и число 100, которое не делится на 3.5. Зафиксируйте число бросков и разбейте все возможные исходы на кое-какие пары.}

\newpage
\section{Криушенков Илья Сергеевич}

\subsection{Переливание крови}
Существует 4 группы крови, которые обозначаются как 0, А, В и АВ. Группу крови АВ можно переливать любым рецепиентам, группы А и В --- только рецепиентам с такой же группой, или с группой 0, а группу 0 можно переливать только рецепиентам с такой же группой крови. Распространение групп крови (по крайней мере в Петербурге) среди населения примерно следующее:
\begin{itemize}
    \item 0 --- 32\% населения,
    \item А --- 35\% населения,
    \item В --- 25\% населения,
    \item АВ --- 8\% населения.
\end{itemize}
Вычислите вероятность того, что хотя бы один из трех случайно выбранных доноров подойдет для переливания случайно выбранному человеку. 

\emph{Замечание: на возможность переливания влияет еще и резус-фактор: человеку с отрицательным р-ф можно переливать только кровь от донора с отрицательным р-ф. Поэтому для приближения задачи к более реальной можно считать, что р-ф рецепиента положительный. Хоть это условие и меняет распределение по группам крови, для упрощения расчетов будем считать, что распределение остается прежним.}

\subsection{Пока не будет слишком много брака}
Вероятность того, что станок произведет бракованную деталь равна $p$. Станку только что провели поверку, а следующую проведут после того, как он изготовит ровно $k$ бракованных деталей. Посчитайте матожидание числа \emph{небракованных} деталей, которые выпустит станок до следующей поверки.

\subsection{Экспедиция}
В экспедиции, которая должна продлиться $n$ дней, каждый день берется часть от продуктов и съедается. Так как начпрод в экспедиции не очень точен, эта часть есть случайная величина, равная в $k$-й день $X_k$. То есть если в начале $k$-ого дня в запасах было $N$ кг еды, то к началу следующего дня их будет $(1 - X_k)N$ кг. Матожидание $X_k$ в $k$-й день экспедиции есть $\frac{1}{n - k + 1}$, причем все $X_k$ независимы друг от друга. Определите матожидание доли еды, оставшейся к началу последнего дня экспедиции.

\subsection{Распад частицы}
Вы наблюдаете частицу, но точно не знаете, какую. У вас есть два варианта: либо это частица $A$, либо $B$, причем с равной вероятностью. Частица распадается через время $t$ после начала наблюдения. Вы знаете, что время распада частицы $A$ следует распределению $\Exp(\lambda_A)$, а время распада частицы $B$ --- распределению $\Exp(\lambda_B)$. Определите вероятность того, что наблюдалась частица $A$.

\newpage
\section{Кузин Максим Сергеевич}

\subsection{Независимость}
Есть три события $A$, $B$ и $C$. Известно, что $A$ и $B$ --- независимые, и $A$ и $C$ --- независимые. Также известно, что \[\Pr(A \cap B \cap C) = \Pr(A)\Pr(B)\Pr(C).\] Являются ли события $B$ и $C$ независимыми? Либо докажите независимость, либо приведите пример, в котором эти события будут зависимы.

\subsection{Игральная кость}Бросаем кость $d6$, пока не выпадет единица. Каково матожидание суммы всех сделанных бросков?

\subsection{Еще одна функция нормального распределения}
Пусть $X \sim N(0, 1)$ и 
\begin{align*}
    Y = \begin{cases}
        X, &\text{ если } |X| \le 1, \\
        -X, &\text{ иначе.}
    \end{cases}
\end{align*}
Найдите распределение $Y$ и $X + Y$.

\subsection{Сколько было лампочек}
У Коли есть $N$ лампочек, время жизни которых следует распределению $\Exp(1)$. Как только одна лампочка перегорает, Коля моментально меняет ее на следующую. Последняя лампочка перегорела через время $t$ после начала работы первой лампочки, но к тому моменту Коля забыл, сколько у него было лампочек в самом начале. Правда Коля знает, что обычно, когда он покупает лампочки, то он покупает случайное число лампочек, выбирая его из геометрического распределения $\Geom(p)$. Найдите распределение $N$ с учетом того, что мы знаем, сколько суммарно проработали лампочки. 

\emph{Подсказка: почитайте про то, что такое гамма-распределение и как оно связано с экспоненциальным.}

\newpage
\section{Купчик Антон Михайлович}

\subsection{Независимость}
Есть три события $A$, $B$ и $C$. Известно, что $A$ и $B$ --- независимые, и $A$ и $C$ --- независимые. Также известно, что \[\Pr(A \cap B \cap C) = \Pr(A)\Pr(B)\Pr(C).\] Являются ли события $B$ и $C$ независимыми? Либо докажите независимость, либо приведите пример, в котором эти события будут зависимы.

\subsection{Дисперсия независимых с.в.}
Докажите, что для двух независимых с.в. $X$ и $Y$ с конечными матожиданием и дисперсией выполняется равенство $\Var[XY] = \Var[X]\Var[Y]$ тогда и только тогда, когда выполнено одно из условий:
\begin{enumerate}
    \item $E[X] = E[Y] = 0$
    \item $X$ и $Y$ --- константы (то есть есть какие-то $x, y \in \R$ такие, что $\Pr[X = x \cap Y = y] = 1$)
    \item Одна из этих с.в. константа, равная нулю
\end{enumerate}

\subsection{Экспедиция}
В экспедиции, которая должна продлиться $n$ дней, каждый день берется часть от продуктов и съедается. Так как начпрод в экспедиции не очень точен, эта часть есть случайная величина, равная в $k$-й день $X_k$. То есть если в начале $k$-ого дня в запасах было $N$ кг еды, то к началу следующего дня их будет $(1 - X_k)N$ кг. Матожидание $X_k$ в $k$-й день экспедиции есть $\frac{1}{n - k + 1}$, причем все $X_k$ независимы друг от друга. Определите матожидание доли еды, оставшейся к началу последнего дня экспедиции.

\subsection{Бросание кости}
Какое минимальное число бросков кости $d6$ надо совершить, чтобы сумма выпавших очков была больше 100 с вероятностью не меньше $\frac{1}{2}$? 

\emph{Подсказка: тут не нужны никакие разбиения на слагаемые, вопрос специально про вероятность 0.5 и число 100, которое не делится на 3.5. Зафиксируйте число бросков и разбейте все возможные исходы на кое-какие пары.}

\newpage
\section{Мартынов Павел Михайлович}

\subsection{Игра}
Есть корзина, в которой лежит $N$ шаров: $K$ черных и $M = N - K$ белых. Два игрока поочередно достают из корзины по шару. Выигрывает тот, кто первым достанет белый шар. Какова вероятность, что выиграет первый игрок? В итоговой формуле можно оставить только одну сумму и биномиальные коэффициенты. Посчитайте также эту вероятность точно для ситуации, когда белый шар единственный (то есть $M = 1$).

\subsection{Прятки}
$K$ детей играют в прятки (плюс один вода). Пока вода считает, каждый ребенок бежит в одно из $N$ мест, где можно спрятаться, выбирая каждое убежище равновероятно. Если оказывается, что более одного человека претендует на одно укрытие, то оно достается только одному, а остальные не успевают перепрятаться и проигрывают сразу же, как вода оборачивается. Вычислите матожидание числа детей, которые проигрывают таким обидным способом. Так как формула матожидания может получиться сложной, докажите, что эта формула неотрицательна.

\subsection{Ремонт}
Петя и Вася решили сделать ремонт (каждый в своем доме). Каждому из них надо нарезать 10 полосок обоев. Петя поступает следующим образом: он отрезает первую полоску обоев, а затем отмеряет длину новой полоски, прикладывая ее к предыдущей отрезанной. Вася же отмеряет каждую полоску, прикладывая к самой первой отрезанной полоске. Каждый раз разница длин новой полоски и той, по которой ее отмеряли, является случайной величиной, которая достаточно хорошо приближается нормальным распределением $N(0, 1)$ (если считать в сантиметрах). Определите:
\begin{enumerate}
    \item Дисперсию длины 10-ой отрезанной полоски у Пети и у Васи.
    \item Матожидание модуля разности длин 10-ой полоски Пети и 10-ой полоски Васи. 
\end{enumerate}

\subsection{Сколько было лампочек}
У Коли есть $N$ лампочек, время жизни которых следует распределению $\Exp(1)$. Как только одна лампочка перегорает, Коля моментально меняет ее на следующую. Последняя лампочка перегорела через время $t$ после начала работы первой лампочки, но к тому моменту Коля забыл, сколько у него было лампочек в самом начале. Правда Коля знает, что обычно, когда он покупает лампочки, то он покупает случайное число лампочек, выбирая его из геометрического распределения $\Geom(p)$. Найдите распределение $N$ с учетом того, что мы знаем, сколько суммарно проработали лампочки. 

\emph{Подсказка: почитайте про то, что такое гамма-распределение и как оно связано с экспоненциальным.}

\newpage
\section{Мозжевилов Данил Дмитриевич}

\subsection{Независимость}
Есть три события $A$, $B$ и $C$. Известно, что $A$ и $B$ --- независимые, и $A$ и $C$ --- независимые. Также известно, что \[\Pr(A \cap B \cap C) = \Pr(A)\Pr(B)\Pr(C).\] Являются ли события $B$ и $C$ независимыми? Либо докажите независимость, либо приведите пример, в котором эти события будут зависимы.

\subsection{Прятки}
$K$ детей играют в прятки (плюс один вода). Пока вода считает, каждый ребенок бежит в одно из $N$ мест, где можно спрятаться, выбирая каждое убежище равновероятно. Если оказывается, что более одного человека претендует на одно укрытие, то оно достается только одному, а остальные не успевают перепрятаться и проигрывают сразу же, как вода оборачивается. Вычислите матожидание числа детей, которые проигрывают таким обидным способом. Так как формула матожидания может получиться сложной, докажите, что эта формула неотрицательна.

\subsection{Еще одна функция нормального распределения}
Пусть $X \sim N(0, 1)$ и 
\begin{align*}
    Y = \begin{cases}
        X, &\text{ если } |X| \le 1, \\
        -X, &\text{ иначе.}
    \end{cases}
\end{align*}
Найдите распределение $Y$ и $X + Y$.

\subsection{Бросание кости}
Какое минимальное число бросков кости $d6$ надо совершить, чтобы сумма выпавших очков была больше 100 с вероятностью не меньше $\frac{1}{2}$? 

\emph{Подсказка: тут не нужны никакие разбиения на слагаемые, вопрос специально про вероятность 0.5 и число 100, которое не делится на 3.5. Зафиксируйте число бросков и разбейте все возможные исходы на кое-какие пары.}

\newpage
\section{Морев Савва Игоревич}

\subsection{Независимость}
Есть три события $A$, $B$ и $C$. Известно, что $A$ и $B$ --- независимые, и $A$ и $C$ --- независимые. Также известно, что \[\Pr(A \cap B \cap C) = \Pr(A)\Pr(B)\Pr(C).\] Являются ли события $B$ и $C$ независимыми? Либо докажите независимость, либо приведите пример, в котором эти события будут зависимы.

\subsection{Дисперсия независимых с.в.}
Докажите, что для двух независимых с.в. $X$ и $Y$ с конечными матожиданием и дисперсией выполняется равенство $\Var[XY] = \Var[X]\Var[Y]$ тогда и только тогда, когда выполнено одно из условий:
\begin{enumerate}
    \item $E[X] = E[Y] = 0$
    \item $X$ и $Y$ --- константы (то есть есть какие-то $x, y \in \R$ такие, что $\Pr[X = x \cap Y = y] = 1$)
    \item Одна из этих с.в. константа, равная нулю
\end{enumerate}

\subsection{Ремонт}
Петя и Вася решили сделать ремонт (каждый в своем доме). Каждому из них надо нарезать 10 полосок обоев. Петя поступает следующим образом: он отрезает первую полоску обоев, а затем отмеряет длину новой полоски, прикладывая ее к предыдущей отрезанной. Вася же отмеряет каждую полоску, прикладывая к самой первой отрезанной полоске. Каждый раз разница длин новой полоски и той, по которой ее отмеряли, является случайной величиной, которая достаточно хорошо приближается нормальным распределением $N(0, 1)$ (если считать в сантиметрах). Определите:
\begin{enumerate}
    \item Дисперсию длины 10-ой отрезанной полоски у Пети и у Васи.
    \item Матожидание модуля разности длин 10-ой полоски Пети и 10-ой полоски Васи. 
\end{enumerate}

\subsection{Точки на окружности}
Точки $A_1, \dots, A_n$ расположены на окружности единичного радиуса, причем положение каждой точки не зависит от положения других и равномерно распределено по всей окружности. Упорядочим точки по часовой стрелке, начиная с $A_1$, получим последовательность $A_1, A_{(2)}, \dots, A_{(n)}$. Найти распределение длины дуги окружности между точками $A_1$ и $A_{(2)}$ (длина дуги считается по часовой стрелке).

\newpage
\section{Наумов Иван Леонидович}

\subsection{Независимость}
Есть три события $A$, $B$ и $C$. Известно, что $A$ и $B$ --- независимые, и $A$ и $C$ --- независимые. Также известно, что \[\Pr(A \cap B \cap C) = \Pr(A)\Pr(B)\Pr(C).\] Являются ли события $B$ и $C$ независимыми? Либо докажите независимость, либо приведите пример, в котором эти события будут зависимы.

\subsection{Пока не будет слишком много брака}
Вероятность того, что станок произведет бракованную деталь равна $p$. Станку только что провели поверку, а следующую проведут после того, как он изготовит ровно $k$ бракованных деталей. Посчитайте матожидание числа \emph{небракованных} деталей, которые выпустит станок до следующей поверки.

\subsection{Экспедиция}
В экспедиции, которая должна продлиться $n$ дней, каждый день берется часть от продуктов и съедается. Так как начпрод в экспедиции не очень точен, эта часть есть случайная величина, равная в $k$-й день $X_k$. То есть если в начале $k$-ого дня в запасах было $N$ кг еды, то к началу следующего дня их будет $(1 - X_k)N$ кг. Матожидание $X_k$ в $k$-й день экспедиции есть $\frac{1}{n - k + 1}$, причем все $X_k$ независимы друг от друга. Определите матожидание доли еды, оставшейся к началу последнего дня экспедиции.

\subsection{Распад частицы}
Вы наблюдаете частицу, но точно не знаете, какую. У вас есть два варианта: либо это частица $A$, либо $B$, причем с равной вероятностью. Частица распадается через время $t$ после начала наблюдения. Вы знаете, что время распада частицы $A$ следует распределению $\Exp(\lambda_A)$, а время распада частицы $B$ --- распределению $\Exp(\lambda_B)$. Определите вероятность того, что наблюдалась частица $A$.

\newpage
\section{Панов Иван Андреевич}

\subsection{Независимость}
Есть три события $A$, $B$ и $C$. Известно, что $A$ и $B$ --- независимые, и $A$ и $C$ --- независимые. Также известно, что \[\Pr(A \cap B \cap C) = \Pr(A)\Pr(B)\Pr(C).\] Являются ли события $B$ и $C$ независимыми? Либо докажите независимость, либо приведите пример, в котором эти события будут зависимы.

\subsection{Прятки}
$K$ детей играют в прятки (плюс один вода). Пока вода считает, каждый ребенок бежит в одно из $N$ мест, где можно спрятаться, выбирая каждое убежище равновероятно. Если оказывается, что более одного человека претендует на одно укрытие, то оно достается только одному, а остальные не успевают перепрятаться и проигрывают сразу же, как вода оборачивается. Вычислите матожидание числа детей, которые проигрывают таким обидным способом. Так как формула матожидания может получиться сложной, докажите, что эта формула неотрицательна.

\subsection{Экспоненциальная лампочка}
Время работы лампочки до того, как она перегорит, есть случайная величина, следующая экспоненциальному распределению с параметром $\lambda$. У вас есть $n$ таких лампочек, и как только одна перегорает, вы заменяете ее следующей. Определите матожидание и дисперсию времени, когда перегорит последняя лампочка.

\subsection{Сколько было лампочек}
У Коли есть $N$ лампочек, время жизни которых следует распределению $\Exp(1)$. Как только одна лампочка перегорает, Коля моментально меняет ее на следующую. Последняя лампочка перегорела через время $t$ после начала работы первой лампочки, но к тому моменту Коля забыл, сколько у него было лампочек в самом начале. Правда Коля знает, что обычно, когда он покупает лампочки, то он покупает случайное число лампочек, выбирая его из геометрического распределения $\Geom(p)$. Найдите распределение $N$ с учетом того, что мы знаем, сколько суммарно проработали лампочки. 

\emph{Подсказка: почитайте про то, что такое гамма-распределение и как оно связано с экспоненциальным.}

\newpage
\section{Степанов Семен Алексеевич}

\subsection{Независимость}
Есть три события $A$, $B$ и $C$. Известно, что $A$ и $B$ --- независимые, и $A$ и $C$ --- независимые. Также известно, что \[\Pr(A \cap B \cap C) = \Pr(A)\Pr(B)\Pr(C).\] Являются ли события $B$ и $C$ независимыми? Либо докажите независимость, либо приведите пример, в котором эти события будут зависимы.

\subsection{Дисперсия независимых с.в.}
Докажите, что для двух независимых с.в. $X$ и $Y$ с конечными матожиданием и дисперсией выполняется равенство $\Var[XY] = \Var[X]\Var[Y]$ тогда и только тогда, когда выполнено одно из условий:
\begin{enumerate}
    \item $E[X] = E[Y] = 0$
    \item $X$ и $Y$ --- константы (то есть есть какие-то $x, y \in \R$ такие, что $\Pr[X = x \cap Y = y] = 1$)
    \item Одна из этих с.в. константа, равная нулю
\end{enumerate}

\subsection{Экспоненциальная лампочка}
Время работы лампочки до того, как она перегорит, есть случайная величина, следующая экспоненциальному распределению с параметром $\lambda$. У вас есть $n$ таких лампочек, и как только одна перегорает, вы заменяете ее следующей. Определите матожидание и дисперсию времени, когда перегорит последняя лампочка.

\subsection{Распад частицы}
Вы наблюдаете частицу, но точно не знаете, какую. У вас есть два варианта: либо это частица $A$, либо $B$, причем с равной вероятностью. Частица распадается через время $t$ после начала наблюдения. Вы знаете, что время распада частицы $A$ следует распределению $\Exp(\lambda_A)$, а время распада частицы $B$ --- распределению $\Exp(\lambda_B)$. Определите вероятность того, что наблюдалась частица $A$.

\newpage
\section{Тушканова Анастасия Дмитриевна}

\subsection{Переливание крови}
Существует 4 группы крови, которые обозначаются как 0, А, В и АВ. Группу крови АВ можно переливать любым рецепиентам, группы А и В --- только рецепиентам с такой же группой, или с группой 0, а группу 0 можно переливать только рецепиентам с такой же группой крови. Распространение групп крови (по крайней мере в Петербурге) среди населения примерно следующее:
\begin{itemize}
    \item 0 --- 32\% населения,
    \item А --- 35\% населения,
    \item В --- 25\% населения,
    \item АВ --- 8\% населения.
\end{itemize}
Вычислите вероятность того, что хотя бы один из трех случайно выбранных доноров подойдет для переливания случайно выбранному человеку. 

\emph{Замечание: на возможность переливания влияет еще и резус-фактор: человеку с отрицательным р-ф можно переливать только кровь от донора с отрицательным р-ф. Поэтому для приближения задачи к более реальной можно считать, что р-ф рецепиента положительный. Хоть это условие и меняет распределение по группам крови, для упрощения расчетов будем считать, что распределение остается прежним.}

\subsection{Прятки}
$K$ детей играют в прятки (плюс один вода). Пока вода считает, каждый ребенок бежит в одно из $N$ мест, где можно спрятаться, выбирая каждое убежище равновероятно. Если оказывается, что более одного человека претендует на одно укрытие, то оно достается только одному, а остальные не успевают перепрятаться и проигрывают сразу же, как вода оборачивается. Вычислите матожидание числа детей, которые проигрывают таким обидным способом. Так как формула матожидания может получиться сложной, докажите, что эта формула неотрицательна.

\subsection{Еще одна функция нормального распределения}
Пусть $X \sim N(0, 1)$ и 
\begin{align*}
    Y = \begin{cases}
        X, &\text{ если } |X| \le 1, \\
        -X, &\text{ иначе.}
    \end{cases}
\end{align*}
Найдите распределение $Y$ и $X + Y$.

\subsection{Точки на окружности}
Точки $A_1, \dots, A_n$ расположены на окружности единичного радиуса, причем положение каждой точки не зависит от положения других и равномерно распределено по всей окружности. Упорядочим точки по часовой стрелке, начиная с $A_1$, получим последовательность $A_1, A_{(2)}, \dots, A_{(n)}$. Найти распределение длины дуги окружности между точками $A_1$ и $A_{(2)}$ (длина дуги считается по часовой стрелке).

\newpage
\section{Холявин Николай Андреевич}

\subsection{Переливание крови}
Существует 4 группы крови, которые обозначаются как 0, А, В и АВ. Группу крови АВ можно переливать любым рецепиентам, группы А и В --- только рецепиентам с такой же группой, или с группой 0, а группу 0 можно переливать только рецепиентам с такой же группой крови. Распространение групп крови (по крайней мере в Петербурге) среди населения примерно следующее:
\begin{itemize}
    \item 0 --- 32\% населения,
    \item А --- 35\% населения,
    \item В --- 25\% населения,
    \item АВ --- 8\% населения.
\end{itemize}
Вычислите вероятность того, что хотя бы один из трех случайно выбранных доноров подойдет для переливания случайно выбранному человеку. 

\emph{Замечание: на возможность переливания влияет еще и резус-фактор: человеку с отрицательным р-ф можно переливать только кровь от донора с отрицательным р-ф. Поэтому для приближения задачи к более реальной можно считать, что р-ф рецепиента положительный. Хоть это условие и меняет распределение по группам крови, для упрощения расчетов будем считать, что распределение остается прежним.}

\subsection{Дисперсия независимых с.в.}
Докажите, что для двух независимых с.в. $X$ и $Y$ с конечными матожиданием и дисперсией выполняется равенство $\Var[XY] = \Var[X]\Var[Y]$ тогда и только тогда, когда выполнено одно из условий:
\begin{enumerate}
    \item $E[X] = E[Y] = 0$
    \item $X$ и $Y$ --- константы (то есть есть какие-то $x, y \in \R$ такие, что $\Pr[X = x \cap Y = y] = 1$)
    \item Одна из этих с.в. константа, равная нулю
\end{enumerate}

\subsection{Экспоненциальная лампочка}
Время работы лампочки до того, как она перегорит, есть случайная величина, следующая экспоненциальному распределению с параметром $\lambda$. У вас есть $n$ таких лампочек, и как только одна перегорает, вы заменяете ее следующей. Определите матожидание и дисперсию времени, когда перегорит последняя лампочка.

\subsection{Распад частицы}
Вы наблюдаете частицу, но точно не знаете, какую. У вас есть два варианта: либо это частица $A$, либо $B$, причем с равной вероятностью. Частица распадается через время $t$ после начала наблюдения. Вы знаете, что время распада частицы $A$ следует распределению $\Exp(\lambda_A)$, а время распада частицы $B$ --- распределению $\Exp(\lambda_B)$. Определите вероятность того, что наблюдалась частица $A$.

\newpage
\section{Черемхина Татьяна Александровна}

\subsection{Переливание крови}
Существует 4 группы крови, которые обозначаются как 0, А, В и АВ. Группу крови АВ можно переливать любым рецепиентам, группы А и В --- только рецепиентам с такой же группой, или с группой 0, а группу 0 можно переливать только рецепиентам с такой же группой крови. Распространение групп крови (по крайней мере в Петербурге) среди населения примерно следующее:
\begin{itemize}
    \item 0 --- 32\% населения,
    \item А --- 35\% населения,
    \item В --- 25\% населения,
    \item АВ --- 8\% населения.
\end{itemize}
Вычислите вероятность того, что хотя бы один из трех случайно выбранных доноров подойдет для переливания случайно выбранному человеку. 

\emph{Замечание: на возможность переливания влияет еще и резус-фактор: человеку с отрицательным р-ф можно переливать только кровь от донора с отрицательным р-ф. Поэтому для приближения задачи к более реальной можно считать, что р-ф рецепиента положительный. Хоть это условие и меняет распределение по группам крови, для упрощения расчетов будем считать, что распределение остается прежним.}

\subsection{Дисперсия независимых с.в.}
Докажите, что для двух независимых с.в. $X$ и $Y$ с конечными матожиданием и дисперсией выполняется равенство $\Var[XY] = \Var[X]\Var[Y]$ тогда и только тогда, когда выполнено одно из условий:
\begin{enumerate}
    \item $E[X] = E[Y] = 0$
    \item $X$ и $Y$ --- константы (то есть есть какие-то $x, y \in \R$ такие, что $\Pr[X = x \cap Y = y] = 1$)
    \item Одна из этих с.в. константа, равная нулю
\end{enumerate}

\subsection{Экспедиция}
В экспедиции, которая должна продлиться $n$ дней, каждый день берется часть от продуктов и съедается. Так как начпрод в экспедиции не очень точен, эта часть есть случайная величина, равная в $k$-й день $X_k$. То есть если в начале $k$-ого дня в запасах было $N$ кг еды, то к началу следующего дня их будет $(1 - X_k)N$ кг. Матожидание $X_k$ в $k$-й день экспедиции есть $\frac{1}{n - k + 1}$, причем все $X_k$ независимы друг от друга. Определите матожидание доли еды, оставшейся к началу последнего дня экспедиции.

\subsection{Точки на окружности}
Точки $A_1, \dots, A_n$ расположены на окружности единичного радиуса, причем положение каждой точки не зависит от положения других и равномерно распределено по всей окружности. Упорядочим точки по часовой стрелке, начиная с $A_1$, получим последовательность $A_1, A_{(2)}, \dots, A_{(n)}$. Найти распределение длины дуги окружности между точками $A_1$ и $A_{(2)}$ (длина дуги считается по часовой стрелке).

\newpage

\end{document}