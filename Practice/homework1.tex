\documentclass[12pt]{article}
\usepackage[utf8]{inputenc} 
\usepackage[russian]{babel}
\usepackage{amssymb}
\usepackage{amsmath}


\title{Первое домашнее задание}

\begin{document}
\maketitle

\section{Сокращение вероятностного пространства}

Пусть есть какое-то вероятностное пространство $(\Omega, \Sigma, \Pr)$, и в нем есть событие $A \in \Sigma$, причем $\Pr(A) > 0$. Докажите, что на событии $A$ можно построить новое вероятностное пространство $(\Omega_A = A, \Sigma_A, \Pr_A)$, такое, чтобы для любых двух событий $B, C \subset A$, являющихся элементами $\Sigma$, было верно $\Pr_A(B \mid C) = \Pr(B \mid C)$.

Определить верность утверждений:
\begin{itemize}
    \item События $B, C \subset A$ --- независимы в пространстве $A$, тогда они независимы и в исходном пространстве.
    \item События $B, C \subset A$ --- независимы в исходном пространстве, тогда они независимы и в пространстве $A$.
    \item События $A, B, C \in \Sigma$ --- попарно независимы в исходном пространстве, тогда $A \cap B$ и $A \cap C$ независимы в пространстве $A$.
\end{itemize}  

\section{Разные кости}

Будем обозначать игральюную кость с $n$ гранями $dn$ (например, $d6, d12$). У нас есть 6 различных костей: $d2, d4, d6, d8, d12$ и $d20$. Кто-то выбирает одну из этих костей равновероятно и бросает. Мы не знаем, какую выбрали кость, но нам сообщают результат этого броска. Какое для нас будет распределение втрого броска (сделанного той же костью, что и первый бросок)? Каково математическое ожидание результата второго броска?

\section{Монетки и биномиальное распределение}

Честную монету бросают 15 раз. Посчитайте вероятность того, что среди первых 10 бросков строго больше пяти орлов, если известно, что среди последних 10 бросков строго больше пяти орлов.

\section{Парадокс двух конвертов без парадокса}

Человек играет в следующую игру. Перед ним кладут два конверта и сообщают, что в одном чек на сумму $2^n$ септимов, а в другом --- на сумму $2^{n + 1}$ (где $n$ --- целое неотрицательное число), но неизвестно, в каком конверте какой чек. Один конверт вскрывается, и игроку становится известна сумма на чеке в этом конверте. Игроку предлага.т забрать один из конвертов. С точки зрения максимизации выигрыша стоит ему взять открытый конверт или запечатанный, если

\begin{enumerate}
    \item Распределение пары конвертов $(2^n, 2^{n + 1})$ следует геометрическому распределению с вероятностью успеха $p$, то есть $\Pr((2^n, 2^{n + 1})) = p(1 - p)^n$.
    \item Распределение пары конвертов $(2^n, 2^{n + 1})$ следует степенному закону со степенью $2$, то есть $\Pr((2^n, 2^{n + 1})) = \frac{6}{\pi^2 (n + 1)^2}$.
\end{enumerate}

\end{document}