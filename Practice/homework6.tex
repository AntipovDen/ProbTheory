\documentclass[12pt]{article}
\usepackage[utf8]{inputenc} 
\usepackage[russian]{babel}
\usepackage{amssymb}
\usepackage{amsmath}

\newcommand\N{\mathbb{N}}
\newcommand\R{\mathbb{R}}
\newcommand\eps{\varepsilon}
\DeclareMathOperator{\Bin}{Bin}
\DeclareMathOperator{\Geom}{Geom}
\DeclareMathOperator{\Exp}{Exp}
\DeclareMathOperator{\pow}{pow}
\DeclareMathOperator{\Bern}{Bern}
\DeclareMathOperator{\Var}{Var}

\title{Шестое домашнее задание: непрерывные с.в.}

\begin{document}
\maketitle

\section{Десятичная запись числа}

Выбираем $X \sim U(0, 1)$, его точно можно записать как бесконечную десятичную дробь:
\begin{align*}
    X = \sum_{i = 1}^{+\infty} X_i 10^{-i},
\end{align*} 
где все $X_i \in [0..9]$. Докажите, что любая пара $X_i, X_j$ -- независимые с.в.

\section{Экспоненциальные с.в.}

Пусть есть $n$ независимых случайных величин $X_1, \dots, X_n$, и все они следуют $\Exp(\lambda)$, то есть $f_{X_i} = \lambda e^{-\lambda x}$.

Покажите, что у случайных величин $Y = \max_{i=1..n}\{X_i\}$ и $Z = X_1 + \frac{X_2}{2} + \dots + \frac{X_n}{n}$ одинаковое распределение.

\section{Неравенство Йенсена}

Докажите, что если функция $g(x)$ непрерывна и выпукла вверх, то для любой с.в. $X$ с конечным матожиданием верно, что
\begin{align*}
    g(E(X)) \ge E(g(x)), 
\end{align*} 
а если она выпукла вниз, то наоборот
\begin{align*}
    g(E(X)) \le E(g(x)), 
\end{align*} 

\section{Пара независимых с.в.}

Пусть с.в. $X$ и $Y$ независимы, и $Y$ имеет симметричное распределение (то есть функции распределения $Y$ и $-Y$ совпадают). Докажите, что для любого $r \in [1, 2]$ верно,что 
\begin{align*}
    E(|X + Y|^r) \le E(|X|^r) + E(|Y|^r),
\end{align*}
если все матожидания конечны.

\section{Независимые нормальные распределения}

Найдите распределение случайной величины $R = \sqrt{X^2 + Y^2}$, если известно, что $X$ и $Y$ следуют стандартному нормальному распределению $N(0, 1)$.

\section{Дискретная плюс непрерывная}

Пусть $X$ --- непрерывная с.в., а $Y$ --- просто какая-то с.в. Докажите, что $Z = X + Y$ является непрерывной с.в. (то есть имеет плотность вероятности)

\section{Два распада частицы}

Одна частица единичной массы распадается на две частицы массы $X$ и $1 - X$, причем $X$ -- с.в. с плотностью вероятности $f_X(x)$, такой, что
\begin{itemize}
    \item $f_X(x) = 0$, если $x \notin [0, 1]$
    \item $f_X(x) = f_X(1 - x)$ (симметрия).
\end{itemize}
Обозначим $X_1$ и $X_2$ --- массы меньшей и большей частицы соответственно. Каждая из этих частиц распадается по такому же закону, и у нас остается четыре частицы с массами
\begin{itemize}
    \item $X_{11}$ -- меньшая часть меньшей частицы.
    \item $X_{12}$ -- большая часть меньшей частицы.
    \item $X_{21}$ -- меньшая часть большей частицы.
    \item $X_{22}$ -- большая часть большей частицы.
\end{itemize}
Определите плотность вероятности $X_{11}$ и совместную плотность вероятности $X_{11}$ и $X_{22}$.

\section{Измерение скорости}

В России на дорогах не штрафуют при превышении скорости на не более, чем 20 км/ч. Оправдывается это тем, что у спидометра и у радара, которым измеряют скорость, есть погрешность, которая складывается. Допустим, что погрешность радара и погрешность спидометра следуют нормальному распределению $N(0, \sigma^2)$. Определите распределение показаний спидометра машины при известном показании радара.

\section{Ожидание на остановке}

Вы приходите на остановку маршрутки, которая ходит каждые 10 минут, но вы не можете знать, как давно уехала предыдущая маршрутка. Но вы видите, что еще $N$ человек уже ждут маршрутку. Определите плотность вероятности с.в. $T$, равной времени, через которое отправится следующая маршрутка, если известно, что число людей на остановке следует распределению Пуассона с параметром $t$, равным времени, которое прошло с отправления предыдущей маршрутки. То есть через $t$ минут после отправления предыдущего рейса $\Pr(N = n) = \frac{t^n}{n!}e^{-t}$.  

\end{document}