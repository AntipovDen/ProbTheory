\documentclass[12pt]{article}
\usepackage[utf8]{inputenc} 
\usepackage[russian]{babel}
\usepackage{amssymb}
\usepackage{amsmath}
\usepackage{a4wide}
\usepackage{hyperref}

\newcommand\N{\mathbb{N}}
\newcommand\R{\mathbb{R}}
\newcommand\eps{\varepsilon}
\DeclareMathOperator{\Bin}{Bin}
\DeclareMathOperator{\Geom}{Geom}
\DeclareMathOperator{\Exp}{Exp}
\DeclareMathOperator{\pow}{pow}
\DeclareMathOperator{\Bern}{Bern}
\DeclareMathOperator{\Var}{Var}
\DeclareMathOperator{\Cov}{Cov}

\begin{document}
\pagenumbering{gobble} 

\section{Белодедова Алина Сергеевна}

\subsection{Потомство}

У некоторого вида численность потомства одного рода следует геометрическому распределению с параметром $p$. При этом каждая особь является самцом или самкой с равной вероятностью. Определите распределение размера потомства $N$, если известно, что в нем было ровно $m$ самцов.



\subsection{Пустые корзины}

Раскидываем $K$ шаров по $N$ корзинам. Причем для каждого шара выбираем случайную корзину равновероятно. Каково матожидание и дисперсия числа непустых корзин?



\subsection{Метание диска}

Мы бросаем диск с радиусом 1 на бесконечную плоскость, на которой начерчена декартова система координат. Найдите матожидание числа точек с целочисленными координатами, которые закрывает диск.



\subsection{Какая была лампочка?}

У нас есть лампочка, время жизни которой следуют экспоненциальному распределению с неизвестным нам параметром $\lambda$. У нас не сохралась информация про лампочку, но мы знаем, что производятся эти лампочки с параметрами $\lambda = \frac{1}{i}$ для всех натуральных $i$. Так как лампочки с большим временем жизни стоят дороже, но и совсем короткоживущие лампочки мы покупать не любим, то вероятность того, что мы купили лампочку с параметром $\lambda$ равна $\frac{2^{1/\lambda}}{(1/\lambda)!e^2}$. Лампочка прожила время $t$. Какова вероятность того, что мы купили самую дешевую лампочку с $\lambda = 1$?




\subsection{Точки на отрезке}

На отрезке $[0, a]$ выбраны две случайные точки (согласно равномерному распределению). Найдите коэффициент корреляции координаты левой и правой точки. 



\subsection{Точное определение нечестности}

У нас есть нечестная монета с неизвестной нам вероятностью выпадения орла $q$. Мы бросаем ее $10^6$ раз, и у нас выпала ровно половина орлов. Мы хотим оценить $q$ с уверенностью в нашей оценке не меньше, чем $0.99$. Как мы можем оценить ее с максимальной точностью? 



\subsection{ЗБЧ для сумм}
Пусть $\{X_n\}_{n \in \N}$ --- последовательность независимых одинаково распределенных с.в. сс матожиданием $\mu$ и дисперсией $\sigma^2 \ne 0$. Пусть $S_n = X_1 + \dots + X_n$. Докажите, что для любой последовательности чисел $a_n = o(1/\sqrt{n})$ выполняется ЗБЧ последовательности $a_n S_n$.



\subsection{Азартный игрок}

Игрок играет в азартную игру с вероятнсотью выигрыша $\frac{1}{2}$. Причем в случае выигрыша, игроку возвращается его удвоенная ставка. Игрок играет, пока не разорится, а потом просто наблюдает за игрой. Докажите, что состояние игрока является мартингалом при любой стратегии ставок игрока.



\newpage
\section{Бородин Евгений Сергеевич}

\subsection{Дуэль}

Два игрока решили провести серию матчей. В среднем первый игрок выигрывал у второго в два раза больше раз, чем второй выигрывал у первого. То есть в одном матче вероятность победы первого есть $\frac{2}{3}$. Поэтому решили, что для победы первому игроку надо победить в 12 матчах, а второму --- в 6. Однако по техническим причинам пришлось завершить серию, когда счет был $8-4$ (надо ли уточнять, что в пользу первого игрока?). Было решено, что победа официально достанется тому, для кого вероятность итоговой победы на момент окончания серии была выше. Кто в итоге победил?



\subsection{Пока не будет слишком много брака}

Вероятность того, что станок произведет бракованную деталь равна $p$. Станку только что провели поверку, а следующую проведут после того, как он изготовит ровно $k$ бракованных деталей. Посчитайте дисперсию числа \emph{небракованных} деталей, которые выпустит станок до следующей поверки.



\subsection{Три случайных величины}

Пусть $X$, $Y$ и $Z$ --- независимые по совокупности с.в., следующие стандартному нормальному распределению $N(0, 1)$. Найдите матожидание и дисперсию с.в. $U = X + XY + XYZ$. 



\subsection{Какая была лампочка?}

У нас есть лампочка, время жизни которой следуют экспоненциальному распределению с неизвестным нам параметром $\lambda$. У нас не сохралась информация про лампочку, но мы знаем, что производятся эти лампочки с параметрами $\lambda = \frac{1}{i}$ для всех натуральных $i$. Так как лампочки с большим временем жизни стоят дороже, но и совсем короткоживущие лампочки мы покупать не любим, то вероятность того, что мы купили лампочку с параметром $\lambda$ равна $\frac{2^{1/\lambda}}{(1/\lambda)!e^2}$. Лампочка прожила время $t$. Какова вероятность того, что мы купили самую дешевую лампочку с $\lambda = 1$?




\subsection{Корреляция с квадратом}

Найти коэффициент корреляции между $X$ и $X^2$, если $X$ следует экспоненциальному распределению с параметром $\lambda$.



\subsection{Кот ученый}

Кот ученый ходит по цепи длиной в $n$ метров, начиная с ее середины. Пройдя каждый метр он случайным образом и равновероятно решает, в какую сторону ему двигаться дальше: вправо или влево. Оцените вероятность того, что он дойдет до края цепи, пройдя не более, чем $\frac{\sqrt{n \ln(n)}}{16}$ метров.



\subsection{Дроны}

Время полета дрона в часах следует экспоненциальному распределению с параметром $\lambda$. Какова вероятность того, что $n$ дронов суммарно налетают на $\frac{\sqrt{n}}{\lambda}$ больше часов, чем от них ожидается?



\subsection{Азартный игрок}

Игрок играет в азартную игру с вероятнсотью выигрыша $\frac{1}{2}$. Причем в случае выигрыша, игроку возвращается его удвоенная ставка. Игрок играет, пока не разорится, а потом просто наблюдает за игрой. Докажите, что состояние игрока является мартингалом при любой стратегии ставок игрока.



\newpage
\section{Вихнин Фёдор Алексеевич}

\subsection{Дуэль}

Два игрока решили провести серию матчей. В среднем первый игрок выигрывал у второго в два раза больше раз, чем второй выигрывал у первого. То есть в одном матче вероятность победы первого есть $\frac{2}{3}$. Поэтому решили, что для победы первому игроку надо победить в 12 матчах, а второму --- в 6. Однако по техническим причинам пришлось завершить серию, когда счет был $8-4$ (надо ли уточнять, что в пользу первого игрока?). Было решено, что победа официально достанется тому, для кого вероятность итоговой победы на момент окончания серии была выше. Кто в итоге победил?



\subsection{Игра в казино}

В казино решили ввести новую игру. Игрок платит первоначальный взнос, а потом кидает честную монету до тех пор, пока он не выбросит 2 орла подряд. Игрок забирает все монеты, выпавшие орлом, себе. Каким должен быть первоначальный взнос, чтобы казино не было в проигрыше?



\subsection{Определение зависимости}

Совместная плотность плотность вероятности двух случайных величин $X$ и $Y$ определена как
\begin{align*}
    f_{X, Y}(x, y) = \frac{c}{1 + x^2 + y^2 + x^2y^2},
\end{align*}
где $c$ --- нормирующая константа. Являются ли случайные величины независимыми?



\subsection{Какая была лампочка?}

У нас есть лампочка, время жизни которой следуют экспоненциальному распределению с неизвестным нам параметром $\lambda$. У нас не сохралась информация про лампочку, но мы знаем, что производятся эти лампочки с параметрами $\lambda = \frac{1}{i}$ для всех натуральных $i$. Так как лампочки с большим временем жизни стоят дороже, но и совсем короткоживущие лампочки мы покупать не любим, то вероятность того, что мы купили лампочку с параметром $\lambda$ равна $\frac{2^{1/\lambda}}{(1/\lambda)!e^2}$. Лампочка прожила время $t$. Какова вероятность того, что мы купили самую дешевую лампочку с $\lambda = 1$?




\subsection{Корреляция с квадратом}

Найти коэффициент корреляции между $X$ и $X^2$, если $X$ следует экспоненциальному распределению с параметром $\lambda$.



\subsection{Кот ученый}

Кот ученый ходит по цепи длиной в $n$ метров, начиная с ее середины. Пройдя каждый метр он случайным образом и равновероятно решает, в какую сторону ему двигаться дальше: вправо или влево. Оцените вероятность того, что он дойдет до края цепи, пройдя не более, чем $\frac{\sqrt{n \ln(n)}}{16}$ метров.



\subsection{Дроны}

Время полета дрона в часах следует экспоненциальному распределению с параметром $\lambda$. Какова вероятность того, что $n$ дронов суммарно налетают на $\frac{\sqrt{n}}{\lambda}$ больше часов, чем от них ожидается?



\subsection{Произведение с.в.}

Пусть $\{X_n\}_{n \in \N}$ --- последовательность с.в. с матожиданием $\mu$. Покажите, что последовательность $Y_n = \mu^{-n} \prod_{i = 1}^n X_i$ является мартингалом.



\newpage
\section{Гарипов Роман Исмагилович}

\subsection{Дуэль}

Два игрока решили провести серию матчей. В среднем первый игрок выигрывал у второго в два раза больше раз, чем второй выигрывал у первого. То есть в одном матче вероятность победы первого есть $\frac{2}{3}$. Поэтому решили, что для победы первому игроку надо победить в 12 матчах, а второму --- в 6. Однако по техническим причинам пришлось завершить серию, когда счет был $8-4$ (надо ли уточнять, что в пользу первого игрока?). Было решено, что победа официально достанется тому, для кого вероятность итоговой победы на момент окончания серии была выше. Кто в итоге победил?



\subsection{Игра в казино}

В казино решили ввести новую игру. Игрок платит первоначальный взнос, а потом кидает честную монету до тех пор, пока он не выбросит 2 орла подряд. Игрок забирает все монеты, выпавшие орлом, себе. Каким должен быть первоначальный взнос, чтобы казино не было в проигрыше?



\subsection{Три случайных величины}

Пусть $X$, $Y$ и $Z$ --- независимые по совокупности с.в., следующие стандартному нормальному распределению $N(0, 1)$. Найдите матожидание и дисперсию с.в. $U = X + XY + XYZ$. 



\subsection{Функция нормальных распределений}

Пусть $X$ и $Y$ --- независисмые с.в., следующие стандартному нормальному распределению $N(0, 1)$. Вычислите функцию распределения случайной величины $Z = X^2 + Y^2$.



\subsection{Точки на отрезке}

На отрезке $[0, a]$ выбраны две случайные точки (согласно равномерному распределению). Найдите коэффициент корреляции координаты левой и правой точки. 



\subsection{Точное определение нечестности}

У нас есть нечестная монета с неизвестной нам вероятностью выпадения орла $q$. Мы бросаем ее $10^6$ раз, и у нас выпала ровно половина орлов. Мы хотим оценить $q$ с уверенностью в нашей оценке не меньше, чем $0.99$. Как мы можем оценить ее с максимальной точностью? 



\subsection{Умножение на какие-то числа}

Пусть $\{X_n\}_{n \in \N}$ --- последовательность независимых одинаково распределенных с.в. сс матожиданием $\mu$ и дисперсией $\sigma^2$. Пусть $\{a_n\}_{n \in \N}$ --- ограниченная положительная последовательность чисел. Выполняется ли слабый закон больших чисел для последовательности с.в. $\{a_n X_n\}_{n \in \N}$?



\subsection{Нечестная игра}

Игрок играет в азартную игру с вероятнсотью выигрыша $\frac{1}{2}$. Причем в случае выигрыша, игроку возвращается его ставка, умноженная на $0.5$. Игрок играет, пока не разорится, а потом просто наблюдает за игрой. Докажите, что состояние игрока является супермартингалом при любой стратегии ставок игрока.



\newpage
\section{Гарипов Эмиль Исмагилович}

\subsection{Дуэль}

Два игрока решили провести серию матчей. В среднем первый игрок выигрывал у второго в два раза больше раз, чем второй выигрывал у первого. То есть в одном матче вероятность победы первого есть $\frac{2}{3}$. Поэтому решили, что для победы первому игроку надо победить в 12 матчах, а второму --- в 6. Однако по техническим причинам пришлось завершить серию, когда счет был $8-4$ (надо ли уточнять, что в пользу первого игрока?). Было решено, что победа официально достанется тому, для кого вероятность итоговой победы на момент окончания серии была выше. Кто в итоге победил?



\subsection{Пока не будет слишком много брака}

Вероятность того, что станок произведет бракованную деталь равна $p$. Станку только что провели поверку, а следующую проведут после того, как он изготовит ровно $k$ бракованных деталей. Посчитайте дисперсию числа \emph{небракованных} деталей, которые выпустит станок до следующей поверки.



\subsection{Определение зависимости}

Совместная плотность плотность вероятности двух случайных величин $X$ и $Y$ определена как
\begin{align*}
    f_{X, Y}(x, y) = \frac{c}{1 + x^2 + y^2 + x^2y^2},
\end{align*}
где $c$ --- нормирующая константа. Являются ли случайные величины независимыми?



\subsection{Какая была лампочка?}

У нас есть лампочка, время жизни которой следуют экспоненциальному распределению с неизвестным нам параметром $\lambda$. У нас не сохралась информация про лампочку, но мы знаем, что производятся эти лампочки с параметрами $\lambda = \frac{1}{i}$ для всех натуральных $i$. Так как лампочки с большим временем жизни стоят дороже, но и совсем короткоживущие лампочки мы покупать не любим, то вероятность того, что мы купили лампочку с параметром $\lambda$ равна $\frac{2^{1/\lambda}}{(1/\lambda)!e^2}$. Лампочка прожила время $t$. Какова вероятность того, что мы купили самую дешевую лампочку с $\lambda = 1$?




\subsection{Корреляция с квадратом}

Найти коэффициент корреляции между $X$ и $X^2$, если $X$ следует экспоненциальному распределению с параметром $\lambda$.



\subsection{Кот ученый}

Кот ученый ходит по цепи длиной в $n$ метров, начиная с ее середины. Пройдя каждый метр он случайным образом и равновероятно решает, в какую сторону ему двигаться дальше: вправо или влево. Оцените вероятность того, что он дойдет до края цепи, пройдя не более, чем $\frac{\sqrt{n \ln(n)}}{16}$ метров.



\subsection{Дроны}

Время полета дрона в часах следует экспоненциальному распределению с параметром $\lambda$. Какова вероятность того, что $n$ дронов суммарно налетают на $\frac{\sqrt{n}}{\lambda}$ больше часов, чем от них ожидается?



\subsection{Азартный игрок}

Игрок играет в азартную игру с вероятнсотью выигрыша $\frac{1}{2}$. Причем в случае выигрыша, игроку возвращается его удвоенная ставка. Игрок играет, пока не разорится, а потом просто наблюдает за игрой. Докажите, что состояние игрока является мартингалом при любой стратегии ставок игрока.



\newpage
\section{Гусев Владислав Сергеевич}

\subsection{Потеря}

В корзине есть $M$ белых шаров и $N - M$ черных. $r < N$ случайных шаров потерялись. Какова теперь вероятность вытащить белый шар?



\subsection{Игра в казино}

В казино решили ввести новую игру. Игрок платит первоначальный взнос, а потом кидает честную монету до тех пор, пока он не выбросит 2 орла подряд. Игрок забирает все монеты, выпавшие орлом, себе. Каким должен быть первоначальный взнос, чтобы казино не было в проигрыше?



\subsection{Определение зависимости}

Совместная плотность плотность вероятности двух случайных величин $X$ и $Y$ определена как
\begin{align*}
    f_{X, Y}(x, y) = \frac{c}{1 + x^2 + y^2 + x^2y^2},
\end{align*}
где $c$ --- нормирующая константа. Являются ли случайные величины независимыми?



\subsection{Нечестная монета}

Мы берем нечестную монету, у которой вероятность выпадания орла равна $X$. Мы не знаем, чему равен $X$ и предполагаем, что это какое-то случайное число из отрезка $[0, 1]$. Мы бросаем монету, пока не выпадет орел. Какое распределение у $X$, если мы совершили $n$ бросков?



\subsection{Точки на отрезке}

На отрезке $[0, a]$ выбраны две случайные точки (согласно равномерному распределению). Найдите коэффициент корреляции координаты левой и правой точки. 



\subsection{Кот ученый}

Кот ученый ходит по цепи длиной в $n$ метров, начиная с ее середины. Пройдя каждый метр он случайным образом и равновероятно решает, в какую сторону ему двигаться дальше: вправо или влево. Оцените вероятность того, что он дойдет до края цепи, пройдя не более, чем $\frac{\sqrt{n \ln(n)}}{16}$ метров.



\subsection{Умножение на какие-то числа}

Пусть $\{X_n\}_{n \in \N}$ --- последовательность независимых одинаково распределенных с.в. сс матожиданием $\mu$ и дисперсией $\sigma^2$. Пусть $\{a_n\}_{n \in \N}$ --- ограниченная положительная последовательность чисел. Выполняется ли слабый закон больших чисел для последовательности с.в. $\{a_n X_n\}_{n \in \N}$?



\subsection{Азартный игрок}

Игрок играет в азартную игру с вероятнсотью выигрыша $\frac{1}{2}$. Причем в случае выигрыша, игроку возвращается его удвоенная ставка. Игрок играет, пока не разорится, а потом просто наблюдает за игрой. Докажите, что состояние игрока является мартингалом при любой стратегии ставок игрока.



\newpage
\section{Давыдов Артём Вадимович}

\subsection{Потеря}

В корзине есть $M$ белых шаров и $N - M$ черных. $r < N$ случайных шаров потерялись. Какова теперь вероятность вытащить белый шар?



\subsection{Игра в казино}

В казино решили ввести новую игру. Игрок платит первоначальный взнос, а потом кидает честную монету до тех пор, пока он не выбросит 2 орла подряд. Игрок забирает все монеты, выпавшие орлом, себе. Каким должен быть первоначальный взнос, чтобы казино не было в проигрыше?



\subsection{Определение зависимости}

Совместная плотность плотность вероятности двух случайных величин $X$ и $Y$ определена как
\begin{align*}
    f_{X, Y}(x, y) = \frac{c}{1 + x^2 + y^2 + x^2y^2},
\end{align*}
где $c$ --- нормирующая константа. Являются ли случайные величины независимыми?



\subsection{Функция нормальных распределений}

Пусть $X$ и $Y$ --- независисмые с.в., следующие стандартному нормальному распределению $N(0, 1)$. Вычислите функцию распределения случайной величины $Z = X^2 + Y^2$.



\subsection{Корреляция с квадратом}

Найти коэффициент корреляции между $X$ и $X^2$, если $X$ следует экспоненциальному распределению с параметром $\lambda$.



\subsection{Кот ученый}

Кот ученый ходит по цепи длиной в $n$ метров, начиная с ее середины. Пройдя каждый метр он случайным образом и равновероятно решает, в какую сторону ему двигаться дальше: вправо или влево. Оцените вероятность того, что он дойдет до края цепи, пройдя не более, чем $\frac{\sqrt{n \ln(n)}}{16}$ метров.



\subsection{Дроны}

Время полета дрона в часах следует экспоненциальному распределению с параметром $\lambda$. Какова вероятность того, что $n$ дронов суммарно налетают на $\frac{\sqrt{n}}{\lambda}$ больше часов, чем от них ожидается?



\subsection{Произведение с.в.}

Пусть $\{X_n\}_{n \in \N}$ --- последовательность с.в. с матожиданием $\mu$. Покажите, что последовательность $Y_n = \mu^{-n} \prod_{i = 1}^n X_i$ является мартингалом.



\newpage
\section{Ибрахим Ахмад Махджуб}

\subsection{2 слона}

На две случайные клетки шахматной (8 на 8) доски ставят двух слонов. Какова вероятность, что они бьют друг друга?



\subsection{Пустые корзины}

Раскидываем $K$ шаров по $N$ корзинам. Причем для каждого шара выбираем случайную корзину равновероятно. Каково матожидание и дисперсия числа непустых корзин?



\subsection{Определение зависимости}

Совместная плотность плотность вероятности двух случайных величин $X$ и $Y$ определена как
\begin{align*}
    f_{X, Y}(x, y) = \frac{c}{1 + x^2 + y^2 + x^2y^2},
\end{align*}
где $c$ --- нормирующая константа. Являются ли случайные величины независимыми?



\subsection{Нечестная монета}

Мы берем нечестную монету, у которой вероятность выпадания орла равна $X$. Мы не знаем, чему равен $X$ и предполагаем, что это какое-то случайное число из отрезка $[0, 1]$. Мы бросаем монету, пока не выпадет орел. Какое распределение у $X$, если мы совершили $n$ бросков?



\subsection{Корреляция по функции распределения}

Две непрерывных с.в. $X$ и $Y$ принимают значения в квадрате $0 \le X \le \frac{\pi}{2},$ $0 \le Y \le \frac{\pi}{2}$. Причем их совместная функция распределения выглядит так:
\begin{align*}
    F_{X, Y} (x, y) = \sin x \sin y.
\end{align*}

Найдите коэффициент корреляции $X$ и $Y$.



\subsection{Кот ученый}

Кот ученый ходит по цепи длиной в $n$ метров, начиная с ее середины. Пройдя каждый метр он случайным образом и равновероятно решает, в какую сторону ему двигаться дальше: вправо или влево. Оцените вероятность того, что он дойдет до края цепи, пройдя не более, чем $\frac{\sqrt{n \ln(n)}}{16}$ метров.



\subsection{Сходящаяся последовательность}

Пусть $\{X_n\}_{n \in \N}$ --- последовательность независимых одинаково распределенных с.в. с матожиданием $\mu$ и дисперсией $\sigma^2$. Пусть 
\begin{align*}
    Y_n = \frac{X_1 + \dots + X_n}{X_1^2 + \dots + X_n^2}.
\end{align*}
Докажите, что $Y_n$ сходится по вероятности и найдите, к чему.



\subsection{Произведение с.в.}

Пусть $\{X_n\}_{n \in \N}$ --- последовательность с.в. с матожиданием $\mu$. Покажите, что последовательность $Y_n = \mu^{-n} \prod_{i = 1}^n X_i$ является мартингалом.



\newpage
\section{Кирсанов Ярослав Николаевич}

\subsection{Дуэль}

Два игрока решили провести серию матчей. В среднем первый игрок выигрывал у второго в два раза больше раз, чем второй выигрывал у первого. То есть в одном матче вероятность победы первого есть $\frac{2}{3}$. Поэтому решили, что для победы первому игроку надо победить в 12 матчах, а второму --- в 6. Однако по техническим причинам пришлось завершить серию, когда счет был $8-4$ (надо ли уточнять, что в пользу первого игрока?). Было решено, что победа официально достанется тому, для кого вероятность итоговой победы на момент окончания серии была выше. Кто в итоге победил?



\subsection{Пока не будет слишком много брака}

Вероятность того, что станок произведет бракованную деталь равна $p$. Станку только что провели поверку, а следующую проведут после того, как он изготовит ровно $k$ бракованных деталей. Посчитайте дисперсию числа \emph{небракованных} деталей, которые выпустит станок до следующей поверки.



\subsection{Три случайных величины}

Пусть $X$, $Y$ и $Z$ --- независимые по совокупности с.в., следующие стандартному нормальному распределению $N(0, 1)$. Найдите матожидание и дисперсию с.в. $U = X + XY + XYZ$. 



\subsection{Функция нормальных распределений}

Пусть $X$ и $Y$ --- независисмые с.в., следующие стандартному нормальному распределению $N(0, 1)$. Вычислите функцию распределения случайной величины $Z = X^2 + Y^2$.



\subsection{Точки на отрезке}

На отрезке $[0, a]$ выбраны две случайные точки (согласно равномерному распределению). Найдите коэффициент корреляции координаты левой и правой точки. 



\subsection{Кот ученый}

Кот ученый ходит по цепи длиной в $n$ метров, начиная с ее середины. Пройдя каждый метр он случайным образом и равновероятно решает, в какую сторону ему двигаться дальше: вправо или влево. Оцените вероятность того, что он дойдет до края цепи, пройдя не более, чем $\frac{\sqrt{n \ln(n)}}{16}$ метров.



\subsection{Дроны}

Время полета дрона в часах следует экспоненциальному распределению с параметром $\lambda$. Какова вероятность того, что $n$ дронов суммарно налетают на $\frac{\sqrt{n}}{\lambda}$ больше часов, чем от них ожидается?



\subsection{Произведение с.в.}

Пусть $\{X_n\}_{n \in \N}$ --- последовательность с.в. с матожиданием $\mu$. Покажите, что последовательность $Y_n = \mu^{-n} \prod_{i = 1}^n X_i$ является мартингалом.



\newpage
\section{Кучма Андрей Андреевич}

\subsection{Два геометрических распределения}

Пусть $X$ и $Y$ две независимые с.в., причем обе следуют геометрическому распределению с параметром $p$ $\Geom(p)$. Найдите вероятность того, что эти две с.в. отличаются не более, чем в 2 раза.



\subsection{Пустые корзины}

Раскидываем $K$ шаров по $N$ корзинам. Причем для каждого шара выбираем случайную корзину равновероятно. Каково матожидание и дисперсия числа непустых корзин?



\subsection{Метание диска}

Мы бросаем диск с радиусом 1 на бесконечную плоскость, на которой начерчена декартова система координат. Найдите матожидание числа точек с целочисленными координатами, которые закрывает диск.



\subsection{Функция нормальных распределений}

Пусть $X$ и $Y$ --- независисмые с.в., следующие стандартному нормальному распределению $N(0, 1)$. Вычислите функцию распределения случайной величины $Z = X^2 + Y^2$.



\subsection{Корреляция с квадратом}

Найти коэффициент корреляции между $X$ и $X^2$, если $X$ следует экспоненциальному распределению с параметром $\lambda$.



\subsection{Кот ученый}

Кот ученый ходит по цепи длиной в $n$ метров, начиная с ее середины. Пройдя каждый метр он случайным образом и равновероятно решает, в какую сторону ему двигаться дальше: вправо или влево. Оцените вероятность того, что он дойдет до края цепи, пройдя не более, чем $\frac{\sqrt{n \ln(n)}}{16}$ метров.



\subsection{Дроны}

Время полета дрона в часах следует экспоненциальному распределению с параметром $\lambda$. Какова вероятность того, что $n$ дронов суммарно налетают на $\frac{\sqrt{n}}{\lambda}$ больше часов, чем от них ожидается?



\subsection{Нечестная игра}

Игрок играет в азартную игру с вероятнсотью выигрыша $\frac{1}{2}$. Причем в случае выигрыша, игроку возвращается его ставка, умноженная на $0.5$. Игрок играет, пока не разорится, а потом просто наблюдает за игрой. Докажите, что состояние игрока является супермартингалом при любой стратегии ставок игрока.



\newpage
\section{Лабазов Артем Александрович}

\subsection{Два геометрических распределения}

Пусть $X$ и $Y$ две независимые с.в., причем обе следуют геометрическому распределению с параметром $p$ $\Geom(p)$. Найдите вероятность того, что эти две с.в. отличаются не более, чем в 2 раза.



\subsection{Пока не будет слишком много брака}

Вероятность того, что станок произведет бракованную деталь равна $p$. Станку только что провели поверку, а следующую проведут после того, как он изготовит ровно $k$ бракованных деталей. Посчитайте дисперсию числа \emph{небракованных} деталей, которые выпустит станок до следующей поверки.



\subsection{Три случайных величины}

Пусть $X$, $Y$ и $Z$ --- независимые по совокупности с.в., следующие стандартному нормальному распределению $N(0, 1)$. Найдите матожидание и дисперсию с.в. $U = X + XY + XYZ$. 



\subsection{Функция нормальных распределений}

Пусть $X$ и $Y$ --- независисмые с.в., следующие стандартному нормальному распределению $N(0, 1)$. Вычислите функцию распределения случайной величины $Z = X^2 + Y^2$.



\subsection{Корреляция по функции распределения}

Две непрерывных с.в. $X$ и $Y$ принимают значения в квадрате $0 \le X \le \frac{\pi}{2},$ $0 \le Y \le \frac{\pi}{2}$. Причем их совместная функция распределения выглядит так:
\begin{align*}
    F_{X, Y} (x, y) = \sin x \sin y.
\end{align*}

Найдите коэффициент корреляции $X$ и $Y$.



\subsection{Точное определение нечестности}

У нас есть нечестная монета с неизвестной нам вероятностью выпадения орла $q$. Мы бросаем ее $10^6$ раз, и у нас выпала ровно половина орлов. Мы хотим оценить $q$ с уверенностью в нашей оценке не меньше, чем $0.99$. Как мы можем оценить ее с максимальной точностью? 



\subsection{Сходящаяся последовательность}

Пусть $\{X_n\}_{n \in \N}$ --- последовательность независимых одинаково распределенных с.в. с матожиданием $\mu$ и дисперсией $\sigma^2$. Пусть 
\begin{align*}
    Y_n = \frac{X_1 + \dots + X_n}{X_1^2 + \dots + X_n^2}.
\end{align*}
Докажите, что $Y_n$ сходится по вероятности и найдите, к чему.



\subsection{Азартный игрок}

Игрок играет в азартную игру с вероятнсотью выигрыша $\frac{1}{2}$. Причем в случае выигрыша, игроку возвращается его удвоенная ставка. Игрок играет, пока не разорится, а потом просто наблюдает за игрой. Докажите, что состояние игрока является мартингалом при любой стратегии ставок игрока.



\newpage
\section{Малько Егор Александрович}

\subsection{2 слона}

На две случайные клетки шахматной (8 на 8) доски ставят двух слонов. Какова вероятность, что они бьют друг друга?



\subsection{Пока не будет слишком много брака}

Вероятность того, что станок произведет бракованную деталь равна $p$. Станку только что провели поверку, а следующую проведут после того, как он изготовит ровно $k$ бракованных деталей. Посчитайте дисперсию числа \emph{небракованных} деталей, которые выпустит станок до следующей поверки.



\subsection{Определение зависимости}

Совместная плотность плотность вероятности двух случайных величин $X$ и $Y$ определена как
\begin{align*}
    f_{X, Y}(x, y) = \frac{c}{1 + x^2 + y^2 + x^2y^2},
\end{align*}
где $c$ --- нормирующая константа. Являются ли случайные величины независимыми?



\subsection{Нечестная монета}

Мы берем нечестную монету, у которой вероятность выпадания орла равна $X$. Мы не знаем, чему равен $X$ и предполагаем, что это какое-то случайное число из отрезка $[0, 1]$. Мы бросаем монету, пока не выпадет орел. Какое распределение у $X$, если мы совершили $n$ бросков?



\subsection{Корреляция по функции распределения}

Две непрерывных с.в. $X$ и $Y$ принимают значения в квадрате $0 \le X \le \frac{\pi}{2},$ $0 \le Y \le \frac{\pi}{2}$. Причем их совместная функция распределения выглядит так:
\begin{align*}
    F_{X, Y} (x, y) = \sin x \sin y.
\end{align*}

Найдите коэффициент корреляции $X$ и $Y$.



\subsection{Точное определение нечестности}

У нас есть нечестная монета с неизвестной нам вероятностью выпадения орла $q$. Мы бросаем ее $10^6$ раз, и у нас выпала ровно половина орлов. Мы хотим оценить $q$ с уверенностью в нашей оценке не меньше, чем $0.99$. Как мы можем оценить ее с максимальной точностью? 



\subsection{ЗБЧ для сумм}
Пусть $\{X_n\}_{n \in \N}$ --- последовательность независимых одинаково распределенных с.в. сс матожиданием $\mu$ и дисперсией $\sigma^2 \ne 0$. Пусть $S_n = X_1 + \dots + X_n$. Докажите, что для любой последовательности чисел $a_n = o(1/\sqrt{n})$ выполняется ЗБЧ последовательности $a_n S_n$.



\subsection{Азартный игрок}

Игрок играет в азартную игру с вероятнсотью выигрыша $\frac{1}{2}$. Причем в случае выигрыша, игроку возвращается его удвоенная ставка. Игрок играет, пока не разорится, а потом просто наблюдает за игрой. Докажите, что состояние игрока является мартингалом при любой стратегии ставок игрока.



\newpage
\section{Мухамеджанов Салават Маратович}

\subsection{Дуэль}

Два игрока решили провести серию матчей. В среднем первый игрок выигрывал у второго в два раза больше раз, чем второй выигрывал у первого. То есть в одном матче вероятность победы первого есть $\frac{2}{3}$. Поэтому решили, что для победы первому игроку надо победить в 12 матчах, а второму --- в 6. Однако по техническим причинам пришлось завершить серию, когда счет был $8-4$ (надо ли уточнять, что в пользу первого игрока?). Было решено, что победа официально достанется тому, для кого вероятность итоговой победы на момент окончания серии была выше. Кто в итоге победил?



\subsection{Пока не будет слишком много брака}

Вероятность того, что станок произведет бракованную деталь равна $p$. Станку только что провели поверку, а следующую проведут после того, как он изготовит ровно $k$ бракованных деталей. Посчитайте дисперсию числа \emph{небракованных} деталей, которые выпустит станок до следующей поверки.



\subsection{Определение зависимости}

Совместная плотность плотность вероятности двух случайных величин $X$ и $Y$ определена как
\begin{align*}
    f_{X, Y}(x, y) = \frac{c}{1 + x^2 + y^2 + x^2y^2},
\end{align*}
где $c$ --- нормирующая константа. Являются ли случайные величины независимыми?



\subsection{Какая была лампочка?}

У нас есть лампочка, время жизни которой следуют экспоненциальному распределению с неизвестным нам параметром $\lambda$. У нас не сохралась информация про лампочку, но мы знаем, что производятся эти лампочки с параметрами $\lambda = \frac{1}{i}$ для всех натуральных $i$. Так как лампочки с большим временем жизни стоят дороже, но и совсем короткоживущие лампочки мы покупать не любим, то вероятность того, что мы купили лампочку с параметром $\lambda$ равна $\frac{2^{1/\lambda}}{(1/\lambda)!e^2}$. Лампочка прожила время $t$. Какова вероятность того, что мы купили самую дешевую лампочку с $\lambda = 1$?




\subsection{Корреляция по функции распределения}

Две непрерывных с.в. $X$ и $Y$ принимают значения в квадрате $0 \le X \le \frac{\pi}{2},$ $0 \le Y \le \frac{\pi}{2}$. Причем их совместная функция распределения выглядит так:
\begin{align*}
    F_{X, Y} (x, y) = \sin x \sin y.
\end{align*}

Найдите коэффициент корреляции $X$ и $Y$.



\subsection{Точное определение нечестности}

У нас есть нечестная монета с неизвестной нам вероятностью выпадения орла $q$. Мы бросаем ее $10^6$ раз, и у нас выпала ровно половина орлов. Мы хотим оценить $q$ с уверенностью в нашей оценке не меньше, чем $0.99$. Как мы можем оценить ее с максимальной точностью? 



\subsection{Умножение на какие-то числа}

Пусть $\{X_n\}_{n \in \N}$ --- последовательность независимых одинаково распределенных с.в. сс матожиданием $\mu$ и дисперсией $\sigma^2$. Пусть $\{a_n\}_{n \in \N}$ --- ограниченная положительная последовательность чисел. Выполняется ли слабый закон больших чисел для последовательности с.в. $\{a_n X_n\}_{n \in \N}$?



\subsection{Азартный игрок}

Игрок играет в азартную игру с вероятнсотью выигрыша $\frac{1}{2}$. Причем в случае выигрыша, игроку возвращается его удвоенная ставка. Игрок играет, пока не разорится, а потом просто наблюдает за игрой. Докажите, что состояние игрока является мартингалом при любой стратегии ставок игрока.



\newpage
\section{Надуткин Федор Максимович}

\subsection{2 слона}

На две случайные клетки шахматной (8 на 8) доски ставят двух слонов. Какова вероятность, что они бьют друг друга?



\subsection{Пустые корзины}

Раскидываем $K$ шаров по $N$ корзинам. Причем для каждого шара выбираем случайную корзину равновероятно. Каково матожидание и дисперсия числа непустых корзин?



\subsection{Три случайных величины}

Пусть $X$, $Y$ и $Z$ --- независимые по совокупности с.в., следующие стандартному нормальному распределению $N(0, 1)$. Найдите матожидание и дисперсию с.в. $U = X + XY + XYZ$. 



\subsection{Какая была лампочка?}

У нас есть лампочка, время жизни которой следуют экспоненциальному распределению с неизвестным нам параметром $\lambda$. У нас не сохралась информация про лампочку, но мы знаем, что производятся эти лампочки с параметрами $\lambda = \frac{1}{i}$ для всех натуральных $i$. Так как лампочки с большим временем жизни стоят дороже, но и совсем короткоживущие лампочки мы покупать не любим, то вероятность того, что мы купили лампочку с параметром $\lambda$ равна $\frac{2^{1/\lambda}}{(1/\lambda)!e^2}$. Лампочка прожила время $t$. Какова вероятность того, что мы купили самую дешевую лампочку с $\lambda = 1$?




\subsection{Корреляция с квадратом}

Найти коэффициент корреляции между $X$ и $X^2$, если $X$ следует экспоненциальному распределению с параметром $\lambda$.



\subsection{Кот ученый}

Кот ученый ходит по цепи длиной в $n$ метров, начиная с ее середины. Пройдя каждый метр он случайным образом и равновероятно решает, в какую сторону ему двигаться дальше: вправо или влево. Оцените вероятность того, что он дойдет до края цепи, пройдя не более, чем $\frac{\sqrt{n \ln(n)}}{16}$ метров.



\subsection{Дроны}

Время полета дрона в часах следует экспоненциальному распределению с параметром $\lambda$. Какова вероятность того, что $n$ дронов суммарно налетают на $\frac{\sqrt{n}}{\lambda}$ больше часов, чем от них ожидается?



\subsection{Нечестная игра}

Игрок играет в азартную игру с вероятнсотью выигрыша $\frac{1}{2}$. Причем в случае выигрыша, игроку возвращается его ставка, умноженная на $0.5$. Игрок играет, пока не разорится, а потом просто наблюдает за игрой. Докажите, что состояние игрока является супермартингалом при любой стратегии ставок игрока.



\newpage
\section{Нестеренко Виктор Евгеньевич}

\subsection{Два геометрических распределения}

Пусть $X$ и $Y$ две независимые с.в., причем обе следуют геометрическому распределению с параметром $p$ $\Geom(p)$. Найдите вероятность того, что эти две с.в. отличаются не более, чем в 2 раза.



\subsection{Пока не будет слишком много брака}

Вероятность того, что станок произведет бракованную деталь равна $p$. Станку только что провели поверку, а следующую проведут после того, как он изготовит ровно $k$ бракованных деталей. Посчитайте дисперсию числа \emph{небракованных} деталей, которые выпустит станок до следующей поверки.



\subsection{Метание диска}

Мы бросаем диск с радиусом 1 на бесконечную плоскость, на которой начерчена декартова система координат. Найдите матожидание числа точек с целочисленными координатами, которые закрывает диск.



\subsection{Какая была лампочка?}

У нас есть лампочка, время жизни которой следуют экспоненциальному распределению с неизвестным нам параметром $\lambda$. У нас не сохралась информация про лампочку, но мы знаем, что производятся эти лампочки с параметрами $\lambda = \frac{1}{i}$ для всех натуральных $i$. Так как лампочки с большим временем жизни стоят дороже, но и совсем короткоживущие лампочки мы покупать не любим, то вероятность того, что мы купили лампочку с параметром $\lambda$ равна $\frac{2^{1/\lambda}}{(1/\lambda)!e^2}$. Лампочка прожила время $t$. Какова вероятность того, что мы купили самую дешевую лампочку с $\lambda = 1$?




\subsection{Корреляция с квадратом}

Найти коэффициент корреляции между $X$ и $X^2$, если $X$ следует экспоненциальному распределению с параметром $\lambda$.



\subsection{Кот ученый}

Кот ученый ходит по цепи длиной в $n$ метров, начиная с ее середины. Пройдя каждый метр он случайным образом и равновероятно решает, в какую сторону ему двигаться дальше: вправо или влево. Оцените вероятность того, что он дойдет до края цепи, пройдя не более, чем $\frac{\sqrt{n \ln(n)}}{16}$ метров.



\subsection{Умножение на какие-то числа}

Пусть $\{X_n\}_{n \in \N}$ --- последовательность независимых одинаково распределенных с.в. сс матожиданием $\mu$ и дисперсией $\sigma^2$. Пусть $\{a_n\}_{n \in \N}$ --- ограниченная положительная последовательность чисел. Выполняется ли слабый закон больших чисел для последовательности с.в. $\{a_n X_n\}_{n \in \N}$?



\subsection{Произведение с.в.}

Пусть $\{X_n\}_{n \in \N}$ --- последовательность с.в. с матожиданием $\mu$. Покажите, что последовательность $Y_n = \mu^{-n} \prod_{i = 1}^n X_i$ является мартингалом.



\newpage
\section{Пак Александр Владимирович}

\subsection{Дуэль}

Два игрока решили провести серию матчей. В среднем первый игрок выигрывал у второго в два раза больше раз, чем второй выигрывал у первого. То есть в одном матче вероятность победы первого есть $\frac{2}{3}$. Поэтому решили, что для победы первому игроку надо победить в 12 матчах, а второму --- в 6. Однако по техническим причинам пришлось завершить серию, когда счет был $8-4$ (надо ли уточнять, что в пользу первого игрока?). Было решено, что победа официально достанется тому, для кого вероятность итоговой победы на момент окончания серии была выше. Кто в итоге победил?



\subsection{Пустые корзины}

Раскидываем $K$ шаров по $N$ корзинам. Причем для каждого шара выбираем случайную корзину равновероятно. Каково матожидание и дисперсия числа непустых корзин?



\subsection{Три случайных величины}

Пусть $X$, $Y$ и $Z$ --- независимые по совокупности с.в., следующие стандартному нормальному распределению $N(0, 1)$. Найдите матожидание и дисперсию с.в. $U = X + XY + XYZ$. 



\subsection{Функция нормальных распределений}

Пусть $X$ и $Y$ --- независисмые с.в., следующие стандартному нормальному распределению $N(0, 1)$. Вычислите функцию распределения случайной величины $Z = X^2 + Y^2$.



\subsection{Корреляция по функции распределения}

Две непрерывных с.в. $X$ и $Y$ принимают значения в квадрате $0 \le X \le \frac{\pi}{2},$ $0 \le Y \le \frac{\pi}{2}$. Причем их совместная функция распределения выглядит так:
\begin{align*}
    F_{X, Y} (x, y) = \sin x \sin y.
\end{align*}

Найдите коэффициент корреляции $X$ и $Y$.



\subsection{Точное определение нечестности}

У нас есть нечестная монета с неизвестной нам вероятностью выпадения орла $q$. Мы бросаем ее $10^6$ раз, и у нас выпала ровно половина орлов. Мы хотим оценить $q$ с уверенностью в нашей оценке не меньше, чем $0.99$. Как мы можем оценить ее с максимальной точностью? 



\subsection{Умножение на какие-то числа}

Пусть $\{X_n\}_{n \in \N}$ --- последовательность независимых одинаково распределенных с.в. сс матожиданием $\mu$ и дисперсией $\sigma^2$. Пусть $\{a_n\}_{n \in \N}$ --- ограниченная положительная последовательность чисел. Выполняется ли слабый закон больших чисел для последовательности с.в. $\{a_n X_n\}_{n \in \N}$?



\subsection{Произведение с.в.}

Пусть $\{X_n\}_{n \in \N}$ --- последовательность с.в. с матожиданием $\mu$. Покажите, что последовательность $Y_n = \mu^{-n} \prod_{i = 1}^n X_i$ является мартингалом.



\newpage
\section{Синяченко Никита Романович}

\subsection{Потеря}

В корзине есть $M$ белых шаров и $N - M$ черных. $r < N$ случайных шаров потерялись. Какова теперь вероятность вытащить белый шар?



\subsection{Пустые корзины}

Раскидываем $K$ шаров по $N$ корзинам. Причем для каждого шара выбираем случайную корзину равновероятно. Каково матожидание и дисперсия числа непустых корзин?



\subsection{Три случайных величины}

Пусть $X$, $Y$ и $Z$ --- независимые по совокупности с.в., следующие стандартному нормальному распределению $N(0, 1)$. Найдите матожидание и дисперсию с.в. $U = X + XY + XYZ$. 



\subsection{Нечестная монета}

Мы берем нечестную монету, у которой вероятность выпадания орла равна $X$. Мы не знаем, чему равен $X$ и предполагаем, что это какое-то случайное число из отрезка $[0, 1]$. Мы бросаем монету, пока не выпадет орел. Какое распределение у $X$, если мы совершили $n$ бросков?



\subsection{Корреляция по функции распределения}

Две непрерывных с.в. $X$ и $Y$ принимают значения в квадрате $0 \le X \le \frac{\pi}{2},$ $0 \le Y \le \frac{\pi}{2}$. Причем их совместная функция распределения выглядит так:
\begin{align*}
    F_{X, Y} (x, y) = \sin x \sin y.
\end{align*}

Найдите коэффициент корреляции $X$ и $Y$.



\subsection{Точное определение нечестности}

У нас есть нечестная монета с неизвестной нам вероятностью выпадения орла $q$. Мы бросаем ее $10^6$ раз, и у нас выпала ровно половина орлов. Мы хотим оценить $q$ с уверенностью в нашей оценке не меньше, чем $0.99$. Как мы можем оценить ее с максимальной точностью? 



\subsection{ЗБЧ для сумм}
Пусть $\{X_n\}_{n \in \N}$ --- последовательность независимых одинаково распределенных с.в. сс матожиданием $\mu$ и дисперсией $\sigma^2 \ne 0$. Пусть $S_n = X_1 + \dots + X_n$. Докажите, что для любой последовательности чисел $a_n = o(1/\sqrt{n})$ выполняется ЗБЧ последовательности $a_n S_n$.



\subsection{Нечестная игра}

Игрок играет в азартную игру с вероятнсотью выигрыша $\frac{1}{2}$. Причем в случае выигрыша, игроку возвращается его ставка, умноженная на $0.5$. Игрок играет, пока не разорится, а потом просто наблюдает за игрой. Докажите, что состояние игрока является супермартингалом при любой стратегии ставок игрока.



\newpage
\section{Сластин Александр Андреевич}

\subsection{Два геометрических распределения}

Пусть $X$ и $Y$ две независимые с.в., причем обе следуют геометрическому распределению с параметром $p$ $\Geom(p)$. Найдите вероятность того, что эти две с.в. отличаются не более, чем в 2 раза.



\subsection{Пустые корзины}

Раскидываем $K$ шаров по $N$ корзинам. Причем для каждого шара выбираем случайную корзину равновероятно. Каково матожидание и дисперсия числа непустых корзин?



\subsection{Метание диска}

Мы бросаем диск с радиусом 1 на бесконечную плоскость, на которой начерчена декартова система координат. Найдите матожидание числа точек с целочисленными координатами, которые закрывает диск.



\subsection{Нечестная монета}

Мы берем нечестную монету, у которой вероятность выпадания орла равна $X$. Мы не знаем, чему равен $X$ и предполагаем, что это какое-то случайное число из отрезка $[0, 1]$. Мы бросаем монету, пока не выпадет орел. Какое распределение у $X$, если мы совершили $n$ бросков?



\subsection{Корреляция с квадратом}

Найти коэффициент корреляции между $X$ и $X^2$, если $X$ следует экспоненциальному распределению с параметром $\lambda$.



\subsection{Кот ученый}

Кот ученый ходит по цепи длиной в $n$ метров, начиная с ее середины. Пройдя каждый метр он случайным образом и равновероятно решает, в какую сторону ему двигаться дальше: вправо или влево. Оцените вероятность того, что он дойдет до края цепи, пройдя не более, чем $\frac{\sqrt{n \ln(n)}}{16}$ метров.



\subsection{ЗБЧ для сумм}
Пусть $\{X_n\}_{n \in \N}$ --- последовательность независимых одинаково распределенных с.в. сс матожиданием $\mu$ и дисперсией $\sigma^2 \ne 0$. Пусть $S_n = X_1 + \dots + X_n$. Докажите, что для любой последовательности чисел $a_n = o(1/\sqrt{n})$ выполняется ЗБЧ последовательности $a_n S_n$.



\subsection{Нечестная игра}

Игрок играет в азартную игру с вероятнсотью выигрыша $\frac{1}{2}$. Причем в случае выигрыша, игроку возвращается его ставка, умноженная на $0.5$. Игрок играет, пока не разорится, а потом просто наблюдает за игрой. Докажите, что состояние игрока является супермартингалом при любой стратегии ставок игрока.



\newpage
\section{Ушков Даниил Анатольевич}

\subsection{Потомство}

У некоторого вида численность потомства одного рода следует геометрическому распределению с параметром $p$. При этом каждая особь является самцом или самкой с равной вероятностью. Определите распределение размера потомства $N$, если известно, что в нем было ровно $m$ самцов.



\subsection{Пока не будет слишком много брака}

Вероятность того, что станок произведет бракованную деталь равна $p$. Станку только что провели поверку, а следующую проведут после того, как он изготовит ровно $k$ бракованных деталей. Посчитайте дисперсию числа \emph{небракованных} деталей, которые выпустит станок до следующей поверки.



\subsection{Три случайных величины}

Пусть $X$, $Y$ и $Z$ --- независимые по совокупности с.в., следующие стандартному нормальному распределению $N(0, 1)$. Найдите матожидание и дисперсию с.в. $U = X + XY + XYZ$. 



\subsection{Функция нормальных распределений}

Пусть $X$ и $Y$ --- независисмые с.в., следующие стандартному нормальному распределению $N(0, 1)$. Вычислите функцию распределения случайной величины $Z = X^2 + Y^2$.



\subsection{Корреляция с квадратом}

Найти коэффициент корреляции между $X$ и $X^2$, если $X$ следует экспоненциальному распределению с параметром $\lambda$.



\subsection{Точное определение нечестности}

У нас есть нечестная монета с неизвестной нам вероятностью выпадения орла $q$. Мы бросаем ее $10^6$ раз, и у нас выпала ровно половина орлов. Мы хотим оценить $q$ с уверенностью в нашей оценке не меньше, чем $0.99$. Как мы можем оценить ее с максимальной точностью? 



\subsection{Умножение на какие-то числа}

Пусть $\{X_n\}_{n \in \N}$ --- последовательность независимых одинаково распределенных с.в. сс матожиданием $\mu$ и дисперсией $\sigma^2$. Пусть $\{a_n\}_{n \in \N}$ --- ограниченная положительная последовательность чисел. Выполняется ли слабый закон больших чисел для последовательности с.в. $\{a_n X_n\}_{n \in \N}$?



\subsection{Азартный игрок}

Игрок играет в азартную игру с вероятнсотью выигрыша $\frac{1}{2}$. Причем в случае выигрыша, игроку возвращается его удвоенная ставка. Игрок играет, пока не разорится, а потом просто наблюдает за игрой. Докажите, что состояние игрока является мартингалом при любой стратегии ставок игрока.



\newpage
\section{Черемхина Татьяна Александровна}

\subsection{Два геометрических распределения}

Пусть $X$ и $Y$ две независимые с.в., причем обе следуют геометрическому распределению с параметром $p$ $\Geom(p)$. Найдите вероятность того, что эти две с.в. отличаются не более, чем в 2 раза.



\subsection{Игра в казино}

В казино решили ввести новую игру. Игрок платит первоначальный взнос, а потом кидает честную монету до тех пор, пока он не выбросит 2 орла подряд. Игрок забирает все монеты, выпавшие орлом, себе. Каким должен быть первоначальный взнос, чтобы казино не было в проигрыше?



\subsection{Три случайных величины}

Пусть $X$, $Y$ и $Z$ --- независимые по совокупности с.в., следующие стандартному нормальному распределению $N(0, 1)$. Найдите матожидание и дисперсию с.в. $U = X + XY + XYZ$. 



\subsection{Какая была лампочка?}

У нас есть лампочка, время жизни которой следуют экспоненциальному распределению с неизвестным нам параметром $\lambda$. У нас не сохралась информация про лампочку, но мы знаем, что производятся эти лампочки с параметрами $\lambda = \frac{1}{i}$ для всех натуральных $i$. Так как лампочки с большим временем жизни стоят дороже, но и совсем короткоживущие лампочки мы покупать не любим, то вероятность того, что мы купили лампочку с параметром $\lambda$ равна $\frac{2^{1/\lambda}}{(1/\lambda)!e^2}$. Лампочка прожила время $t$. Какова вероятность того, что мы купили самую дешевую лампочку с $\lambda = 1$?




\subsection{Корреляция с квадратом}

Найти коэффициент корреляции между $X$ и $X^2$, если $X$ следует экспоненциальному распределению с параметром $\lambda$.



\subsection{Кот ученый}

Кот ученый ходит по цепи длиной в $n$ метров, начиная с ее середины. Пройдя каждый метр он случайным образом и равновероятно решает, в какую сторону ему двигаться дальше: вправо или влево. Оцените вероятность того, что он дойдет до края цепи, пройдя не более, чем $\frac{\sqrt{n \ln(n)}}{16}$ метров.



\subsection{ЗБЧ для сумм}
Пусть $\{X_n\}_{n \in \N}$ --- последовательность независимых одинаково распределенных с.в. сс матожиданием $\mu$ и дисперсией $\sigma^2 \ne 0$. Пусть $S_n = X_1 + \dots + X_n$. Докажите, что для любой последовательности чисел $a_n = o(1/\sqrt{n})$ выполняется ЗБЧ последовательности $a_n S_n$.



\subsection{Нечестная игра}

Игрок играет в азартную игру с вероятнсотью выигрыша $\frac{1}{2}$. Причем в случае выигрыша, игроку возвращается его ставка, умноженная на $0.5$. Игрок играет, пока не разорится, а потом просто наблюдает за игрой. Докажите, что состояние игрока является супермартингалом при любой стратегии ставок игрока.



\newpage
\section{Чернацкий Евгений Геннадьевич}

\subsection{Потомство}

У некоторого вида численность потомства одного рода следует геометрическому распределению с параметром $p$. При этом каждая особь является самцом или самкой с равной вероятностью. Определите распределение размера потомства $N$, если известно, что в нем было ровно $m$ самцов.



\subsection{Пока не будет слишком много брака}

Вероятность того, что станок произведет бракованную деталь равна $p$. Станку только что провели поверку, а следующую проведут после того, как он изготовит ровно $k$ бракованных деталей. Посчитайте дисперсию числа \emph{небракованных} деталей, которые выпустит станок до следующей поверки.



\subsection{Три случайных величины}

Пусть $X$, $Y$ и $Z$ --- независимые по совокупности с.в., следующие стандартному нормальному распределению $N(0, 1)$. Найдите матожидание и дисперсию с.в. $U = X + XY + XYZ$. 



\subsection{Функция нормальных распределений}

Пусть $X$ и $Y$ --- независисмые с.в., следующие стандартному нормальному распределению $N(0, 1)$. Вычислите функцию распределения случайной величины $Z = X^2 + Y^2$.



\subsection{Точки на отрезке}

На отрезке $[0, a]$ выбраны две случайные точки (согласно равномерному распределению). Найдите коэффициент корреляции координаты левой и правой точки. 



\subsection{Кот ученый}

Кот ученый ходит по цепи длиной в $n$ метров, начиная с ее середины. Пройдя каждый метр он случайным образом и равновероятно решает, в какую сторону ему двигаться дальше: вправо или влево. Оцените вероятность того, что он дойдет до края цепи, пройдя не более, чем $\frac{\sqrt{n \ln(n)}}{16}$ метров.



\subsection{ЗБЧ для сумм}
Пусть $\{X_n\}_{n \in \N}$ --- последовательность независимых одинаково распределенных с.в. сс матожиданием $\mu$ и дисперсией $\sigma^2 \ne 0$. Пусть $S_n = X_1 + \dots + X_n$. Докажите, что для любой последовательности чисел $a_n = o(1/\sqrt{n})$ выполняется ЗБЧ последовательности $a_n S_n$.



\subsection{Произведение с.в.}

Пусть $\{X_n\}_{n \in \N}$ --- последовательность с.в. с матожиданием $\mu$. Покажите, что последовательность $Y_n = \mu^{-n} \prod_{i = 1}^n X_i$ является мартингалом.



\newpage
\section{Шашуловский Артем Владимирович}

\subsection{Два геометрических распределения}

Пусть $X$ и $Y$ две независимые с.в., причем обе следуют геометрическому распределению с параметром $p$ $\Geom(p)$. Найдите вероятность того, что эти две с.в. отличаются не более, чем в 2 раза.



\subsection{Пока не будет слишком много брака}

Вероятность того, что станок произведет бракованную деталь равна $p$. Станку только что провели поверку, а следующую проведут после того, как он изготовит ровно $k$ бракованных деталей. Посчитайте дисперсию числа \emph{небракованных} деталей, которые выпустит станок до следующей поверки.



\subsection{Метание диска}

Мы бросаем диск с радиусом 1 на бесконечную плоскость, на которой начерчена декартова система координат. Найдите матожидание числа точек с целочисленными координатами, которые закрывает диск.



\subsection{Функция нормальных распределений}

Пусть $X$ и $Y$ --- независисмые с.в., следующие стандартному нормальному распределению $N(0, 1)$. Вычислите функцию распределения случайной величины $Z = X^2 + Y^2$.



\subsection{Корреляция с квадратом}

Найти коэффициент корреляции между $X$ и $X^2$, если $X$ следует экспоненциальному распределению с параметром $\lambda$.



\subsection{Кот ученый}

Кот ученый ходит по цепи длиной в $n$ метров, начиная с ее середины. Пройдя каждый метр он случайным образом и равновероятно решает, в какую сторону ему двигаться дальше: вправо или влево. Оцените вероятность того, что он дойдет до края цепи, пройдя не более, чем $\frac{\sqrt{n \ln(n)}}{16}$ метров.



\subsection{Умножение на какие-то числа}

Пусть $\{X_n\}_{n \in \N}$ --- последовательность независимых одинаково распределенных с.в. сс матожиданием $\mu$ и дисперсией $\sigma^2$. Пусть $\{a_n\}_{n \in \N}$ --- ограниченная положительная последовательность чисел. Выполняется ли слабый закон больших чисел для последовательности с.в. $\{a_n X_n\}_{n \in \N}$?



\subsection{Нечестная игра}

Игрок играет в азартную игру с вероятнсотью выигрыша $\frac{1}{2}$. Причем в случае выигрыша, игроку возвращается его ставка, умноженная на $0.5$. Игрок играет, пока не разорится, а потом просто наблюдает за игрой. Докажите, что состояние игрока является супермартингалом при любой стратегии ставок игрока.



\newpage
\section{Шик Алексей Александрович}

\subsection{Потеря}

В корзине есть $M$ белых шаров и $N - M$ черных. $r < N$ случайных шаров потерялись. Какова теперь вероятность вытащить белый шар?



\subsection{Пока не будет слишком много брака}

Вероятность того, что станок произведет бракованную деталь равна $p$. Станку только что провели поверку, а следующую проведут после того, как он изготовит ровно $k$ бракованных деталей. Посчитайте дисперсию числа \emph{небракованных} деталей, которые выпустит станок до следующей поверки.



\subsection{Метание диска}

Мы бросаем диск с радиусом 1 на бесконечную плоскость, на которой начерчена декартова система координат. Найдите матожидание числа точек с целочисленными координатами, которые закрывает диск.



\subsection{Какая была лампочка?}

У нас есть лампочка, время жизни которой следуют экспоненциальному распределению с неизвестным нам параметром $\lambda$. У нас не сохралась информация про лампочку, но мы знаем, что производятся эти лампочки с параметрами $\lambda = \frac{1}{i}$ для всех натуральных $i$. Так как лампочки с большим временем жизни стоят дороже, но и совсем короткоживущие лампочки мы покупать не любим, то вероятность того, что мы купили лампочку с параметром $\lambda$ равна $\frac{2^{1/\lambda}}{(1/\lambda)!e^2}$. Лампочка прожила время $t$. Какова вероятность того, что мы купили самую дешевую лампочку с $\lambda = 1$?




\subsection{Точки на отрезке}

На отрезке $[0, a]$ выбраны две случайные точки (согласно равномерному распределению). Найдите коэффициент корреляции координаты левой и правой точки. 



\subsection{Кот ученый}

Кот ученый ходит по цепи длиной в $n$ метров, начиная с ее середины. Пройдя каждый метр он случайным образом и равновероятно решает, в какую сторону ему двигаться дальше: вправо или влево. Оцените вероятность того, что он дойдет до края цепи, пройдя не более, чем $\frac{\sqrt{n \ln(n)}}{16}$ метров.



\subsection{ЗБЧ для сумм}
Пусть $\{X_n\}_{n \in \N}$ --- последовательность независимых одинаково распределенных с.в. сс матожиданием $\mu$ и дисперсией $\sigma^2 \ne 0$. Пусть $S_n = X_1 + \dots + X_n$. Докажите, что для любой последовательности чисел $a_n = o(1/\sqrt{n})$ выполняется ЗБЧ последовательности $a_n S_n$.



\subsection{Нечестная игра}

Игрок играет в азартную игру с вероятнсотью выигрыша $\frac{1}{2}$. Причем в случае выигрыша, игроку возвращается его ставка, умноженная на $0.5$. Игрок играет, пока не разорится, а потом просто наблюдает за игрой. Докажите, что состояние игрока является супермартингалом при любой стратегии ставок игрока.

\end{document}